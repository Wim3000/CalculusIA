\documentclass{ximera}

%\usepackage{todonotes}

\newcommand{\todo}{}

\usepackage{esint} % for \oiint
\graphicspath{
  {./}
  {ximeraTutorial/}
  {basicPhilosophy/}
  {functionsOfSeveralVariables/}
  {normalVectors/}
  {lagrangeMultipliers/}
  {vectorFields/}
  {greensTheorem/}
  {shapeOfThingsToCome/}
}
\usepackage{comment} %% used in what is a limit
\usepackage[valunder]{signchart} %% used in graphing sign chart

\newcommand{\mooculus}{\textsf{\textbf{MOOC}\textnormal{\textsf{ULUS}}}}

\usepackage{tkz-euclide}
\tikzset{>=stealth} %% cool arrow head
\tikzset{shorten <>/.style={ shorten >=#1, shorten <=#1 } } %% allows shorter vectors


\pgfplotsset{soldot/.style={color=black,only marks,mark=*}} %% USED by piecewise functions
\pgfplotsset{holdot/.style={color=black,fill=white,only marks,mark=*}}
\usetikzlibrary{arrows.meta}


\usepackage{tkz-tab}  %% sign charts
\usepackage{polynom}

\usetikzlibrary{backgrounds} %% for boxes around graphs
\usetikzlibrary{shapes,positioning}  %% Clouds and stars
\usetikzlibrary{matrix} %% for matrix
\usepgfplotslibrary{polar} %% for polar plots
%\usetkzobj{all}
\usepackage[makeroom]{cancel} %% for strike outs
%\usepackage{mathtools} %% for pretty underbrace % Breaks Ximera
\usepackage{multicol}
\usepackage{pgffor} %% required for integral for loops


%% http://tex.stackexchange.com/questions/66490/drawing-a-tikz-arc-specifying-the-center
%% Draws beach ball
\tikzset{pics/carc/.style args={#1:#2:#3}{code={\draw[pic actions] (#1:#3) arc(#1:#2:#3);}}}



\usepackage{array}
\setlength{\extrarowheight}{+.1cm}   
\newdimen\digitwidth
\settowidth\digitwidth{9}
\def\divrule#1#2{
\noalign{\moveright#1\digitwidth
\vbox{\hrule width#2\digitwidth}}}




\newcommand{\RR}{\mathbb R}
\newcommand{\R}{\mathbb R}
\newcommand{\N}{\mathbb N}
\newcommand{\Z}{\mathbb Z}

\newcommand{\sagemath}{\textsf{SageMath}}


%\renewcommand{\d}{\,d\!}
\renewcommand{\d}{\mathop{}\!d}
\newcommand{\dd}[2][]{\frac{\d #1}{\d #2}}
\newcommand{\pp}[2][]{\frac{\partial #1}{\partial #2}}
\renewcommand{\l}{\ell}
\newcommand{\ddx}{\frac{d}{\d x}}
\newcommand{\ddt}{\frac{d}{\d t}}


\newcommand{\zeroOverZero}{\ensuremath{\boldsymbol{\tfrac{0}{0}}}}
\newcommand{\inftyOverInfty}{\ensuremath{\boldsymbol{\tfrac{\infty}{\infty}}}}
\newcommand{\zeroOverInfty}{\ensuremath{\boldsymbol{\tfrac{0}{\infty}}}}
\newcommand{\zeroTimesInfty}{\ensuremath{\small\boldsymbol{0\cdot \infty}}}
\newcommand{\inftyMinusInfty}{\ensuremath{\small\boldsymbol{\infty - \infty}}}
\newcommand{\oneToInfty}{\ensuremath{\boldsymbol{1^\infty}}}
\newcommand{\zeroToZero}{\ensuremath{\boldsymbol{0^0}}}
\newcommand{\inftyToZero}{\ensuremath{\boldsymbol{\infty^0}}}



\newcommand{\numOverZero}{\ensuremath{\boldsymbol{\tfrac{\#}{0}}}}
\newcommand{\dfn}{\textbf}
%\newcommand{\unit}{\,\mathrm}
\newcommand{\unit}{\mathop{}\!\mathrm}
\newcommand{\eval}[1]{\bigg[ #1 \bigg]}
\newcommand{\seq}[1]{\left( #1 \right)}
\renewcommand{\epsilon}{\varepsilon}
\renewcommand{\phi}{\varphi}


\renewcommand{\iff}{\Leftrightarrow}

\DeclareMathOperator{\arccot}{arccot}
\DeclareMathOperator{\arcsec}{arcsec}
\DeclareMathOperator{\arccsc}{arccsc}
\DeclareMathOperator{\si}{Si}
\DeclareMathOperator{\proj}{\vec{proj}}
\DeclareMathOperator{\scal}{scal}
\DeclareMathOperator{\sign}{sign}


%% \newcommand{\tightoverset}[2]{% for arrow vec
%%   \mathop{#2}\limits^{\vbox to -.5ex{\kern-0.75ex\hbox{$#1$}\vss}}}
\newcommand{\arrowvec}{\overrightarrow}
%\renewcommand{\vec}[1]{\arrowvec{\mathbf{#1}}}
\renewcommand{\vec}{\mathbf}
\newcommand{\veci}{{\boldsymbol{\hat{\imath}}}}
\newcommand{\vecj}{{\boldsymbol{\hat{\jmath}}}}
\newcommand{\veck}{{\boldsymbol{\hat{k}}}}
\newcommand{\vecl}{\boldsymbol{\l}}
\newcommand{\uvec}[1]{\mathbf{\hat{#1}}}
\newcommand{\utan}{\mathbf{\hat{t}}}
\newcommand{\unormal}{\mathbf{\hat{n}}}
\newcommand{\ubinormal}{\mathbf{\hat{b}}}

\newcommand{\dotp}{\bullet}
\newcommand{\cross}{\boldsymbol\times}
\newcommand{\grad}{\boldsymbol\nabla}
\newcommand{\divergence}{\grad\dotp}
\newcommand{\curl}{\grad\cross}
%\DeclareMathOperator{\divergence}{divergence}
%\DeclareMathOperator{\curl}[1]{\grad\cross #1}
\newcommand{\lto}{\mathop{\longrightarrow\,}\limits}

\renewcommand{\bar}{\overline}

\colorlet{textColor}{black} 
\colorlet{background}{white}
\colorlet{penColor}{blue!50!black} % Color of a curve in a plot
\colorlet{penColor2}{red!50!black}% Color of a curve in a plot
\colorlet{penColor3}{red!50!blue} % Color of a curve in a plot
\colorlet{penColor4}{green!50!black} % Color of a curve in a plot
\colorlet{penColor5}{orange!80!black} % Color of a curve in a plot
\colorlet{penColor6}{yellow!70!black} % Color of a curve in a plot
\colorlet{fill1}{penColor!20} % Color of fill in a plot
\colorlet{fill2}{penColor2!20} % Color of fill in a plot
\colorlet{fillp}{fill1} % Color of positive area
\colorlet{filln}{penColor2!20} % Color of negative area
\colorlet{fill3}{penColor3!20} % Fill
\colorlet{fill4}{penColor4!20} % Fill
\colorlet{fill5}{penColor5!20} % Fill
\colorlet{gridColor}{gray!50} % Color of grid in a plot

\newcommand{\surfaceColor}{violet}
\newcommand{\surfaceColorTwo}{redyellow}
\newcommand{\sliceColor}{greenyellow}




\pgfmathdeclarefunction{gauss}{2}{% gives gaussian
  \pgfmathparse{1/(#2*sqrt(2*pi))*exp(-((x-#1)^2)/(2*#2^2))}%
}


%%%%%%%%%%%%%
%% Vectors
%%%%%%%%%%%%%

%% Simple horiz vectors
\renewcommand{\vector}[1]{\left\langle #1\right\rangle}


%% %% Complex Horiz Vectors with angle brackets
%% \makeatletter
%% \renewcommand{\vector}[2][ , ]{\left\langle%
%%   \def\nextitem{\def\nextitem{#1}}%
%%   \@for \el:=#2\do{\nextitem\el}\right\rangle%
%% }
%% \makeatother

%% %% Vertical Vectors
%% \def\vector#1{\begin{bmatrix}\vecListA#1,,\end{bmatrix}}
%% \def\vecListA#1,{\if,#1,\else #1\cr \expandafter \vecListA \fi}

%%%%%%%%%%%%%
%% End of vectors
%%%%%%%%%%%%%

%\newcommand{\fullwidth}{}
%\newcommand{\normalwidth}{}



%% makes a snazzy t-chart for evaluating functions
%\newenvironment{tchart}{\rowcolors{2}{}{background!90!textColor}\array}{\endarray}

%%This is to help with formatting on future title pages.
\newenvironment{sectionOutcomes}{}{} 



%% Flowchart stuff
%\tikzstyle{startstop} = [rectangle, rounded corners, minimum width=3cm, minimum height=1cm,text centered, draw=black]
%\tikzstyle{question} = [rectangle, minimum width=3cm, minimum height=1cm, text centered, draw=black]
%\tikzstyle{decision} = [trapezium, trapezium left angle=70, trapezium right angle=110, minimum width=3cm, minimum height=1cm, text centered, draw=black]
%\tikzstyle{question} = [rectangle, rounded corners, minimum width=3cm, minimum height=1cm,text centered, draw=black]
%\tikzstyle{process} = [rectangle, minimum width=3cm, minimum height=1cm, text centered, draw=black]
%\tikzstyle{decision} = [trapezium, trapezium left angle=70, trapezium right angle=110, minimum width=3cm, minimum height=1cm, text centered, draw=black]


\title{Local maxima and minima}

\outcome{Define a critical point.}
\outcome{Find critical points.}

\outcome{Define local maximum and local minimum.}

\outcome{Classify critical points.}
\outcome{State the First Derivative Test.}
\outcome{Apply the First Derivative Test.}

  
\begin{document}
\begin{abstract}
We use derivatives to help locate extrema.  
\end{abstract}
\maketitle


Whether we are interested in a function as a purely mathematical
object or in connection with some application to the real world, it is
often useful to know what the graph of the function looks like. We can
obtain a good picture of the graph using certain crucial information
provided by derivatives of the function.

\section{Extrema}

A local \textit{extremum} of a function $f$ is a point $(a,f(a))$ on the graph of  $f$ where the
$y$-coordinate is larger (or smaller) than all other $y$-coordinates
 of points on the graph whose $x$-coordinates are ``close to''  $a$. 

\begin{definition}\hfil\index{maximum/minimum!local}
\begin{enumerate}
\item A function $f$ has a \dfn{local maximum} at $x=a$, if $f(a)\ge
  f(x)$ for every $x$ in some open interval I containing $a$.
\item A function $f$ has a \dfn{local minimum} at $x=a$, if $f(a)\le
  f(x)$ for every $x$  in some open interval I containing $a$.
\end{enumerate}
A \dfn{local extremum}\index{extremum!local} is either a local
maximum or a local minimum.
\end{definition}

%% \begin{question} BADBAD I want this question
%%   True or false: ``All absolute extrema are also local extrema.''
%%   \begin{multipleChoice}
%%     \choice[correct]{true}
%%     \choice{false}
%%   \end{multipleChoice}
%%   \begin{feedback}
%%     All global extrema are local extrema.
%%   \end{feedback}
%% \end{question}
In our next example, we clarify the definition of a local minimum.
\begin{example}
Consider the graph of a function $f$:
\begin{image}
\begin{tikzpicture}
	\begin{axis}[
            domain=-3:2,
            ymax=1,
            ymin=-5.5,
            %samples=100,
            axis lines =middle, xlabel=$x$, ylabel=$y$,
            every axis y label/.style={at=(current axis.above origin),anchor=south},
            every axis x label/.style={at=(current axis.right of origin),anchor=west}
          ]
          \addplot [dashed, penColor2, smooth] plot coordinates {(1,0) (1,-5)}; %% {.451};
          \addplot [dashed, textColor, smooth] plot coordinates {(0.85,-0.6) (0.85,-4.75)}; %% axis{2.215};
           \addplot [dashed, textColor, smooth] plot coordinates {(0.85,0) (0.85,-0.21)}; %% axis{2.215};
       \addplot [ very thick, penColor, smooth] plot coordinates {(0.75,0) (1.25,0)}; 
          \addplot [very thick, penColor, smooth] {-2*x^3+9*x^2-12*x};
          \node at (axis cs:2.2,5/2) [anchor=west] {\color{penColor}$f$};  
              \node at (axis cs:0.6,-3) [anchor=west] {\color{penColor}$f(x)$};  
                \node at (axis cs:1,-3) [anchor=west] {\color{penColor2}$f(1)$};  
     \node at (axis cs:0.78,-0.35) [anchor=west] {\color{penColor}$x$}; 
       \node at (axis cs:0.9,0.6) [anchor=west] {\color{penColor}$I$};  
          \addplot[color=penColor2,fill=penColor2,only marks,mark=*] coordinates{(1,0)};  %% closed hole
        \addplot[color=penColor,fill=penColor,only marks,mark=*] coordinates{(0.85,0)};  %% closed hole
          \addplot[color=penColor,fill=penColor2,only marks,mark=*] coordinates{(1,-5)};  %% closed hole
          \addplot[color=penColor,fill=penColor,only marks,mark=*] coordinates{(0.85,-4.9)};  %% closed hole
              \addplot [color=penColor,fill=fill1,only marks,mark=*] coordinates{(0.75,0)};  %% open hole   
                 \addplot [color=penColor,fill=fill1,only marks,mark=*] coordinates{(1.25,0)};  %% open hole  
        \end{axis}
     
\end{tikzpicture}
%% \caption{A plot of $f(x) = x^3-4x^2+3x$ and $f'(x) = 3x^2-8x+3$.}
%% \label{figure:x^3-4x^2+3x}
\end{image}
Identify the local extrema of $f$ and give an explanation.
\begin{explanation}
In the figure above the function $f$ has a local minimum at,
\[
a=\answer[given]{1},
\]
because we can find an \textbf{open} interval $I$ (marked in the
figure) that contains the number
\[
a=\answer[given]{1},
\]
and for all numbers $x$ in $I$ the following statement is true:
 \[
f\left(\answer[given]{1}\right)\le f\left(\answer[given]{x}\right).
\]
\end{explanation}
\end{example}

Local maximum and minimum points are quite distinctive on the graph of
a function, and are, therefore, useful in understanding the shape of the
graph. Many problems in real world and in different scientific fields turn out to be about
finding the smallest (or largest) value that a function achieves (for example, we might want
to find the minimum cost at which some task can be performed).



\section{Critical points}

Consider the graph of the function $f$.
\begin{image}
\begin{tikzpicture}
	\begin{axis}[
            domain=-6:6, xmin=-6, xmax=6, ymin=-2,ymax=7,    
            unit vector ratio*=1 1 1,
         axis lines =center, xlabel=$x$, ylabel=$y$,
            every axis y label/.style={at=(current axis.above origin),anchor=south},
            every axis x label/.style={at=(current axis.right of origin),anchor=west},
            xtick={-6,...,6}, ytick={-3,...,10},
            xticklabels={-6,,-4,,-2,,0,,2,,4,,6}, yticklabels={,-2,,0,,2,,4,,6,,8,,10},
            grid=major,width=4in,
            grid style={dashed, gridColor},
          ]
          \addplot [very thick, penColor, smooth, domain=(-6:-4)] {x+9};
	  \addplot [very thick, penColor, smooth, domain=(-4:-2)] {1-x};
	  \addplot [very thick, penColor, smooth, domain=(-2:3)] {0.5*x^2-1};
          \addplot [very thick, penColor, smooth, domain=(3:4)] {(x-4)^3+4.5};
           \addplot [very thick, penColor, smooth, domain=(4:6)] {-(x-4)^2+4.5};
          \addplot[color=penColor,fill=background,only marks,mark=*] coordinates{(-2,3)};  %% open hole
          \addplot[color=penColor,fill=background,only marks,mark=*] coordinates{(-2,1)};  %% open hole
          \addplot[color=penColor,fill=penColor,only marks,mark=*] coordinates{(-2,6)};  %% closed hole
              \addplot [ very thick, penColor2, smooth] plot coordinates {(3.2,4.5) (4.8,4.5)}; 
                 \addplot [ very thick, penColor2, smooth] plot coordinates {(-0.75,-1) (0.75,-1)}; 
                   \node at (axis cs:1,6) [anchor=west] {\color{penColor}$y=f(x)$};  
        \end{axis}
\end{tikzpicture}
\end{image}
The function $f$  has four local extremums: at  $x=-4$, $x=-2$, $x=0$ and $x=4$.
Notice that the function $f$ is \textbf{not differentiable} at  $x=-4$ and  $x=-2$.
Notice that $f'(0)=0$ and $f'(4)=0$.

After this example, the following theorem should not come as a surprise. 


\begin{theorem}[Fermat's Theorem]\index{Fermat's Theorem}\label{theorem:fermat}
If $f$ has a local extremum at $x=a$ and $f$ is differentiable
at $a$, then $f'(a)=0$.
\end{theorem}
\begin{question}
  Does Fermat's Theorem say that if $f'(a) = 0$, then $f$ has a local
  extrema at $x=a$?
  \begin{multipleChoice}
    \choice{yes}
    \choice[correct]{no}
  \end{multipleChoice}
  \begin{feedback}
    Consider $f(x) = x^3$, $f'(0) = 0$, but $f$ does not have a local
    maximum or minimum at $x=0$.
  \end{feedback}
\end{question}


Fermat's Theorem says that the only points at which a function can
have a local maximum or minimum are points at which the derivative is
zero or the derivative is undefined. As an illustration of the first scenario, consider the plots of $f(x) = x^3-4.5x^2+6x$ and its derivative $f'(x) =
3x^2-9x+6$.
\begin{image}
\begin{tikzpicture}
	\begin{axis}[
            domain=-3:3,
            ymax=3,
            ymin=-3,
            %samples=100,
            axis lines =middle, xlabel=$x$, ylabel=$y$,
            every axis y label/.style={at=(current axis.above origin),anchor=south},
            every axis x label/.style={at=(current axis.right of origin),anchor=west}
          ]
          \addplot [dashed, textColor, smooth] plot coordinates {(1,0) (1,5/2)}; %% {.451};
          \addplot [dashed, textColor, smooth] plot coordinates {(2,0) (2,2)}; %% axis{2.215};
          \addplot [very thick, penColor2, smooth] {3*x^2-9*x+6};
          \addplot [very thick, penColor, smooth] {x^3-(9/2)*x^2+6*x};
          \node at (axis cs:2.2,5/2) [anchor=west] {\color{penColor}$f$};  
          \node at (axis cs:0.18,5/2) [anchor=west] {\color{penColor2}$f'$};
          \addplot[color=penColor2,fill=penColor2,only marks,mark=*] coordinates{(1,0)};  %% closed hole
          \addplot[color=penColor2,fill=penColor2,only marks,mark=*] coordinates{(2,0)};  %% closed hole
          \addplot[color=penColor,fill=penColor,only marks,mark=*] coordinates{(1,2.5)};  %% closed hole
          \addplot[color=penColor,fill=penColor,only marks,mark=*] coordinates{(2,2)};  %% closed hole
        \end{axis}
\end{tikzpicture}
%% \caption{A plot of $f(x) = x^3-4x^2+3x$ and $f'(x) = 3x^2-8x+3$.}
%% \label{figure:x^3-4x^2+3x}
\end{image}
\begin{question}
 Make a correct choice that completes the sentence below. \\
 
  At the point $(1,f(1))$, the  function $f$ has 

  \begin{multipleChoice}
    \choice[correct]{a local maximum}
    \choice{a local minimum}
    \choice{no local extremum}
  \end{multipleChoice}
  \end{question}
  \begin{question}
Select the correct statement.
  \begin{multipleChoice}
    \choice{$f'(1)$ is undefined}
    \choice{$f'(1)>0$}
    \choice[correct]{$f'(1)=0$}
     \choice{$f'(1)<0$}
  \end{multipleChoice}
\end{question}
\begin{question}
 Make a correct choice that completes the sentence below. \\
 
  At the point $(1.5,f(1.5))$, the  function $f$ has 

  \begin{multipleChoice}
    \choice{a local maximum}
    \choice{a local minimum}
    \choice[correct]{no local extremum}
  \end{multipleChoice}
  \end{question}
  \begin{question}
Select the correct statement.
  \begin{multipleChoice}
    \choice{$f'(1.5)$ is undefined}
    \choice{$f'(1.5)>0$}
    \choice{$f'(1.5)=0$}
     \choice[correct]{$f'(1.5)<0$}
  \end{multipleChoice}
\end{question}

\begin{question}
 Make a correct choice that completes the sentence below. \\
 
  At the point $(2,f(2))$, the  function $f$ has 

  \begin{multipleChoice}
    \choice{a local maximum}
    \choice[correct]{a local minimum}
    \choice{no local extremum}
  \end{multipleChoice}
  \end{question}
  \begin{question}
Select the correct statement.
  \begin{multipleChoice}
    \choice{$f'(2)$ is undefined}
    \choice{$f'(2)>0$}
    \choice[correct]{$f'(2)=0$}
     \choice{$f'(2)<0$}
  \end{multipleChoice}
\end{question}
 As an illustration of the second scenario, consider the plots of $f(x) = x^{2/3}$ and its derivative $f'(x) = \frac{2}{3x^{1/3}}$:
\begin{image}
\begin{tikzpicture}
	\begin{axis}[
            domain=-3:3,
            ymax=2,
            ymin=-2,
            axis lines =middle, xlabel=$x$, ylabel=$y$,
            every axis y label/.style={at=(current axis.above origin),anchor=south},
            every axis x label/.style={at=(current axis.right of origin),anchor=west}
          ]
          \addplot [very thick, penColor2, samples=100, smooth,domain=(-3:-.01)] {-(2/3)*abs(x)^(-1/3)};
          \addplot [very thick, penColor2, samples=100, smooth,domain=(.01:3)] {(2/3)*abs(x)^(-1/3)};
          \addplot [very thick, penColor, smooth,domain=(-3:-0.001)] {(abs(x))^(2/3)}; 
           \addplot [very thick, penColor, smooth,domain=(0.0015:3)] {x^(2/3)};         
          \node at (axis cs:-2,1.7) [anchor=west] {\color{penColor}$f$};  
          \node at (axis cs:2,.7) [anchor=west] {\color{penColor2}$f'$};
        \end{axis}
\end{tikzpicture}
%% \caption{A plot of $f(x) = x^{2/3}$ and $f'(x) = \frac{2}{3x^{1/3}}$.}
%% \label{figure:x^{2/3}}
\end{image}
\begin{question}
 Make a correct choice that completes the sentence below. \\
 
  At the point $(-2,f(-2))$, the  function $f$ has 

  \begin{multipleChoice}
    \choice{a local maximum}
    \choice{a local minimum}
    \choice[correct]{no local extremum}
  \end{multipleChoice}
  \end{question}
  \begin{question}
Select the correct statement.
  \begin{multipleChoice}
    \choice{$f'(-2)$ is undefined}
    \choice{$f'(-2)>0$}
    \choice{$f'(-2)=0$}
     \choice[correct]{$f'(-2)<0$}
  \end{multipleChoice}
\end{question}
\begin{question}
 Make a correct choice that completes the sentence below. \\
 
  At the point $(0,0)$, the  function $f$ has 

  \begin{multipleChoice}
    \choice{a local maximum}
    \choice[correct]{a local minimum}
    \choice{no local extremum}
  \end{multipleChoice}
  \end{question}
  \begin{question}
Select the correct statement.
  \begin{multipleChoice}
    \choice[correct]{$f'(0)$ is undefined}
    \choice{$f'(0)>0$}
    \choice{$f'(0)=0$}
     \choice{$f'(0)<0$}
  \end{multipleChoice}
\end{question}

This brings us to our next definition.

\begin{definition}\index{critical point}
  Assume that a function $f$ is defined on an open interval I that contains a point $a$. Then we say that the  function $f$ has a \dfn{critical point} at $x=a$ if 
  \[
  f'(a) = 0\qquad\text{or}\qquad \text{$f'(a)$ does not exist.}
  \]
 Notice, if a function $f$ has a critical point at $x=a$, then the number $a$ is inside some open interval $I$, and $I$ is in the  domain of $f$. 
 
\end{definition}

\begin{warning} 
When looking for local maximum and minimum points, be careful not to
make two sorts of mistakes: 
\begin{itemize}
\item You may forget that a maximum or minimum can occur where the
  derivative does not exist, and forget to check where the derivative exists. 
\item You might assume that any place that the derivative is zero is a
  local maximum or minimum point, but this is not true, consider the
  plots of $f(x) = x^3$ and $f'(x) = 3x^2$.
\begin{image}
\begin{tikzpicture}
	\begin{axis}[
            domain=-3:3,
            ymax=3,
            ymin=-3,
            axis lines =middle, xlabel=$x$, ylabel=$y$,
            every axis y label/.style={at=(current axis.above origin),anchor=south},
            every axis x label/.style={at=(current axis.right of origin),anchor=west}
          ]
          \addplot [very thick, penColor2, smooth] {3*x^2};
          \addplot [very thick, penColor, smooth] {x^3};         
          \node at (axis cs:1,.9) [anchor=west] {\color{penColor}$f$};  
          \node at (axis cs:-.5,1) [anchor=west] {\color{penColor2}$f'$};
        \end{axis}
\end{tikzpicture}
%% \caption{A plot of $f(x) = x^3$ and $f'(x) = 3x^2$. While $f'(0)=0$,
%%   there is neither a maximum nor minimum at $(0,f(0))$.}
%% \label{figure:x^3}
\end{image}
While $f'(0)=0$, there is neither a maximum nor minimum at $x=0$.
\end{itemize}
\end{warning}



Since both local maximum and
local minimum occur at a critical point, when we locate a critical point, we need to determine which, if either,
actually occurs. 
\begin{example}
Find all local maximum and minimum points for the function 
$f(x)=x^3-x$. 
\begin{explanation} 
Write
\[
\ddx f(x)=\answer[given]{3x^2-1}.
\] 
We want to determine when this equals $0$. We can express  $f'(x)$ as a product of its factors
\[
\ddx
f(x)=3\left(x+\answer[given]{\frac{\sqrt{3}}{3}}\right)\left(x-\answer[given]{\frac{\sqrt{3}}{3}}\right)=0,
\] 
solving we find the function $f$ has only two critical points,
$x=-\frac{\sqrt{3}}{3}$ and $x=\frac{\sqrt{3}}{3}$. Notice that the
derivative $f'(x)$ is a polynomial, and polynomials do not change
signs except possibly around their zeros. This implies that derivative
$f'(x)$ \textbf{does not change signs} on the intervals
$\left(-\infty,-\frac{\sqrt{3}}{3}\right)$,
$\left(-\frac{\sqrt{3}}{3},\frac{\sqrt{3}}{3}\right)$, and
$\left(\frac{\sqrt{3}}{3},\infty\right)$, because these intervals do not contain
any values that make $f'(x)=0$.
 
\begin{question}
Select the correct statement about the sign of $f'(x)$ on the
intervals $\left(-\infty,-\frac{\sqrt{3}}{3}\right)$ and
$\left(-\frac{\sqrt{3}}{3},\frac{\sqrt{3}}{3}\right)$. 
\begin{multipleChoice}
  \choice{$f'(x)>0$ on $\left(-\infty,-\frac{\sqrt{3}}{3}\right)$ and $f'(x)>0$ on $\left(-\frac{\sqrt{3}}{3},\frac{\sqrt{3}}{3}\right)$. }
  \choice[correct]{$f'(x)>0$ on  $\left(-\infty,-\frac{\sqrt{3}}{3}\right)$ and $f'(x)<0$ on $\left(-\frac{\sqrt{3}}{3},\frac{\sqrt{3}}{3}\right)$. }
  \choice{$f'(x)<0$ on $\left(-\infty,-\frac{\sqrt{3}}{3}\right)$ and $f'(x)>0$ on $\left(-\frac{\sqrt{3}}{3},\frac{\sqrt{3}}{3}\right)$. }
  \choice{$f'(x)<0$ on $\left(-\infty,-\frac{\sqrt{3}}{3}\right)$ and $f'(x)<0$ on $\left(-\frac{\sqrt{3}}{3},\frac{\sqrt{3}}{3}\right)$}
\end{multipleChoice}
  \end{question}
  If we know the sign of the derivative on an interval, we saw that in Section \ref{}, we know whether the function is increasing or decreasing on that interval. This will help us determine whether the function has a local extremum at the critical point  where $x=-\frac{\sqrt{3}}{3}$. \\
  \begin{question}

At the critical point where $x=-\frac{\sqrt{3}}{3}$, the function $f$ has  \\
 
  \begin{multipleChoice}
    \choice{no local extremum, because $f$ is increasing on $\left(-\infty,-\frac{\sqrt{3}}{3}\right)$ and increasing on $\left(-\frac{\sqrt{3}}{3},\frac{\sqrt{3}}{3}\right)$. }
      \choice[correct]{a local maximum, because $f$ is increasing on $\left(-\infty,-\frac{\sqrt{3}}{3}\right)$ and decreasing on $\left(-\frac{\sqrt{3}}{3},\frac{\sqrt{3}}{3}\right)$. }
      \choice{a local minimum, because $f$ is decreasing on $\left(-\infty,-\frac{\sqrt{3}}{3}\right)$ and increasing on $\left(-\frac{\sqrt{3}}{3},\frac{\sqrt{3}}{3}\right)$. }
         \choice{no local extremum, because $f$ is decreasing on $\left(-\infty,-\frac{\sqrt{3}}{3}\right)$ and decreasing on $\left(-\frac{\sqrt{3}}{3},\frac{\sqrt{3}}{3}\right)$. }  \end{multipleChoice}
  \end{question}
  \begin{question}
Select the correct  statement about the sign of $f'(x)$ on the intervals  $\left(-\frac{\sqrt{3}}{3},\frac{\sqrt{3}}{3}\right)$ and  $\left(\frac{\sqrt{3}}{3},\infty\right)$. \\
 
  \begin{multipleChoice}
    \choice{$f'(x)>0$ on $\left(-\frac{\sqrt{3}}{3},\frac{\sqrt{3}}{3}\right)$  and $f'(x)>0$ on $\left(\frac{\sqrt{3}}{3},\infty\right)$.}
    \choice{$f'(x)>0$ on $\left(-\frac{\sqrt{3}}{3},\frac{\sqrt{3}}{3}\right)$ and $f'(x)<0$ on $\left(\frac{\sqrt{3}}{3},\infty\right)$.}
       \choice[correct]{$f'(x)<0$ on   $\left(-\frac{\sqrt{3}}{3},\frac{\sqrt{3}}{3}\right)$  and $f'(x)>0$ on $\left(\frac{\sqrt{3}}{3},\infty\right)$.}
         \choice{$f'(x)<0$ on  $\left(-\frac{\sqrt{3}}{3},\frac{\sqrt{3}}{3}\right)$  and $f'(x)<0$ on $\left(\frac{\sqrt{3}}{3},\infty\right)$.}
  \end{multipleChoice}
  \end{question}
Again, the sign of the derivative on an interval determines whether the function is increasing or decreasing on that interval. This will help us determine whether the function has a local extremum at the critical point  where $x=\frac{\sqrt{3}}{3}$. \\
  \begin{question}

At the critical point where $x=\frac{\sqrt{3}}{3}$, the function $f$ has  \\
 
  \begin{multipleChoice}
    \choice{no local extremum, because $f$ is increasing on $\left(-\frac{\sqrt{3}}{3},\frac{\sqrt{3}}{3}\right)$ and increasing on $\left(\frac{\sqrt{3}}{3},\infty\right)$. }
       \choice{a local maximum, because $f$ is increasing on $\left(-\frac{\sqrt{3}}{3},\frac{\sqrt{3}}{3}\right)$ and decreasing on $\left(\frac{\sqrt{3}}{3},\infty\right)$. }
      \choice[correct]{a local minimum, because $f$ is decreasing on $\left(-\frac{\sqrt{3}}{3},\frac{\sqrt{3}}{3}\right)$ and increasing on $\left(\frac{\sqrt{3}}{3},\infty\right)$. }
        \choice{no local extremum, because $f$ is decreasing on$\left(-\frac{\sqrt{3}}{3},\frac{\sqrt{3}}{3}\right)$ and decreasing on $\left(\frac{\sqrt{3}}{3},\infty\right)$. }
      \end{multipleChoice}
  \end{question}
 Do your answers agree with the graphs of $f$ and $f'$ given in the picture below?
\begin{image}
\begin{tikzpicture}
	\begin{axis}[
            domain=-2:2,
            ymax=2,
            ymin=-2,
            %samples=100,
            axis lines =middle, xlabel=$x$, ylabel=$y$,
            every axis y label/.style={at=(current axis.above origin),anchor=south},
            every axis x label/.style={at=(current axis.right of origin),anchor=west}
          ]
          \addplot [dashed, textColor, smooth] plot coordinates {(-.577,0) (-.577,.385)}; %% {.451};
          \addplot [dashed, textColor, smooth] plot coordinates {(.577,-.385) (.577,0)}; %% axis{2.215};

          \addplot [very thick, penColor2, smooth] {3*x^2-1};
          \addplot [very thick, penColor, smooth] {x^3-x};

          \node at (axis cs:1.2,.3) [anchor=west] {\color{penColor}$f$};  
          \node at (axis cs:-.75,1) [anchor=west] {\color{penColor2}$f'$};

          \addplot[color=penColor2,fill=penColor2,only marks,mark=*] coordinates{(-.577,0)};  %% closed hole
          \addplot[color=penColor2,fill=penColor2,only marks,mark=*] coordinates{(.577,0)};  %% closed hole
          \addplot[color=penColor,fill=penColor,only marks,mark=*] coordinates{(-.577,.385)};  %% closed hole
          \addplot[color=penColor,fill=penColor,only marks,mark=*] coordinates{(.577,-.385)};  %% closed hole
        \end{axis}
\end{tikzpicture}
%%\caption{A plot of $f(x) = x^3-x$ and $f'(x) = 3x^2-1$.}
%%\label{figure:x^3-x}
\end{image}
\end{explanation}
\end{example}






\section{The first derivative test}

We will further explore and refine the method for deciding whether there is a
local maximum or minimum at a critical point.
 Recall that
\begin{itemize}
\item If $f'(x) >0$ on an interval, then $f$ is increasing on that interval.
\item If $f'(x) <0$ on an interval, then $f$ is decreasing on that interval.
\end{itemize}

So how exactly does the derivative tell us whether there is a maximum,
minimum, or neither at a point? Use the \textit{first derivative test}.

\begin{theorem}[First Derivative Test]\index{first derivative test}\label{T:fdt}
Suppose that $f$ is continuous on an open interval, and that $f$  has a critical point
at $x=a$, for some value $a$ in that interval.
\begin{itemize}
\item If $f'(x)>0$ to the left of $a$ and $f'(x)<0$ to the right of
  $a$, then the function $f$ has a local maximum at $a$.
\item If $f'(x)<0$ to the left of $a$ and $f'(x)>0$ to the right of
  $a$, then the function $f$ has a  local minimum at $a$.
\item If $f'(x)$ has the same sign to the left and right of $a$,
  then the function $f$ has no  local extremum at $a$.
\end{itemize}
\end{theorem}

\begin{example}\label{E:localextrema}
Consider the function 
\[
f(x) = \frac{x^4}{4}+\frac{x^3}{3}-x^2
\]
Find the intervals on which $f$ is increasing, the intervals on which $f$ is decreasing and
identify the local extrema of $f$.


\begin{explanation}
Start by computing
\[
\ddx f(x) = \answer[given]{x^3+x^2-2x}.
\]
Now we need to find on what intervals is $f'$ positive and what intervals it is
negative. To do this, solve 
\[
f'(x) = \answer[given]{x^3+x^2-2x} =0.
\]
Factor $f'(x)$
\begin{align*}
f'(x) &= \answer[given]{x^3+x^2-2x} \\
&=x(\answer[given]{x^2+x-2})\\
&=x(x+2)(\answer[given]{x-1}).
\end{align*}
So the critical points (when $f'(x)=0$) are when $x=-2$, $x=0$, and
$x=1$. Since the derivative, $f'(x)$, is a polynomial, it does not change the sign on intervals between its zeros, i.e., between the critical points. Now we can check the sign of $f'(x)$ at some points \textbf{between} the critical points to find
where $f'(x)$ is positive and where negative:
\begin{align*}
  f'(-3)&=\answer[given]{-12},\\
  f'(0.5)&=\answer[given]{-0.625},\\
  f'(-1)&=\answer[given]{2},\\
  f'(2)&=\answer[given]{8}.
\end{align*}
From this we can make a sign table:

\begin{image}
\begin{tikzpicture}
	\begin{axis}[
            trim axis left,
            scale only axis,
            domain=-3:3,
            ymax=2,
            ymin=-2,
            axis lines=none,
            height=3cm, %% Hard coded height! 
            width=\textwidth, %% width
          ]
          %\addplot [draw=none, fill=fill1, domain=(-3:-2)] {2} \closedcycle;
          %\addplot [draw=none, fill=fill2, domain=(-2:0)] {2} \closedcycle;
          %\addplot [draw=none, fill=fill1, domain=(0:1)] {2} \closedcycle;
          %\addplot [draw=none, fill=fill2, domain=(1:3)] {2} \closedcycle;
          
          \addplot [->,textColor] plot coordinates {(-3,0) (3,0)}; %% axis{0};

          \addplot [->,ultra thick,textColor,shorten <=2pt,shorten >=2pt] plot coordinates {(-3,1.5) (-2,.5)}; %% decreasing
          \addplot [->,ultra thick,textColor,shorten <=2pt,shorten >=2pt] plot coordinates {(-2,.5) (0,1.5)}; %% increasing
          \addplot [->,ultra thick,textColor,shorten <=2pt,shorten >=2pt] plot coordinates {(0,1.5) (1,.5)}; %% decreasing
          \addplot [->,ultra thick,textColor,shorten <=2pt,shorten >=2pt] plot coordinates {(1,.5) (3,1.5)}; %% increasing
          
          \addplot [dashed, textColor] plot coordinates {(-2,0) (-2,2)};
          \addplot [dashed, textColor] plot coordinates {(0,0) (0,2)};
          \addplot [dashed, textColor] plot coordinates {(1,0) (1,2)};
          
          \node at (axis cs:-2,0) [anchor=north,textColor] {\footnotesize$-2$};
          \node at (axis cs:0,0) [anchor=north,textColor] {\footnotesize$0$};
          \node at (axis cs:1,0) [anchor=north,textColor] {\footnotesize$1$};

          \node at (axis cs:-2.5,-.7) [textColor] {\footnotesize$f'(x)<0$};
          \node at (axis cs:.5,-.7) [textColor] {\footnotesize$f'(x)<0$};
          \node at (axis cs:-1,-.7) [textColor] {\footnotesize$f'(x)>0$};
          \node at (axis cs:2,-.7) [textColor] {\footnotesize$f'(x)>0$};

          %% \node at (axis cs:-2.5,-.5) [anchor=north,textColor] {\footnotesize Decreasing};
          %% \node at (axis cs:.5,-.5) [anchor=north,textColor] {\footnotesize Decreasing};
          %% \node at (axis cs:-1,-.5) [anchor=north,textColor] {\footnotesize Increasing};
          %% \node at (axis cs:2,-.5) [anchor=north,textColor] {\footnotesize Increasing};

        \end{axis}
\end{tikzpicture}
\end{image}

Hence $f$ is increasing on $(-2,0)$ and $(1,\infty)$ and $f$ is
decreasing on $(-\infty,-2)$ and $(0,1)$. Moreover, from the first
derivative test, the local maximum is at $x=0$ while the local minimums
are at $x=-2$ and $x=1$.

This can be confirmed by checking the graphs of $f(x) =x^4/4 + x^3/3
-x^2$ and $f'(x) = x^3 + x^2 -2x$.
\begin{image}
\begin{tikzpicture}
	\begin{axis}[
            domain=-4:4,
            ymax=5,
            ymin=-5,
            %samples=100,
            axis lines =middle, xlabel=$x$, ylabel=$y$,
            every axis y label/.style={at=(current axis.above origin),anchor=south},
            every axis x label/.style={at=(current axis.right of origin),anchor=west}
          ]
          \addplot [dashed, textColor, smooth] plot coordinates {(-2,0) (-2,-2.667)}; %% {.451};
          \addplot [dashed, textColor, smooth] plot coordinates {(1,0) (1,-.4167)}; %% axis{2.215};

          \addplot [very thick, penColor, smooth] {(x^4)/4 + (x^3)/3 -x^2};
          \addplot [very thick, penColor2, smooth] {x^3 + x^2 -2*x};

          \node at (axis cs:-1.3,-2) [anchor=west] {\color{penColor}$f$};  
          \node at (axis cs:-2.1,2) [anchor=west] {\color{penColor2}$f'$};

          \addplot[color=penColor2,fill=penColor2,only marks,mark=*] coordinates{(-2,0)};  %% closed hole
          \addplot[color=penColor2,fill=penColor2,only marks,mark=*] coordinates{(1,0)};  %% closed hole
          \addplot[color=penColor2,fill=penColor3,only marks,mark=*] coordinates{(0,0)};  %% closed hole
          \addplot[color=penColor,fill=penColor,only marks,mark=*] coordinates{(-2,.-2.667)};  %% closed hole
          \addplot[color=penColor,fill=penColor,only marks,mark=*] coordinates{(1,-.4167)};  %% closed hole
        \end{axis}
\end{tikzpicture}
\end{image}
\end{explanation}
\end{example}


Hence we have seen that if $f'$ is zero at a point and increasing on an interval containing that  point,
then $f$ has a local minimum at the point. If $f'$ is zero at a point and
decreasing on an interval containing that point, then $f$ has a local maximum at the
point. Thus, we see that we can gain information about $f$ by
studying how $f'$ changes. This leads us to our next section.








\end{document}