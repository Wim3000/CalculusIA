\documentclass{ximera}

%\usepackage{todonotes}

\newcommand{\todo}{}

\usepackage{esint} % for \oiint
\graphicspath{
  {./}
  {ximeraTutorial/}
  {basicPhilosophy/}
  {functionsOfSeveralVariables/}
  {normalVectors/}
  {lagrangeMultipliers/}
  {vectorFields/}
  {greensTheorem/}
  {shapeOfThingsToCome/}
}
\usepackage{comment} %% used in what is a limit
\usepackage[valunder]{signchart} %% used in graphing sign chart

\newcommand{\mooculus}{\textsf{\textbf{MOOC}\textnormal{\textsf{ULUS}}}}

\usepackage{tkz-euclide}
\tikzset{>=stealth} %% cool arrow head
\tikzset{shorten <>/.style={ shorten >=#1, shorten <=#1 } } %% allows shorter vectors


\pgfplotsset{soldot/.style={color=black,only marks,mark=*}} %% USED by piecewise functions
\pgfplotsset{holdot/.style={color=black,fill=white,only marks,mark=*}}
\usetikzlibrary{arrows.meta}


\usepackage{tkz-tab}  %% sign charts
\usepackage{polynom}

\usetikzlibrary{backgrounds} %% for boxes around graphs
\usetikzlibrary{shapes,positioning}  %% Clouds and stars
\usetikzlibrary{matrix} %% for matrix
\usepgfplotslibrary{polar} %% for polar plots
%\usetkzobj{all}
\usepackage[makeroom]{cancel} %% for strike outs
%\usepackage{mathtools} %% for pretty underbrace % Breaks Ximera
\usepackage{multicol}
\usepackage{pgffor} %% required for integral for loops


%% http://tex.stackexchange.com/questions/66490/drawing-a-tikz-arc-specifying-the-center
%% Draws beach ball
\tikzset{pics/carc/.style args={#1:#2:#3}{code={\draw[pic actions] (#1:#3) arc(#1:#2:#3);}}}



\usepackage{array}
\setlength{\extrarowheight}{+.1cm}   
\newdimen\digitwidth
\settowidth\digitwidth{9}
\def\divrule#1#2{
\noalign{\moveright#1\digitwidth
\vbox{\hrule width#2\digitwidth}}}




\newcommand{\RR}{\mathbb R}
\newcommand{\R}{\mathbb R}
\newcommand{\N}{\mathbb N}
\newcommand{\Z}{\mathbb Z}

\newcommand{\sagemath}{\textsf{SageMath}}


%\renewcommand{\d}{\,d\!}
\renewcommand{\d}{\mathop{}\!d}
\newcommand{\dd}[2][]{\frac{\d #1}{\d #2}}
\newcommand{\pp}[2][]{\frac{\partial #1}{\partial #2}}
\renewcommand{\l}{\ell}
\newcommand{\ddx}{\frac{d}{\d x}}
\newcommand{\ddt}{\frac{d}{\d t}}


\newcommand{\zeroOverZero}{\ensuremath{\boldsymbol{\tfrac{0}{0}}}}
\newcommand{\inftyOverInfty}{\ensuremath{\boldsymbol{\tfrac{\infty}{\infty}}}}
\newcommand{\zeroOverInfty}{\ensuremath{\boldsymbol{\tfrac{0}{\infty}}}}
\newcommand{\zeroTimesInfty}{\ensuremath{\small\boldsymbol{0\cdot \infty}}}
\newcommand{\inftyMinusInfty}{\ensuremath{\small\boldsymbol{\infty - \infty}}}
\newcommand{\oneToInfty}{\ensuremath{\boldsymbol{1^\infty}}}
\newcommand{\zeroToZero}{\ensuremath{\boldsymbol{0^0}}}
\newcommand{\inftyToZero}{\ensuremath{\boldsymbol{\infty^0}}}



\newcommand{\numOverZero}{\ensuremath{\boldsymbol{\tfrac{\#}{0}}}}
\newcommand{\dfn}{\textbf}
%\newcommand{\unit}{\,\mathrm}
\newcommand{\unit}{\mathop{}\!\mathrm}
\newcommand{\eval}[1]{\bigg[ #1 \bigg]}
\newcommand{\seq}[1]{\left( #1 \right)}
\renewcommand{\epsilon}{\varepsilon}
\renewcommand{\phi}{\varphi}


\renewcommand{\iff}{\Leftrightarrow}

\DeclareMathOperator{\arccot}{arccot}
\DeclareMathOperator{\arcsec}{arcsec}
\DeclareMathOperator{\arccsc}{arccsc}
\DeclareMathOperator{\si}{Si}
\DeclareMathOperator{\proj}{\vec{proj}}
\DeclareMathOperator{\scal}{scal}
\DeclareMathOperator{\sign}{sign}


%% \newcommand{\tightoverset}[2]{% for arrow vec
%%   \mathop{#2}\limits^{\vbox to -.5ex{\kern-0.75ex\hbox{$#1$}\vss}}}
\newcommand{\arrowvec}{\overrightarrow}
%\renewcommand{\vec}[1]{\arrowvec{\mathbf{#1}}}
\renewcommand{\vec}{\mathbf}
\newcommand{\veci}{{\boldsymbol{\hat{\imath}}}}
\newcommand{\vecj}{{\boldsymbol{\hat{\jmath}}}}
\newcommand{\veck}{{\boldsymbol{\hat{k}}}}
\newcommand{\vecl}{\boldsymbol{\l}}
\newcommand{\uvec}[1]{\mathbf{\hat{#1}}}
\newcommand{\utan}{\mathbf{\hat{t}}}
\newcommand{\unormal}{\mathbf{\hat{n}}}
\newcommand{\ubinormal}{\mathbf{\hat{b}}}

\newcommand{\dotp}{\bullet}
\newcommand{\cross}{\boldsymbol\times}
\newcommand{\grad}{\boldsymbol\nabla}
\newcommand{\divergence}{\grad\dotp}
\newcommand{\curl}{\grad\cross}
%\DeclareMathOperator{\divergence}{divergence}
%\DeclareMathOperator{\curl}[1]{\grad\cross #1}
\newcommand{\lto}{\mathop{\longrightarrow\,}\limits}

\renewcommand{\bar}{\overline}

\colorlet{textColor}{black} 
\colorlet{background}{white}
\colorlet{penColor}{blue!50!black} % Color of a curve in a plot
\colorlet{penColor2}{red!50!black}% Color of a curve in a plot
\colorlet{penColor3}{red!50!blue} % Color of a curve in a plot
\colorlet{penColor4}{green!50!black} % Color of a curve in a plot
\colorlet{penColor5}{orange!80!black} % Color of a curve in a plot
\colorlet{penColor6}{yellow!70!black} % Color of a curve in a plot
\colorlet{fill1}{penColor!20} % Color of fill in a plot
\colorlet{fill2}{penColor2!20} % Color of fill in a plot
\colorlet{fillp}{fill1} % Color of positive area
\colorlet{filln}{penColor2!20} % Color of negative area
\colorlet{fill3}{penColor3!20} % Fill
\colorlet{fill4}{penColor4!20} % Fill
\colorlet{fill5}{penColor5!20} % Fill
\colorlet{gridColor}{gray!50} % Color of grid in a plot

\newcommand{\surfaceColor}{violet}
\newcommand{\surfaceColorTwo}{redyellow}
\newcommand{\sliceColor}{greenyellow}




\pgfmathdeclarefunction{gauss}{2}{% gives gaussian
  \pgfmathparse{1/(#2*sqrt(2*pi))*exp(-((x-#1)^2)/(2*#2^2))}%
}


%%%%%%%%%%%%%
%% Vectors
%%%%%%%%%%%%%

%% Simple horiz vectors
\renewcommand{\vector}[1]{\left\langle #1\right\rangle}


%% %% Complex Horiz Vectors with angle brackets
%% \makeatletter
%% \renewcommand{\vector}[2][ , ]{\left\langle%
%%   \def\nextitem{\def\nextitem{#1}}%
%%   \@for \el:=#2\do{\nextitem\el}\right\rangle%
%% }
%% \makeatother

%% %% Vertical Vectors
%% \def\vector#1{\begin{bmatrix}\vecListA#1,,\end{bmatrix}}
%% \def\vecListA#1,{\if,#1,\else #1\cr \expandafter \vecListA \fi}

%%%%%%%%%%%%%
%% End of vectors
%%%%%%%%%%%%%

%\newcommand{\fullwidth}{}
%\newcommand{\normalwidth}{}



%% makes a snazzy t-chart for evaluating functions
%\newenvironment{tchart}{\rowcolors{2}{}{background!90!textColor}\array}{\endarray}

%%This is to help with formatting on future title pages.
\newenvironment{sectionOutcomes}{}{} 



%% Flowchart stuff
%\tikzstyle{startstop} = [rectangle, rounded corners, minimum width=3cm, minimum height=1cm,text centered, draw=black]
%\tikzstyle{question} = [rectangle, minimum width=3cm, minimum height=1cm, text centered, draw=black]
%\tikzstyle{decision} = [trapezium, trapezium left angle=70, trapezium right angle=110, minimum width=3cm, minimum height=1cm, text centered, draw=black]
%\tikzstyle{question} = [rectangle, rounded corners, minimum width=3cm, minimum height=1cm,text centered, draw=black]
%\tikzstyle{process} = [rectangle, minimum width=3cm, minimum height=1cm, text centered, draw=black]
%\tikzstyle{decision} = [trapezium, trapezium left angle=70, trapezium right angle=110, minimum width=3cm, minimum height=1cm, text centered, draw=black]


\title{Composition of Functions}
\begin{document}

\begin{abstract} \end{abstract}
\maketitle

At the end of this section, you should be able to:
\begin{itemize}
    \item Interpret a composition of functions when functions have meaning.
    \item Perform basic operations and compositions on functions.
      \item Evaluate a composition of functions.
      \item Identify the ``inner" and ``outer" function for composition.
    \item Given a composition of functions, decompose them.
    
  
\end{itemize}

The notation that you should be aware of is:
\begin{itemize}
    \item A composition of functions can be denoted as $$(f\circ g)(x)=f(g(x)).$$
\end{itemize}
Let's start with an example in a
``real context.''

\begin{example}
  Let
  \[
  g(m) = \text{the amount of gas one can buy with $m$ dollars,}
  \]
  and let
  \[
  f(g) = \text{how far one can drive with $g$ gallons of gas.}
  \]
  What does $f(g(m))$ represent in this setting?
  \begin{explanation}
    With $f(g(m))$ we first relate how far one can drive with
    $\answer[given]{g}$ gallons of gas, and this in turn is determined
    by how much money $\answer[given]{m}$ one has. Hence $f(g(m))$ represents how far
    one can drive with $\answer[given]{m}$ dollars.
  \end{explanation}
\end{example}

Composition of functions can be thought of as putting one function
inside another.  We use the notation
\[
(f\circ g)(x) = f(g(x)).
\]
Here we can see that our input for our function $f(g(x))$ would $g(x)$ which has itself an input of $x$. So, the output of $g(x)$ is the input for $(f\circ g)(x)$.


\begin{example}
 Suppose we have
\begin{align*}
  f(x)&={{x}^{2}}+5x+4 &&\text{for $-\infty< x< \infty$,}\\
  g(x)&= x+7 &&\text{for $-\infty< x< \infty$.}\\
\end{align*}
\begin{enumerate}
    \item[(a)] Find $f(g(-5))$ 
    \item[(b)] Find $f(g(x))$ and state its domain.
\end{enumerate}

\begin{explanation}

\begin{enumerate}
    
\item[(a)] To find $f(g(-5))$, we must start by finding that $g(-5)=\answer[given]{-5}+7=\answer[given]{2}$. Now, we take the output for $g(-5)$ and use it as the input for $f(x)$. Thus $f(g(-5))=f(\answer[given]{2})=(\answer[given]{2})^2+5(\answer[given]{2})+4=\answer[given]{18}$. 
  \item[(b)] The range of $g$ is $-\infty< x< \infty$, which is equal to the
  domain of $f$. This means the domain of $f\circ g$ is $-\infty< x<
  \infty$. Next, we substitute $x+7$ for each instance of $\answer[given]{x}$ found
  in
  \[
  f(x)={{x}^{2}}+5x+4
  \]
  and so
  \begin{align*}
  f(g(x)) &=f(x+7)\\
  &=\answer[given]{{{(x+7)}^{2}}+5(x+7)+4}.
  \end{align*}
  \end{enumerate}
\end{explanation}
\end{example}
\begin{example}
    Let
    \begin{align*}
  f(x)&=x^2 &&\text{for $-\infty< x< \infty$,}\\
  g(x)&= \sqrt{x} &&\text{for $0\le x< \infty$.}\\
\end{align*}
Determine $f(g(-1))$.
\begin{explanation}
    Notice our first step would be to evaluate $g(-1)$. We can see that $g(-1)=\sqrt{-1}$ which is not a real number and thus $g(-1)$ is not defined. So that means that $-1$ would be outside the domain of $f(g(x))$. 
\end{explanation}
\end{example}
\begin{warning}
  The composition $f\circ g$ only makes sense if
  \[
  \{\text{the range of $g$}\}
  \text{ is contained in or equal to }
  \{\text{the domain of $f$}\}
  \]
\end{warning}
Now let's try an example with a more restricted domain.

\begin{example}
 Suppose we have:
\begin{align*}
  f(x)&=x^2 &&\text{for $-\infty< x< \infty$,}\\
  g(x)&= \sqrt{x} &&\text{for $0\le x< \infty$.}\\
\end{align*}
Find $f(g(x))$ and state its domain.
\begin{explanation}
  The domain of $g$ is $0\le x< \infty$. From this we can see that the
  range of $g$ is $\answer[given]{0}\le x< \infty$. This is contained
  in the domain of $f$.

  This means that the domain of $f\circ g$ is $0\le x< \infty$.  Next,
  we substitute $\answer[given]{\sqrt{x}}$ for each instance of $x$
  found in
  \[
  f(x)={{x}^{2}}
  \]
  and so
  \begin{align*}
  f(g(x))&=f(\sqrt{x})\\
  &=\left(\sqrt{x}\right)^2.
  \end{align*}
  We can summarize our results as a piecewise function, which
  looks somewhat interesting:
  \[
  (f\circ g)(x) = 
  \begin{cases}
    x & \text{if $0\le x < \infty$}\\
   \text{undefined} &\text{otherwise}. 
  \end{cases}
  \]
\end{explanation}
\end{example}


\begin{example}
 What if we instead compose the other way? So again let:
\begin{align*}
  f(x)&=x^2 &&\text{for $-\infty< x< \infty$,}\\
  g(x)&= \sqrt{x} &&\text{for $0\le x< \infty$.}
\end{align*}
Find $g(f(x))$ and state its domain.
\begin{explanation}
  While the domain of $f$ is $-\infty< x< \infty$, its range is only
  $0 \le x<\infty$ since the output of $x^2$ can never be negative. This is exactly the domain of $g$. This means that
  the domain of $g\circ f$ is $-\infty< x< \infty$ because no matter what $x$-value we input for $f(x)$, we will always end up with an $f(x)$ value that is in the domain of $g(x)$.
  Now we may substitute $\answer[given]{x^2}$ for each instance of
  $\answer[given]{x}$ found in
  \[
  g(x)=\sqrt{x}
  \]
  and so
  \begin{align*}
  g(f(x))&=g(x^2)\\
  &=\sqrt{x^2},\\
  &=|x|.
  \end{align*}
\end{explanation}
\end{example}

Compare and contrast the previous two examples.  We used the same
functions for each example, but composed them in different ways.  The resulting
compositions are not only different, they have different domains!



It is not only important to be able to compose two functions, it is also important to be able to \textbf{decompose} a composition of them! By decompose, we mean that we want to determine the two functions, $f(x)$ and $g(x)$, that create the composition of two functions $h(x)=f(g(x))$. Now, it is not unusual to have multiple possible answers, but we do want to avoid the trivial answers, when either $f(x)=x$ or $g(x)=x$.

\begin{example}
   Let $h(x)=f(g(x))$. Decompose the following compositions of functions.
    \begin{enumerate}
        \item[(a)] $h(x)=\sqrt[3]{2x-3}$ 
        \item[(b)] $h(x)=\frac{2}{(x+2)^2}$
    \end{enumerate}
    \begin{explanation}
    We know that $h(x)=f(g(x))$. We will often think about $g(x)$ as the ``inner" function as it is the function that is plugged into $f(x)$ and thus $f(x)$ is the ``outer" function. We can keep this in mind as we proceed to answering these questions.
        \begin{enumerate}
            \item[(a)] We start by identifying the ``inner" function, our $g(x)$. We can let. $g(x)=2x-3$. If that is the case, then $f(x)$, our ``outer" function, is $\answer[given]{\sqrt[3]{x}}$. We can then check that this forms our composition by plugging in $g(x)=\answer[given]{2x-3}$ for the $x$ in $f(x)=\answer[given]{\sqrt[3]{x}}$. Doing that, we confirm that $h(x)=f(g(x))=\sqrt[3]{2x-3}$. Note that we could have looked at another composition of functions where $g(x)=2x$ and $f(x)=\sqrt[3]{x-3}$. This is a valid decomposition, but for our future purposes (how mysterious!) not the preferred decomposition.
            
            \item[(b)] We can follow a similar path for this problem. There are two common decompositions. First, we will let our ``outer" function be $f(x)=\frac{2}{x}$.  In this case, we can see that our ``inner" function will be, $g(x)=\answer[given]{(x+2)^2}$. Composing these two functions and get our desired result $h(x)=f(g(x))=\answer[given]{\frac{2}{(x+2)^2}}$.
            
            For our second option, let's identify an ``inner" function, $g(x)=\answer[given]{x+2}$. Then our ``outer" function, $f(x)=\answer[given]{\frac{2}{x^2}}$. We can compose these two functions and get the result $h(x)=f(g(x))=\answer[given]{\frac{2}{(x+2)^2}}$.
           
        \end{enumerate}
    \end{explanation}
\end{example}

\end{document}
