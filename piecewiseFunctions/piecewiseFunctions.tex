\documentclass{ximera}


\usepackage{pgfplots}
\usepackage{tikz}
\pgfplotsset{soldot/.style={color=black,only marks,mark=*}}
\pgfplotsset{holdot/.style={color=black,fill=white,only marks,mark=*}}
\usetikzlibrary{arrows.meta}

%\usepackage{todonotes}

\newcommand{\todo}{}

\usepackage{esint} % for \oiint
\graphicspath{
  {./}
  {ximeraTutorial/}
  {basicPhilosophy/}
  {functionsOfSeveralVariables/}
  {normalVectors/}
  {lagrangeMultipliers/}
  {vectorFields/}
  {greensTheorem/}
  {shapeOfThingsToCome/}
}
\usepackage{comment} %% used in what is a limit
\usepackage[valunder]{signchart} %% used in graphing sign chart

\newcommand{\mooculus}{\textsf{\textbf{MOOC}\textnormal{\textsf{ULUS}}}}

\usepackage{tkz-euclide}
\tikzset{>=stealth} %% cool arrow head
\tikzset{shorten <>/.style={ shorten >=#1, shorten <=#1 } } %% allows shorter vectors


\pgfplotsset{soldot/.style={color=black,only marks,mark=*}} %% USED by piecewise functions
\pgfplotsset{holdot/.style={color=black,fill=white,only marks,mark=*}}
\usetikzlibrary{arrows.meta}


\usepackage{tkz-tab}  %% sign charts
\usepackage{polynom}

\usetikzlibrary{backgrounds} %% for boxes around graphs
\usetikzlibrary{shapes,positioning}  %% Clouds and stars
\usetikzlibrary{matrix} %% for matrix
\usepgfplotslibrary{polar} %% for polar plots
%\usetkzobj{all}
\usepackage[makeroom]{cancel} %% for strike outs
%\usepackage{mathtools} %% for pretty underbrace % Breaks Ximera
\usepackage{multicol}
\usepackage{pgffor} %% required for integral for loops


%% http://tex.stackexchange.com/questions/66490/drawing-a-tikz-arc-specifying-the-center
%% Draws beach ball
\tikzset{pics/carc/.style args={#1:#2:#3}{code={\draw[pic actions] (#1:#3) arc(#1:#2:#3);}}}



\usepackage{array}
\setlength{\extrarowheight}{+.1cm}   
\newdimen\digitwidth
\settowidth\digitwidth{9}
\def\divrule#1#2{
\noalign{\moveright#1\digitwidth
\vbox{\hrule width#2\digitwidth}}}




\newcommand{\RR}{\mathbb R}
\newcommand{\R}{\mathbb R}
\newcommand{\N}{\mathbb N}
\newcommand{\Z}{\mathbb Z}

\newcommand{\sagemath}{\textsf{SageMath}}


%\renewcommand{\d}{\,d\!}
\renewcommand{\d}{\mathop{}\!d}
\newcommand{\dd}[2][]{\frac{\d #1}{\d #2}}
\newcommand{\pp}[2][]{\frac{\partial #1}{\partial #2}}
\renewcommand{\l}{\ell}
\newcommand{\ddx}{\frac{d}{\d x}}
\newcommand{\ddt}{\frac{d}{\d t}}


\newcommand{\zeroOverZero}{\ensuremath{\boldsymbol{\tfrac{0}{0}}}}
\newcommand{\inftyOverInfty}{\ensuremath{\boldsymbol{\tfrac{\infty}{\infty}}}}
\newcommand{\zeroOverInfty}{\ensuremath{\boldsymbol{\tfrac{0}{\infty}}}}
\newcommand{\zeroTimesInfty}{\ensuremath{\small\boldsymbol{0\cdot \infty}}}
\newcommand{\inftyMinusInfty}{\ensuremath{\small\boldsymbol{\infty - \infty}}}
\newcommand{\oneToInfty}{\ensuremath{\boldsymbol{1^\infty}}}
\newcommand{\zeroToZero}{\ensuremath{\boldsymbol{0^0}}}
\newcommand{\inftyToZero}{\ensuremath{\boldsymbol{\infty^0}}}



\newcommand{\numOverZero}{\ensuremath{\boldsymbol{\tfrac{\#}{0}}}}
\newcommand{\dfn}{\textbf}
%\newcommand{\unit}{\,\mathrm}
\newcommand{\unit}{\mathop{}\!\mathrm}
\newcommand{\eval}[1]{\bigg[ #1 \bigg]}
\newcommand{\seq}[1]{\left( #1 \right)}
\renewcommand{\epsilon}{\varepsilon}
\renewcommand{\phi}{\varphi}


\renewcommand{\iff}{\Leftrightarrow}

\DeclareMathOperator{\arccot}{arccot}
\DeclareMathOperator{\arcsec}{arcsec}
\DeclareMathOperator{\arccsc}{arccsc}
\DeclareMathOperator{\si}{Si}
\DeclareMathOperator{\proj}{\vec{proj}}
\DeclareMathOperator{\scal}{scal}
\DeclareMathOperator{\sign}{sign}


%% \newcommand{\tightoverset}[2]{% for arrow vec
%%   \mathop{#2}\limits^{\vbox to -.5ex{\kern-0.75ex\hbox{$#1$}\vss}}}
\newcommand{\arrowvec}{\overrightarrow}
%\renewcommand{\vec}[1]{\arrowvec{\mathbf{#1}}}
\renewcommand{\vec}{\mathbf}
\newcommand{\veci}{{\boldsymbol{\hat{\imath}}}}
\newcommand{\vecj}{{\boldsymbol{\hat{\jmath}}}}
\newcommand{\veck}{{\boldsymbol{\hat{k}}}}
\newcommand{\vecl}{\boldsymbol{\l}}
\newcommand{\uvec}[1]{\mathbf{\hat{#1}}}
\newcommand{\utan}{\mathbf{\hat{t}}}
\newcommand{\unormal}{\mathbf{\hat{n}}}
\newcommand{\ubinormal}{\mathbf{\hat{b}}}

\newcommand{\dotp}{\bullet}
\newcommand{\cross}{\boldsymbol\times}
\newcommand{\grad}{\boldsymbol\nabla}
\newcommand{\divergence}{\grad\dotp}
\newcommand{\curl}{\grad\cross}
%\DeclareMathOperator{\divergence}{divergence}
%\DeclareMathOperator{\curl}[1]{\grad\cross #1}
\newcommand{\lto}{\mathop{\longrightarrow\,}\limits}

\renewcommand{\bar}{\overline}

\colorlet{textColor}{black} 
\colorlet{background}{white}
\colorlet{penColor}{blue!50!black} % Color of a curve in a plot
\colorlet{penColor2}{red!50!black}% Color of a curve in a plot
\colorlet{penColor3}{red!50!blue} % Color of a curve in a plot
\colorlet{penColor4}{green!50!black} % Color of a curve in a plot
\colorlet{penColor5}{orange!80!black} % Color of a curve in a plot
\colorlet{penColor6}{yellow!70!black} % Color of a curve in a plot
\colorlet{fill1}{penColor!20} % Color of fill in a plot
\colorlet{fill2}{penColor2!20} % Color of fill in a plot
\colorlet{fillp}{fill1} % Color of positive area
\colorlet{filln}{penColor2!20} % Color of negative area
\colorlet{fill3}{penColor3!20} % Fill
\colorlet{fill4}{penColor4!20} % Fill
\colorlet{fill5}{penColor5!20} % Fill
\colorlet{gridColor}{gray!50} % Color of grid in a plot

\newcommand{\surfaceColor}{violet}
\newcommand{\surfaceColorTwo}{redyellow}
\newcommand{\sliceColor}{greenyellow}




\pgfmathdeclarefunction{gauss}{2}{% gives gaussian
  \pgfmathparse{1/(#2*sqrt(2*pi))*exp(-((x-#1)^2)/(2*#2^2))}%
}


%%%%%%%%%%%%%
%% Vectors
%%%%%%%%%%%%%

%% Simple horiz vectors
\renewcommand{\vector}[1]{\left\langle #1\right\rangle}


%% %% Complex Horiz Vectors with angle brackets
%% \makeatletter
%% \renewcommand{\vector}[2][ , ]{\left\langle%
%%   \def\nextitem{\def\nextitem{#1}}%
%%   \@for \el:=#2\do{\nextitem\el}\right\rangle%
%% }
%% \makeatother

%% %% Vertical Vectors
%% \def\vector#1{\begin{bmatrix}\vecListA#1,,\end{bmatrix}}
%% \def\vecListA#1,{\if,#1,\else #1\cr \expandafter \vecListA \fi}

%%%%%%%%%%%%%
%% End of vectors
%%%%%%%%%%%%%

%\newcommand{\fullwidth}{}
%\newcommand{\normalwidth}{}



%% makes a snazzy t-chart for evaluating functions
%\newenvironment{tchart}{\rowcolors{2}{}{background!90!textColor}\array}{\endarray}

%%This is to help with formatting on future title pages.
\newenvironment{sectionOutcomes}{}{} 



%% Flowchart stuff
%\tikzstyle{startstop} = [rectangle, rounded corners, minimum width=3cm, minimum height=1cm,text centered, draw=black]
%\tikzstyle{question} = [rectangle, minimum width=3cm, minimum height=1cm, text centered, draw=black]
%\tikzstyle{decision} = [trapezium, trapezium left angle=70, trapezium right angle=110, minimum width=3cm, minimum height=1cm, text centered, draw=black]
%\tikzstyle{question} = [rectangle, rounded corners, minimum width=3cm, minimum height=1cm,text centered, draw=black]
%\tikzstyle{process} = [rectangle, minimum width=3cm, minimum height=1cm, text centered, draw=black]
%\tikzstyle{decision} = [trapezium, trapezium left angle=70, trapezium right angle=110, minimum width=3cm, minimum height=1cm, text centered, draw=black]

%preamble previously had ../ before it, deleted to compile??

%\outcome{Know the graphs and properties of ``famous'' functions.}

\title{1.2 - Piecewise Functions}


\begin{document}
\begin{abstract}
  Some functions use more than one formula.
\end{abstract}
\maketitle

\subsection*{Introduction}
Sometimes a function uses different formulas depending on what number is being plugged into it. Functions that use different formulas for different parts of their domain are called \textbf{piecewise functions}, or sometimes \textbf{piecewise-defined functions}.

\subsection*{Notation}
When defining a piecewise function, we need to show the various formulas being used as well as which parts of the domain each applies to. We typically write the formulas in a vertical list, grouped by a brace symbol, with their associated domain intervals on the right.

\begin{example}
$f(x)=\begin{cases} x+10 &  \mbox{ \emph{if} } x<0\\x^2+1 &  \mbox{ \emph{if} } x\geq 0\end{cases}$

This notation defines a single function $f(x)$ that uses the formula $y=x+10$ when plugging in $x$ values less than $0$, and the formula $y=x^2+1$ when plugging in $x$ values $0$ or higher.
\end{example}

\begin{example}
$g(x)=\begin{cases} 2x+5 & \mbox{ \emph{if} }x\leq 2\\ -(x-2)(x-10) & \mbox{ \emph{if} } 2<x<10\\ x-10 & \mbox{ \emph{if} } x\geq 10\end{cases}$

This notation defines a single function $g(x)$ that uses the formula $y=2x+5$ when plugging in $x$ values $2$ or less, and the formula $y=-(x-2)(x-10)$ when plugging in $x$ values strictly between $2$ and $10$, and the formula $y=x-10$ when plugging in $x$ values $10$ or higher.
\end{example}

\subsection*{Plugging in}
When we want to plug a number into a piecewise function to compute its corresponding output, we start by deciding which of the intervals the number falls into. These intervals will not overlap, so for each $x$ there will always be only one interval that it belongs to. We plug into the formula next to the chosen interval; the other formulas will not be used for that input.

\begin{example}
$f(x)=\begin{cases} x+10 & \mbox{ \emph{if} }x<0\\x^2+1 & \mbox{ \emph{if} } x\geq 0\end{cases}$

Suppose we want to compute $f(-3)$, which is the output corresponding to $x=-3$. We start by determining which interval $-3$ is in: $x<0$ or $x\geq 0$. \\Since $-3<0$, we choose the first interval $x<0$, which means we should plug into the first formula $y=x+10$. Thus $f(-3)=-3+10=7$.
\\
\\Suppose we want to compute $f(4)$. Since 
\begin{multipleChoice}
    \choice{$4< 0$}
    \choice[correct]{$4\geq 0$}
  \end{multipleChoice}
, we choose the 
\begin{multipleChoice}
    \choice{first formula $y=x+10$}
    \choice[correct]{second formula $y=x^2+1$}
  \end{multipleChoice}. Thus $f(4)=4^2+1=$
  \begin{multipleChoice}
    \choice[correct]{$17$}
    \choice{$14$}
    \choice{$4$}
    \choice{$14$ and $17$}
  \end{multipleChoice}
\end{example}

\begin{example}
$g(x)=\begin{cases} 2x+5 & \mbox{ \emph{if} }x\leq 2\\ -(x-2)(x-10) & \mbox{ \emph{if} } 2<x<10\\ x-10 & \mbox{ \emph{if} } x\geq 10\end{cases}$

To compute $g(6)$, we start by determining which of the three intervals contains $x=6$. Since 
\begin{multipleChoice}
    \choice{$6\leq 2$}
    \choice[correct]{$2<6<10$}
    \choice{$6\geq 10$}
  \end{multipleChoice}
  , we plug into the 
  \begin{multipleChoice}
    \choice{frst}
    \choice[correct]{second}
    \choice{third}
  \end{multipleChoice}
formula. Thus $g(6)=$
\begin{multipleChoice}
    \choice{$17$}
    \choice[correct]{$16$}
    \choice{$-4$}
  \end{multipleChoice}.\\
\\Similarly, we can compute $g(0)=$
\begin{multipleChoice}
    \choice[correct]{$5$}
    \choice{$-20$}
    \choice{$-10$}
  \end{multipleChoice} using the first formula and $g(12)=$ 
  \begin{multipleChoice}
    \choice{$29$}
    \choice{$-20$}
    \choice[correct]{$2$}
  \end{multipleChoice}using the third formula.
\end{example}



\subsection*{Graphing}
The graphs of piecewise functions often have a corner or a jump where the formula changes. 

\begin{example}
$f(x)=\begin{cases} x+10 & \mbox{ \emph{if} }x<0\\x^2+1 & \mbox{ \emph{if} } x\geq 0\end{cases}$

This graph will look like the graph of $y=x+10$ when $x<0$, which is on the left side of the $y$-axis. Then when $x=0$, the graph jumps and starts showing the curve $y=x^2+1$ on the right side of the $y$-axis.

\begin{tikzpicture} [scale=.8]
    \begin{axis}[samples=50,xmin=-6,xmax=6, ymin=-2, ymax=20, axis x line= middle, axis y line = middle,xtick={},ytick={}]
	\addplot[black,domain=-5:0,<-] {x+10};
	\addplot[black,domain=0:4,->] {x^2+1};
	\addplot[holdot] coordinates{(0,10)};
	\addplot[soldot] coordinates{(0,1)};
	\end{axis}
\end{tikzpicture}

\textbf{What's up with the open and closed points?} The open dot at $(0,10)$ shows that there would have been a point there but it is missing. When $x$ reaches $0$, the curve jumps down and plots the point $(0,1)$ (the solid dot, meaning it really is there) using the second formula.

In general, showing open and closed points at the transition clarifies what's happening there. Without this extra detail, it would not be clear to our eyes precisely where or how the jump happened. The solid dot will only be graphed on the function that truly uses the $x$ value.
\end{example}

\begin{example}
$g(x)=\begin{cases} 2x+5 & \mbox{ \emph{if} }x\leq 2\\ 9(x-2)(x-10) & \mbox{ \emph{if} } 2<x<10\\ x-10 & \mbox{ \emph{if} } x\geq 10\end{cases}$

This graph will have transitions at $x=2$ and $x=10$. The left-most part will look like $y=2x+5$, the middle part will look like $y=-(x-2)(x-10)$, and the right-most part will look like $y=x-20$. If you're not sure what these graphs look like, plug in some $x$ values and plot points to connect!

\begin{tikzpicture} [scale=.8]
    \begin{axis}[samples=50,xmin=-6,xmax=15, ymin=-5, ymax=20, axis x line= middle, axis y line = middle,xtick={},ytick={}]
	\addplot[black,domain=-5:2,<-] {2*x+5};
	\addplot[black,domain=2:10,-] {-1*(x-2)*(x-10)};
	\addplot[black,domain=10:14,->] {x-10};
	\addplot[soldot] coordinates{(2,9)};
	\addplot[holdot] coordinates{(2,0)};
%	\addplot[holdot] coordinates{(10,2.8)};
	\addplot[soldot] coordinates{(10,0)};
	\end{axis}
\end{tikzpicture}

Notice that at $x=10$, $y=-(x-2)(x-10)$ and $y=x-10$ both output $y=0$, so instead of a jump this graph has a corner at the transition.
\end{example}

\subsection*{Learning Objectives}
After completing this section, students should be able to:
\vspace{.05in}

\noindent$\bullet$ understand the notation of a piecewise function
\\$\bullet$ compute and plot points using a given piecewise function
\\$\bullet$ sketch the graph of a given piecewise function



\end{document}