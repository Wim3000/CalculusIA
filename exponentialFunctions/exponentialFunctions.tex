\documentclass{ximera}

%\usepackage{todonotes}

\newcommand{\todo}{}

\usepackage{esint} % for \oiint
\graphicspath{
  {./}
  {ximeraTutorial/}
  {basicPhilosophy/}
  {functionsOfSeveralVariables/}
  {normalVectors/}
  {lagrangeMultipliers/}
  {vectorFields/}
  {greensTheorem/}
  {shapeOfThingsToCome/}
}
\usepackage{comment} %% used in what is a limit
\usepackage[valunder]{signchart} %% used in graphing sign chart

\newcommand{\mooculus}{\textsf{\textbf{MOOC}\textnormal{\textsf{ULUS}}}}

\usepackage{tkz-euclide}
\tikzset{>=stealth} %% cool arrow head
\tikzset{shorten <>/.style={ shorten >=#1, shorten <=#1 } } %% allows shorter vectors


\pgfplotsset{soldot/.style={color=black,only marks,mark=*}} %% USED by piecewise functions
\pgfplotsset{holdot/.style={color=black,fill=white,only marks,mark=*}}
\usetikzlibrary{arrows.meta}


\usepackage{tkz-tab}  %% sign charts
\usepackage{polynom}

\usetikzlibrary{backgrounds} %% for boxes around graphs
\usetikzlibrary{shapes,positioning}  %% Clouds and stars
\usetikzlibrary{matrix} %% for matrix
\usepgfplotslibrary{polar} %% for polar plots
%\usetkzobj{all}
\usepackage[makeroom]{cancel} %% for strike outs
%\usepackage{mathtools} %% for pretty underbrace % Breaks Ximera
\usepackage{multicol}
\usepackage{pgffor} %% required for integral for loops


%% http://tex.stackexchange.com/questions/66490/drawing-a-tikz-arc-specifying-the-center
%% Draws beach ball
\tikzset{pics/carc/.style args={#1:#2:#3}{code={\draw[pic actions] (#1:#3) arc(#1:#2:#3);}}}



\usepackage{array}
\setlength{\extrarowheight}{+.1cm}   
\newdimen\digitwidth
\settowidth\digitwidth{9}
\def\divrule#1#2{
\noalign{\moveright#1\digitwidth
\vbox{\hrule width#2\digitwidth}}}




\newcommand{\RR}{\mathbb R}
\newcommand{\R}{\mathbb R}
\newcommand{\N}{\mathbb N}
\newcommand{\Z}{\mathbb Z}

\newcommand{\sagemath}{\textsf{SageMath}}


%\renewcommand{\d}{\,d\!}
\renewcommand{\d}{\mathop{}\!d}
\newcommand{\dd}[2][]{\frac{\d #1}{\d #2}}
\newcommand{\pp}[2][]{\frac{\partial #1}{\partial #2}}
\renewcommand{\l}{\ell}
\newcommand{\ddx}{\frac{d}{\d x}}
\newcommand{\ddt}{\frac{d}{\d t}}


\newcommand{\zeroOverZero}{\ensuremath{\boldsymbol{\tfrac{0}{0}}}}
\newcommand{\inftyOverInfty}{\ensuremath{\boldsymbol{\tfrac{\infty}{\infty}}}}
\newcommand{\zeroOverInfty}{\ensuremath{\boldsymbol{\tfrac{0}{\infty}}}}
\newcommand{\zeroTimesInfty}{\ensuremath{\small\boldsymbol{0\cdot \infty}}}
\newcommand{\inftyMinusInfty}{\ensuremath{\small\boldsymbol{\infty - \infty}}}
\newcommand{\oneToInfty}{\ensuremath{\boldsymbol{1^\infty}}}
\newcommand{\zeroToZero}{\ensuremath{\boldsymbol{0^0}}}
\newcommand{\inftyToZero}{\ensuremath{\boldsymbol{\infty^0}}}



\newcommand{\numOverZero}{\ensuremath{\boldsymbol{\tfrac{\#}{0}}}}
\newcommand{\dfn}{\textbf}
%\newcommand{\unit}{\,\mathrm}
\newcommand{\unit}{\mathop{}\!\mathrm}
\newcommand{\eval}[1]{\bigg[ #1 \bigg]}
\newcommand{\seq}[1]{\left( #1 \right)}
\renewcommand{\epsilon}{\varepsilon}
\renewcommand{\phi}{\varphi}


\renewcommand{\iff}{\Leftrightarrow}

\DeclareMathOperator{\arccot}{arccot}
\DeclareMathOperator{\arcsec}{arcsec}
\DeclareMathOperator{\arccsc}{arccsc}
\DeclareMathOperator{\si}{Si}
\DeclareMathOperator{\proj}{\vec{proj}}
\DeclareMathOperator{\scal}{scal}
\DeclareMathOperator{\sign}{sign}


%% \newcommand{\tightoverset}[2]{% for arrow vec
%%   \mathop{#2}\limits^{\vbox to -.5ex{\kern-0.75ex\hbox{$#1$}\vss}}}
\newcommand{\arrowvec}{\overrightarrow}
%\renewcommand{\vec}[1]{\arrowvec{\mathbf{#1}}}
\renewcommand{\vec}{\mathbf}
\newcommand{\veci}{{\boldsymbol{\hat{\imath}}}}
\newcommand{\vecj}{{\boldsymbol{\hat{\jmath}}}}
\newcommand{\veck}{{\boldsymbol{\hat{k}}}}
\newcommand{\vecl}{\boldsymbol{\l}}
\newcommand{\uvec}[1]{\mathbf{\hat{#1}}}
\newcommand{\utan}{\mathbf{\hat{t}}}
\newcommand{\unormal}{\mathbf{\hat{n}}}
\newcommand{\ubinormal}{\mathbf{\hat{b}}}

\newcommand{\dotp}{\bullet}
\newcommand{\cross}{\boldsymbol\times}
\newcommand{\grad}{\boldsymbol\nabla}
\newcommand{\divergence}{\grad\dotp}
\newcommand{\curl}{\grad\cross}
%\DeclareMathOperator{\divergence}{divergence}
%\DeclareMathOperator{\curl}[1]{\grad\cross #1}
\newcommand{\lto}{\mathop{\longrightarrow\,}\limits}

\renewcommand{\bar}{\overline}

\colorlet{textColor}{black} 
\colorlet{background}{white}
\colorlet{penColor}{blue!50!black} % Color of a curve in a plot
\colorlet{penColor2}{red!50!black}% Color of a curve in a plot
\colorlet{penColor3}{red!50!blue} % Color of a curve in a plot
\colorlet{penColor4}{green!50!black} % Color of a curve in a plot
\colorlet{penColor5}{orange!80!black} % Color of a curve in a plot
\colorlet{penColor6}{yellow!70!black} % Color of a curve in a plot
\colorlet{fill1}{penColor!20} % Color of fill in a plot
\colorlet{fill2}{penColor2!20} % Color of fill in a plot
\colorlet{fillp}{fill1} % Color of positive area
\colorlet{filln}{penColor2!20} % Color of negative area
\colorlet{fill3}{penColor3!20} % Fill
\colorlet{fill4}{penColor4!20} % Fill
\colorlet{fill5}{penColor5!20} % Fill
\colorlet{gridColor}{gray!50} % Color of grid in a plot

\newcommand{\surfaceColor}{violet}
\newcommand{\surfaceColorTwo}{redyellow}
\newcommand{\sliceColor}{greenyellow}




\pgfmathdeclarefunction{gauss}{2}{% gives gaussian
  \pgfmathparse{1/(#2*sqrt(2*pi))*exp(-((x-#1)^2)/(2*#2^2))}%
}


%%%%%%%%%%%%%
%% Vectors
%%%%%%%%%%%%%

%% Simple horiz vectors
\renewcommand{\vector}[1]{\left\langle #1\right\rangle}


%% %% Complex Horiz Vectors with angle brackets
%% \makeatletter
%% \renewcommand{\vector}[2][ , ]{\left\langle%
%%   \def\nextitem{\def\nextitem{#1}}%
%%   \@for \el:=#2\do{\nextitem\el}\right\rangle%
%% }
%% \makeatother

%% %% Vertical Vectors
%% \def\vector#1{\begin{bmatrix}\vecListA#1,,\end{bmatrix}}
%% \def\vecListA#1,{\if,#1,\else #1\cr \expandafter \vecListA \fi}

%%%%%%%%%%%%%
%% End of vectors
%%%%%%%%%%%%%

%\newcommand{\fullwidth}{}
%\newcommand{\normalwidth}{}



%% makes a snazzy t-chart for evaluating functions
%\newenvironment{tchart}{\rowcolors{2}{}{background!90!textColor}\array}{\endarray}

%%This is to help with formatting on future title pages.
\newenvironment{sectionOutcomes}{}{} 



%% Flowchart stuff
%\tikzstyle{startstop} = [rectangle, rounded corners, minimum width=3cm, minimum height=1cm,text centered, draw=black]
%\tikzstyle{question} = [rectangle, minimum width=3cm, minimum height=1cm, text centered, draw=black]
%\tikzstyle{decision} = [trapezium, trapezium left angle=70, trapezium right angle=110, minimum width=3cm, minimum height=1cm, text centered, draw=black]
%\tikzstyle{question} = [rectangle, rounded corners, minimum width=3cm, minimum height=1cm,text centered, draw=black]
%\tikzstyle{process} = [rectangle, minimum width=3cm, minimum height=1cm, text centered, draw=black]
%\tikzstyle{decision} = [trapezium, trapezium left angle=70, trapezium right angle=110, minimum width=3cm, minimum height=1cm, text centered, draw=black]


\title{Exponential functions}


\begin{document}
\begin{abstract}
  Exponential functions illuminated.
\end{abstract}
\maketitle

Exponential functions and their inverses logarithmic functions may seem somewhat esoteric at
first, but they model many phenomena in the real-world.




\section{What are exponential functions?}


\begin{definition}
  An \dfn{exponential function} is a function of the form
  \[
  f(x) = b^x
  \]
  where the \dfn{base} $b\ne 1$ is a positive real number. The domain of an
  exponential function is $(-\infty,\infty)$ and the range is $(0, \infty)$.
\end{definition}

\begin{question}
  Is $2^{-x}$ an exponential function?
  \begin{multipleChoice}
    \choice[correct]{yes}
    \choice{no}
  \end{multipleChoice}
  \begin{feedback}
    Note that
    \[
    2^{-x} = \left(2^{-1}\right)^x = \left(\frac{1}{2}\right)^x.
    \]
  \end{feedback}
\end{question}

%\textcolor{red}{MOVE TO INVERSE FUNCTIONS SECTION
%\begin{definition}
%  A \dfn{logarithmic function} is the inverse of an exponential:
%  \[
%  \log_b(x) = y \qquad\text{means that}\qquad b^y = x
%  \]
%  where  $b\ne 1$ is a positive real number. The domain of a
%  logarithmic function is $(0,\infty)$ and the range is $(- \infty, \infty)$.
%\end{definition}}

%In either definition above $b$ is called the \dfn{base}. Exponential and logarithmic functions are both continuous on their domains.
%
%
%\begin{question}
%   $\log_2(8)=\answer[given]{3}$
%
%  \begin{feedback}
%   $\log_2(8)$ will output the exponent that $2$ needs to get to $8$. Since $2^3=8$, we have $\log_2(8)=3$.
%   
%  \end{feedback}
%\end{question}

%\begin{question}
%   $\log_4(16)=\answer[given]{2}$
%
%  \begin{feedback}
%   $\log_4(16)$ will output the exponent that $4$ needs to get to $16$. Since $4^2=16$, we have $\log_4(16)=2$.
%   
%  \end{feedback}
%\end{question}

Remember that with exponential, there is one very special
base:
\[ e = 2.7182818284590\ldots \]
This is an irrational number that you will see frequently. The exponential with base $e$,
$f(x) = e^x$, is often called the `natural exponential' function because its special properties model many real-world phenomena. %\textcolor{red}{For the logarithm with base $e$,we have a special notation, $\ln (x)$ is `natural logarithm' function.} We'll talk about where $e$comes from when we talk about derivatives.


%\subsection{Connections between exponential functions and logarithms}
%
%\textcolor{red}{Let $b$ be a positive real number with $b\ne 1$.
%\begin{itemize}
%\item $b^{\log_b(x)} = x$ for all positive $x$
%\item $\log_b(b^x) = x$ for all real $x$
%\item $\log_b(x^n)=x\log_b(x)$ for all real $x$
%\end{itemize}}
%
%\begin{question}
%  What exponent makes the following expression true?
%  \[
%  3^x = e^{\left( x \cdot \answer{\ln (3)} \right)}.
%  \]
%  \begin{feedback}
%  Using the rules, we have $3^x=e^{\ln(3^x)}=e^{x\ln(3)}$.
%  
%  \end{feedback}
%\end{question}


\section{Properties of exponential functions}

Working with exponential and logarithmic functions is often simplified by  
applying properties of these functions.  These properties will make appearances 
throughout our work.

%\subsection{Properties of exponents}
Let $b$ be a positive real number with $b\ne 1$.
\begin{itemize}
  \item $b^m\cdot b^n = b^{m+n}$
  \item $b^{-1} = \frac{1}{b}$
  \item $\left(b^m\right)^n = b^{mn}$
\end{itemize}
\begin{question}
  What exponent makes the following true?
  \[
  2^4 \cdot 2^3 = 2^{\answer{7}}
  \]
  \begin{hint}
    \[
    (2^4) \cdot (2^3) = (2 \cdot 2\cdot 2 \cdot 2) \cdot  (2 \cdot 2\cdot 2)=2^{4+3}
    \]
  \end{hint}
\end{question}

\begin{question}
  What exponent makes the following true?
  \[
  (2^4)^3 = 2^{\answer{12}}
  \]
  \begin{hint}
    \[
    (2^4)^3 = 2^{3\cdot 4}=2^{12}
    \]
  \end{hint}
\end{question}

\begin{question}
  Compute
  \[
  2^{-3} = \answer{\frac{1}{8}}
  \]
  \begin{hint}
    \[
    2^{-3}=\frac{1}{2^3}=\frac{1}{8}
    \]
  \end{hint}
\end{question}




%\section{What can the graphs look like?}

\section{Graphs of exponential functions}

All exponential graphs exhibit the following properties.

\begin{itemize}
\item They are continuous on $(-\infty,\infty)$.
\item They cross the $y$-axis the point $(0,1)$.
\item They contain the point $(1,b)$.
\item They have a horizontal asymptote at $y=0$.
\item There are two basic graph shapes, depending on whether the $b$ in $b^x$ is between 0 and 1, or greater than 1.
\end{itemize}

If $0<b<1$, then the graph of $f(x)=b^x$ will be decreasing from left to right, with a steeper rate of decrease corresponding to smaller values of $b$. If $b>1$, then the graph of $f(x)=b^x$ will be increasing from left to right, with a steeper rate of increase corresponding to larger values of $b$. %Since $b^0=1$ for all $b\neq 0$, all exponential graphs pass through the point $(0,1)$.

\begin{example}
  Here we see the the graphs of four exponential functions.
  \begin{image}
    \begin{tikzpicture}
      \begin{axis}[
          domain=-2:2,
          xmin=-2, xmax=2,
          ymin=-.5, ymax=4,
          axis lines =middle, xlabel=$x$, ylabel=$y$,
          every axis y label/.style={at=(current axis.above origin),anchor=south},
          every axis x label/.style={at=(current axis.right of origin),anchor=west},
        ]
	\addplot [very thick, penColor, smooth] {e^x};
        \addplot [very thick, penColor2, smooth] {2^x)};
        \addplot [very thick, penColor3, smooth] {(1/2)^x)};
        \addplot [very thick, penColor4, smooth] {(1/3)^x)};
        
        
        
        \node at (axis cs:-1.5, 2 ) [penColor3,anchor=west] {$A$};
        \node at (axis cs:-.8, 2.6 ) [penColor4,anchor=west] {$B$};
        \node at (axis cs:0.6, 2.6 ) [penColor,anchor=west] {$C$};
        \node at (axis cs:1.2, 2 ) [penColor2,anchor=west] {$D$};
        
      \end{axis}
    \end{tikzpicture}
  \end{image}
  Match the curves $A$, $B$, $C$, and $D$ with the functions
  \[
 f(x)= e^x, \qquad g(x)=\left(\frac{1}{2}\right)^{x}, \qquad  h(x)=\left(\frac{1}{3}\right)^{x}, \qquad j(x)=2^{x}.
  \]
  \begin{explanation}
    Since all exponential graphs pass through the point $(1,b)$, one way to solve these problems is to compare these functions along the vertical line $x=1$.
    \begin{image}
      \begin{tikzpicture}
        \begin{axis}[
            domain=-2:2,
            xmin=-2, xmax=2,
            ymin=-.5, ymax=4,
            axis lines =middle, xlabel=$x$, ylabel=$y$,
            every axis y label/.style={at=(current axis.above origin),anchor=south},
            every axis x label/.style={at=(current axis.right of origin),anchor=west},
          ]
	  \addplot [very thick, penColor, smooth] {e^x}; %C
          \addplot [very thick, penColor2, smooth] {2^x)};%D
          \addplot [very thick, penColor3, smooth] {(1/2)^x)};%A
          \addplot [very thick, penColor4, smooth] {(1/3)^x)};%B
            
          \node at (axis cs:-1.5, 2 ) [penColor3,anchor=west] {$A$};
          \node at (axis cs:-.8, 2.6 ) [penColor4,anchor=west] {$B$};
          \node at (axis cs:0.6, 2.6 ) [penColor,anchor=west] {$C$};
          \node at (axis cs:1.2, 2 ) [penColor2,anchor=west] {$D$};

          \addplot [textColor, dashed] plot coordinates {(1,-.5) (1,4)};

          \addplot[color=penColor,fill=penColor,only marks,mark=*] coordinates{(1,e)}; %C
          \addplot[color=penColor2,fill=penColor2,only marks,mark=*] coordinates{(1,2)}; %D
          \addplot[color=penColor3,fill=penColor3,only marks,mark=*] coordinates{(1,1/2)}; %A
          \addplot[color=penColor4,fill=penColor4,only marks,mark=*] coordinates{(1,1/3)}; %B
        \end{axis}
      \end{tikzpicture}
    \end{image}
    Note
    \[
    \left(\frac{1}{3}\right)^1 < \left(\frac{1}{2}\right)^1  < 2^1 < e^1.
    \]
    Hence we see:
    \begin{itemize}
    \item $h(x)=\left(\frac{1}{3}\right)^{x}$ corresponds to
      $\answer[given]{B}$.
    \item $g(x)=\left(\frac{1}{2}\right)^{x}$ corresponds to $\answer[given]{A}$.
    \item $j(x)=2^x$ corresponds to $\answer[given]{D}$.
    \item $f(x)=e^x$ corresponds to $\answer[given]{C}$.
    \end{itemize}
  \end{explanation}
\end{example}

\begin{question}
Using the two graph types, compute the following limits.

$\displaystyle\lim_{x\rightarrow -\infty}e^x=\answer[given]{0}\\
\displaystyle\lim_{x\rightarrow \infty}e^x=\answer[given]{\infty}\\
\displaystyle\lim_{x\rightarrow -\infty}3^x=\answer[given]{0}\\
\displaystyle\lim_{x\rightarrow \infty}3^x=\answer[given]{\infty}\\
\displaystyle\lim_{x\rightarrow -\infty}\left(\frac{2}{3}\right)^x=\answer[given]{\infty}\\
\displaystyle\lim_{x\rightarrow \infty}\left(\frac{2}{3}\right)^x=\answer[given]{0}$
\end{question}

Notice that due to the consistency within the two graph types, if $b>1$, $\displaystyle\lim_{x\rightarrow -\infty}b^x=0$ and $\displaystyle\lim_{x\rightarrow \infty}b^x=\infty$, and if $0<b<1$, $\displaystyle\lim_{x\rightarrow -\infty}b^x=\infty$ and $\displaystyle\lim_{x\rightarrow \infty}b^x=0$.

We can tackle more complicated limits with this graph knowledge as well.

\begin{question}
Using the two graph types, compute the following limits.

$\displaystyle\lim_{x\rightarrow -\infty}e^{-x}=\lim_{x\rightarrow \infty}e^{x}=\answer[given]{\infty}\\
\displaystyle\lim_{x\rightarrow \infty}e^{-x}=\lim_{x\rightarrow -\infty}e^{x}=\answer[given]{0}\\
\displaystyle\lim_{x\rightarrow -\infty}e^{1/x}=\lim_{x\rightarrow 0^-}e^{x}=\answer[given]{1}\\
\displaystyle\lim_{x\rightarrow \infty}e^{1/x}=\lim_{x\rightarrow 0^+}e^{x}=\answer[given]{1}$
\end{question}

%\subsection{Graphs of logarithmic functions}
%
%
%\begin{example}
%  Here we see the the graphs of four logarithmic functions.
%  \begin{image}
%    \begin{tikzpicture}
%      \begin{axis}[
%          domain=0.05:4,
%          xmin=-.5, xmax=4,
%          ymin=-2, ymax=2,
%          axis lines =middle, xlabel=$x$, ylabel=$y$,
%          every axis y label/.style={at=(current axis.above origin),anchor=south},
%          every axis x label/.style={at=(current axis.right of origin),anchor=west},
%        ]
%	\addplot [very thick, penColor, smooth] {ln(x)}; % C
%        \addplot [very thick, penColor2, smooth] {ln(x)/ln(2)}; % D
%        \addplot [very thick, penColor3, smooth, samples=100] {ln(x)/ln(1/2))}; % A
%        \addplot [very thick, penColor4, smooth, samples=100] {ln(x)/ln(1/3))}; %B
%        
%        
%        \node at (axis cs:.5, 1.3 ) [penColor3,anchor=west] {$A$};
%        \node at (axis cs:.2, .5 ) [penColor4,anchor=west] {$B$};
%        \node at (axis cs:0.2, -.5 ) [penColor,anchor=west] {$C$};
%        \node at (axis cs:.5, -1.3 ) [penColor2,anchor=west] {$D$};
%        
%      \end{axis}
%    \end{tikzpicture}
%  \end{image}
%  Match the curves $A$, $B$, $C$, and $D$ with the functions
%  \[
%  \ln(x),\qquad \log_{1/2}(x), \qquad \log_{1/3}(x),\qquad \log_2(x).
%  \]
%  \begin{explanation}
%    First remember what $\log_b(x)=y$ means:
%    \[
%    \log_b(x) = y \qquad\text{means that}\qquad b^y = x.
%    \]
%    Moreover, $\ln(x) = \log_e(x)$ where $e= 2.71828\dots$.  So now
%    examine each of these functions along the horizontal line $y=1$
%    \begin{image}
%      \begin{tikzpicture}
%        \begin{axis}[
%            domain=0.05:4,
%            xmin=-.5, xmax=4,
%            ymin=-2, ymax=2,
%            axis lines =middle, xlabel=$x$, ylabel=$y$,
%            every axis y label/.style={at=(current axis.above origin),anchor=south},
%            every axis x label/.style={at=(current axis.right of origin),anchor=west},
%          ]
%	  \addplot [very thick, penColor, smooth] {ln(x)}; % C
%          \addplot [very thick, penColor2, smooth] {ln(x)/ln(2)}; % D
%          \addplot [very thick, penColor3, smooth, samples=100] {ln(x)/ln(1/2))}; % A
%          \addplot [very thick, penColor4, smooth, samples=100] {ln(x)/ln(1/3))}; %B
%          \addplot [dashed] {1};
%        
%          
%          \node at (axis cs:.5, 1.3 ) [penColor3,anchor=west] {$A$};
%          \node at (axis cs:.2, .5 ) [penColor4,anchor=west] {$B$};
%          \node at (axis cs:0.2, -.5 ) [penColor,anchor=west] {$C$};
%          \node at (axis cs:.5, -1.3 ) [penColor2,anchor=west] {$D$};
%
%          \addplot[color=penColor,fill=penColor,only marks,mark=*] coordinates{(e,1)}; %C
%          \addplot[color=penColor2,fill=penColor2,only marks,mark=*] coordinates{(2,1)}; %D
%          \addplot[color=penColor3,fill=penColor3,only marks,mark=*] coordinates{(1/2,1)}; %A
%          \addplot[color=penColor4,fill=penColor4,only marks,mark=*] coordinates{(1/3,1)}; %B
%        \end{axis}
%      \end{tikzpicture}
%    \end{image}
%    Note again (this is from the definition of a logarithm)
%    \[
%    \left(\frac{1}{3}\right)^1 < \left(\frac{1}{2}\right)^1  < 2^1 < e^1.
%    \]
%    Hence we see:
%    \begin{itemize}
%    \item $\log_{1/3}(x)$ corresponds to $\answer[given]{B}$.
%    \item $\log_{1/2}(x)$ corresponds to $\answer[given]{A}$.
%    \item $\log_2(x)$ corresponds to $\answer[given]{D}$.
%    \item $\ln(x)$ corresponds to $\answer[given]{C}$.
%    \end{itemize}
%  \end{explanation}
%\end{example}




%\subsection{Properties of logarithms}
%Let $b$ be a positive real number with $b\ne 1$.
%\begin{itemize}
%\item $\log_b(m\cdot n) = \log_b(m) + \log_b(n)$
%\item $\log_b(m^n) = n\cdot \log_b(m)$
%\item $\log_b\left(\frac{1}{m}\right) = \log_b(m^{-1}) = -\log_b(m)$
%\item $\log_a(m) = \frac{\log_b(m)}{\log_b(a)}$
%\end{itemize}
%
%\begin{question}
%  What value makes the following expression true?
%  \[
%  \log_2\left(\frac{8}{16}\right) = 3-\answer{4}
%  \]
%\end{question}
%
%
%\begin{question}
%  What makes the following expression true?
%  \[
%  \log_3(x) = \frac{\ln(x)}{\answer{\ln(3)}}
%  \]
%\end{question}
%
%
%\section{Exponential equations}
%Let's look into solving equations involving these functions.  We'll start with a straightforward example.
%\begin{example}
%	Solve the equation: $\displaystyle 4^x = 8$.
%	\begin{explanation}
%		We know $4$ and $8$ are each powers of $2$, we start by rewriting in terms of this base.
%		\[ 4^x = 2^{2x}  \,\,\, \textrm{ and } \,\,\, 8 =2^{\answer{3}} \,\,\,\, \textrm{ so } 2^{2x} = 2^{\answer{3}} \]
%		Since exponential functions are one-to-one, the only way for $a^m = a^n$ is if $m=n$.  In this case,
%		that means $\displaystyle 2x = 3$.
%		
%		The solution is: $\displaystyle x = \answer{3/2}$.
%	\end{explanation}
%\end{example}
%
%
%Of course, if we couldn't rewrite both sides with the same base, we can still use the properties of logarithms to solve.
%\begin{example}
%	Solve the equation: $\displaystyle 5^{2x-3} = 7$.
%	\begin{explanation}
%		Since we can't easily rewrite both sides as exponentials with the same base, we'll use logarithms instead.  Above we said that
%		$\log_b(x) = y$ means that $b^y = x$.  That statement means that each exponential equation has an equivalent logarithmic form
%		and vice-versa.  We'll convert to a logarithmic equation and solve from there.
%		\begin{align*}
%			5^{2x-3} &= 7\\
%			\log_{\answer{5}}\left(  \answer{7} \right) &= 2x-3
%		\end{align*}
%		From here, we can solve for $x$ directly.
%		\begin{align*}
%			2x &= \log_{5}\left(7\right) + 3\\
%			x &= \frac{\log_{5}\left(7\right) + 3}{2}
%		\end{align*}
%	\end{explanation} 
%\end{example}
%
%
%\begin{example}
%	Solve the equation: $\displaystyle e^{2x} = e^x + 6$. 
%	\begin{explanation}
%		Immediately taking logarithms of both sides will not help here, as the right side has multiple terms.  We know that logarithms do not
%		behave well with sums, but with products/quotients.  Instead, we notice that $e^{2x} = \left(e^x\right)^2$. (This is a common trick that
%		you will likely see many times.)  
%		\begin{align*}
%			e^{2x} &= e^x + 6\\
%			\left(e^x\right)^2 &= e^x + 6\\
%			\left(e^x\right)^2 - e^x - 6 &= 0
%		\end{align*}
%		Our equation is really a quadratic equation in $e^x$!  The left-hand side factors as $\left( e^x - \answer{3}\right) \left(e^x + \answer{2}\right)$, so we are dealing
%		with \[ e^x - \answer{3} = 0  \qquad  \textrm{and} \qquad e^x+\answer{2} = 0.\]
%		For the first:
%		\begin{align*}
%			e^x &= \answer{3}\\
%			x &= \ln\left( \answer{3}\right).
%		\end{align*}
%		
%		From the second: $\displaystyle e^x = \answer{-2}$.  Look back at the graph of $y=e^x$ above.  What was the range of the exponential function?  It didn't include any negative
%		numbers, so $e^x = -2$ has no solutions. 
%		
%		The solution to $\displaystyle e^{2x} = e^x + 6$ is $x = \answer{\ln(3)}$.
%	\end{explanation}
%\end{example}
%
%\begin{problem}
%	Solve the equation: $\displaystyle 2\left(5^{2x} + 6\right) = 11 \cdot 5^x$.
%	\begin{selectAll}
%		\choice[correct]{$\displaystyle \log_{5}\left(\dfrac{3}{2} \right)$}
%		\choice{$\displaystyle \frac{\ln\left(\dfrac{3}{2} \right)}{5}$}
%		\choice{$\displaystyle \log_{4}\left(5\right)$}
%		\choice[correct]{$\displaystyle \log_{5}\left(4 \right)$}
%		\choice{The equation has no solutions.}
%	\end{selectAll}
%\end{problem}
%
%\begin{example}
%	Solve the inequality: $\displaystyle \frac{6^x - 7 \cdot 3^x}{4^x - 15} \ge 0$.
%	\begin{explanation}
%		Since this isn't a linear inequality, we'll solve it using a sign-chart.  Luckily, the right-side is already $0$.  Let's factor the numerator on the left:
%		\begin{align*} 
%			6^x - 7 \cdot 3^x &= \left(2\cdot 3\right)^x - 7 \cdot 3^x \\
%				&= 2^x \cdot 3^x - 7 \cdot 3^x\\
%				&= 3^x \left( 2^x - 7 \right).
%		\end{align*}
%		That means we need to construct a sign chart for $\displaystyle \frac{3^x \left( 2^x - 7\right)}{4^x - 15}$.
%		(Note: $\log_{2}(7)$ is about $2.81$ and $\log_{4}(15)$ is about $1.95$.)
%		
%		\begin{center}
%		\begin{tikzpicture} 
%			\tkzTabInit[lgt=2,espcl=1] 
%				{$x$         /1, 
%				$3^x$   /1, 
%				$2^x-7$  /1,
%				$4^x-15$       /1}% 
%				{  , $\log_{4}(15)$ \,\,\,\,\, , \,\,\,\,\,\,$\log_{2}(7)$ ,  }% 
%			\tkzTabLine{ , + , d , +  , t , + ,}
%			\tkzTabLine{ , - , d , - , t , + ,}
%			\tkzTabLine{ , - , d ,  + , t , +, }
%		\end{tikzpicture} 
%		\end{center}
%		The solution is:
%		\[ \left( -\infty, \, \log_{4}(15) \, \right) \bigcup \left[ \,\log_{2}(7) \,, \infty \right) \]
%	\end{explanation}
%\end{example}
%\section{Logarithmic equations}
%
%\begin{example}
%	Solve the equation: $\displaystyle \log_5( 2x+1) = 3$.
%	\begin{explanation}
%		Our first step will be to rewrite this logarithmic equation into its exponential form.
%		\[ \log_5(2x+1) = 3 \qquad \textrm{ means } \qquad 2x+1 = 5^{\answer{3}} \]
%		From here we solve directly.
%		\begin{align*}
%			2x+1 &= \answer{125}\\
%			2x &= \answer{124}\\
%			x &= \answer{62}.
%		\end{align*}
%	\end{explanation}
%\end{example}
%
%
%
%\begin{example}
%	Solve the equation: \[ \log_3(2x+1) = 1-\log_3(x+2). \]
%	\begin{explanation}
%		With more than one logarithm, we'll typically try to use the properties of logarithms to combine them into a single term.
%		\begin{align*}
%			\log_3(2x+1) &= 1-\log_3(x+2) \\
%			\log_3(2x+1) + \log_3(x+2) &= 1\\
%			\log_3\left( (2x+1)(x+2)\right) &= 1\\
%			\log_3\left( 2x^2 + 5x + 2 \right) &= 1\\
%			2x^2 + 5x + 2 &= 3\\
%			2x^2 + 5x - 1 &= 0
%		\end{align*}
%		Let's use quadratic formula to solve this.
%		\[ x = \frac{-5 \pm \sqrt{5^2 - 4 \cdot 2\cdot -1}}{2 \cdot 2} = \frac{ -5\pm \sqrt{ \answer{33} } }{4}. \]
%		
%		What happens if we try to plug $x = \dfrac{-5 - \sqrt{33}}{4}$ into the equation?  Both $2\left( \dfrac{-5-\sqrt{33}}{4} \right) + 1$ and $\dfrac{-5-\sqrt{33}}{4} + 2$ are 
%		negative.  That means, the logarithms of these values is not defined.  
%		
%		It turns out that $\dfrac{-5 - \sqrt{33}}{4}$ is a solution of the equation $2x^2+5x-1 = 0$,
%		but not a solution of the original equation $\log_3(2x+1) = 1-\log_3(x+2)$. 
%		
%		When working with logarithmic equations, we must always check that the solutions we find 
%		actually satisfy the original equation.
%		
%		The only solution is $x = \dfrac{-5 + \sqrt{33}}{4}$.
%		
%	\end{explanation}		
%\end{example}

\subsection{Learning Objectives}
After completing this section, students should be able to:
\vspace{.05in}

\noindent$\bullet$ Sketch the graph of an exponential function.
\\$\bullet$ Compute limits of an exponential function.
\\$\bullet$ Combine and simplify exponential expressions.



\end{document}