\documentclass{ximera}

%\usepackage{todonotes}

\newcommand{\todo}{}

\usepackage{esint} % for \oiint
\graphicspath{
  {./}
  {ximeraTutorial/}
  {basicPhilosophy/}
  {functionsOfSeveralVariables/}
  {normalVectors/}
  {lagrangeMultipliers/}
  {vectorFields/}
  {greensTheorem/}
  {shapeOfThingsToCome/}
}
\usepackage{comment} %% used in what is a limit
\usepackage[valunder]{signchart} %% used in graphing sign chart

\newcommand{\mooculus}{\textsf{\textbf{MOOC}\textnormal{\textsf{ULUS}}}}

\usepackage{tkz-euclide}
\tikzset{>=stealth} %% cool arrow head
\tikzset{shorten <>/.style={ shorten >=#1, shorten <=#1 } } %% allows shorter vectors


\pgfplotsset{soldot/.style={color=black,only marks,mark=*}} %% USED by piecewise functions
\pgfplotsset{holdot/.style={color=black,fill=white,only marks,mark=*}}
\usetikzlibrary{arrows.meta}


\usepackage{tkz-tab}  %% sign charts
\usepackage{polynom}

\usetikzlibrary{backgrounds} %% for boxes around graphs
\usetikzlibrary{shapes,positioning}  %% Clouds and stars
\usetikzlibrary{matrix} %% for matrix
\usepgfplotslibrary{polar} %% for polar plots
%\usetkzobj{all}
\usepackage[makeroom]{cancel} %% for strike outs
%\usepackage{mathtools} %% for pretty underbrace % Breaks Ximera
\usepackage{multicol}
\usepackage{pgffor} %% required for integral for loops


%% http://tex.stackexchange.com/questions/66490/drawing-a-tikz-arc-specifying-the-center
%% Draws beach ball
\tikzset{pics/carc/.style args={#1:#2:#3}{code={\draw[pic actions] (#1:#3) arc(#1:#2:#3);}}}



\usepackage{array}
\setlength{\extrarowheight}{+.1cm}   
\newdimen\digitwidth
\settowidth\digitwidth{9}
\def\divrule#1#2{
\noalign{\moveright#1\digitwidth
\vbox{\hrule width#2\digitwidth}}}




\newcommand{\RR}{\mathbb R}
\newcommand{\R}{\mathbb R}
\newcommand{\N}{\mathbb N}
\newcommand{\Z}{\mathbb Z}

\newcommand{\sagemath}{\textsf{SageMath}}


%\renewcommand{\d}{\,d\!}
\renewcommand{\d}{\mathop{}\!d}
\newcommand{\dd}[2][]{\frac{\d #1}{\d #2}}
\newcommand{\pp}[2][]{\frac{\partial #1}{\partial #2}}
\renewcommand{\l}{\ell}
\newcommand{\ddx}{\frac{d}{\d x}}
\newcommand{\ddt}{\frac{d}{\d t}}


\newcommand{\zeroOverZero}{\ensuremath{\boldsymbol{\tfrac{0}{0}}}}
\newcommand{\inftyOverInfty}{\ensuremath{\boldsymbol{\tfrac{\infty}{\infty}}}}
\newcommand{\zeroOverInfty}{\ensuremath{\boldsymbol{\tfrac{0}{\infty}}}}
\newcommand{\zeroTimesInfty}{\ensuremath{\small\boldsymbol{0\cdot \infty}}}
\newcommand{\inftyMinusInfty}{\ensuremath{\small\boldsymbol{\infty - \infty}}}
\newcommand{\oneToInfty}{\ensuremath{\boldsymbol{1^\infty}}}
\newcommand{\zeroToZero}{\ensuremath{\boldsymbol{0^0}}}
\newcommand{\inftyToZero}{\ensuremath{\boldsymbol{\infty^0}}}



\newcommand{\numOverZero}{\ensuremath{\boldsymbol{\tfrac{\#}{0}}}}
\newcommand{\dfn}{\textbf}
%\newcommand{\unit}{\,\mathrm}
\newcommand{\unit}{\mathop{}\!\mathrm}
\newcommand{\eval}[1]{\bigg[ #1 \bigg]}
\newcommand{\seq}[1]{\left( #1 \right)}
\renewcommand{\epsilon}{\varepsilon}
\renewcommand{\phi}{\varphi}


\renewcommand{\iff}{\Leftrightarrow}

\DeclareMathOperator{\arccot}{arccot}
\DeclareMathOperator{\arcsec}{arcsec}
\DeclareMathOperator{\arccsc}{arccsc}
\DeclareMathOperator{\si}{Si}
\DeclareMathOperator{\proj}{\vec{proj}}
\DeclareMathOperator{\scal}{scal}
\DeclareMathOperator{\sign}{sign}


%% \newcommand{\tightoverset}[2]{% for arrow vec
%%   \mathop{#2}\limits^{\vbox to -.5ex{\kern-0.75ex\hbox{$#1$}\vss}}}
\newcommand{\arrowvec}{\overrightarrow}
%\renewcommand{\vec}[1]{\arrowvec{\mathbf{#1}}}
\renewcommand{\vec}{\mathbf}
\newcommand{\veci}{{\boldsymbol{\hat{\imath}}}}
\newcommand{\vecj}{{\boldsymbol{\hat{\jmath}}}}
\newcommand{\veck}{{\boldsymbol{\hat{k}}}}
\newcommand{\vecl}{\boldsymbol{\l}}
\newcommand{\uvec}[1]{\mathbf{\hat{#1}}}
\newcommand{\utan}{\mathbf{\hat{t}}}
\newcommand{\unormal}{\mathbf{\hat{n}}}
\newcommand{\ubinormal}{\mathbf{\hat{b}}}

\newcommand{\dotp}{\bullet}
\newcommand{\cross}{\boldsymbol\times}
\newcommand{\grad}{\boldsymbol\nabla}
\newcommand{\divergence}{\grad\dotp}
\newcommand{\curl}{\grad\cross}
%\DeclareMathOperator{\divergence}{divergence}
%\DeclareMathOperator{\curl}[1]{\grad\cross #1}
\newcommand{\lto}{\mathop{\longrightarrow\,}\limits}

\renewcommand{\bar}{\overline}

\colorlet{textColor}{black} 
\colorlet{background}{white}
\colorlet{penColor}{blue!50!black} % Color of a curve in a plot
\colorlet{penColor2}{red!50!black}% Color of a curve in a plot
\colorlet{penColor3}{red!50!blue} % Color of a curve in a plot
\colorlet{penColor4}{green!50!black} % Color of a curve in a plot
\colorlet{penColor5}{orange!80!black} % Color of a curve in a plot
\colorlet{penColor6}{yellow!70!black} % Color of a curve in a plot
\colorlet{fill1}{penColor!20} % Color of fill in a plot
\colorlet{fill2}{penColor2!20} % Color of fill in a plot
\colorlet{fillp}{fill1} % Color of positive area
\colorlet{filln}{penColor2!20} % Color of negative area
\colorlet{fill3}{penColor3!20} % Fill
\colorlet{fill4}{penColor4!20} % Fill
\colorlet{fill5}{penColor5!20} % Fill
\colorlet{gridColor}{gray!50} % Color of grid in a plot

\newcommand{\surfaceColor}{violet}
\newcommand{\surfaceColorTwo}{redyellow}
\newcommand{\sliceColor}{greenyellow}




\pgfmathdeclarefunction{gauss}{2}{% gives gaussian
  \pgfmathparse{1/(#2*sqrt(2*pi))*exp(-((x-#1)^2)/(2*#2^2))}%
}


%%%%%%%%%%%%%
%% Vectors
%%%%%%%%%%%%%

%% Simple horiz vectors
\renewcommand{\vector}[1]{\left\langle #1\right\rangle}


%% %% Complex Horiz Vectors with angle brackets
%% \makeatletter
%% \renewcommand{\vector}[2][ , ]{\left\langle%
%%   \def\nextitem{\def\nextitem{#1}}%
%%   \@for \el:=#2\do{\nextitem\el}\right\rangle%
%% }
%% \makeatother

%% %% Vertical Vectors
%% \def\vector#1{\begin{bmatrix}\vecListA#1,,\end{bmatrix}}
%% \def\vecListA#1,{\if,#1,\else #1\cr \expandafter \vecListA \fi}

%%%%%%%%%%%%%
%% End of vectors
%%%%%%%%%%%%%

%\newcommand{\fullwidth}{}
%\newcommand{\normalwidth}{}



%% makes a snazzy t-chart for evaluating functions
%\newenvironment{tchart}{\rowcolors{2}{}{background!90!textColor}\array}{\endarray}

%%This is to help with formatting on future title pages.
\newenvironment{sectionOutcomes}{}{} 



%% Flowchart stuff
%\tikzstyle{startstop} = [rectangle, rounded corners, minimum width=3cm, minimum height=1cm,text centered, draw=black]
%\tikzstyle{question} = [rectangle, minimum width=3cm, minimum height=1cm, text centered, draw=black]
%\tikzstyle{decision} = [trapezium, trapezium left angle=70, trapezium right angle=110, minimum width=3cm, minimum height=1cm, text centered, draw=black]
%\tikzstyle{question} = [rectangle, rounded corners, minimum width=3cm, minimum height=1cm,text centered, draw=black]
%\tikzstyle{process} = [rectangle, minimum width=3cm, minimum height=1cm, text centered, draw=black]
%\tikzstyle{decision} = [trapezium, trapezium left angle=70, trapezium right angle=110, minimum width=3cm, minimum height=1cm, text centered, draw=black]


\title{2.5 - Limits at infinity and horizontal asymptotes}



\begin{document}
\begin{abstract}
We explore functions that behave like horizontal lines as the input
grows without bound.
\end{abstract}
\maketitle

In a limit, the input can approach $\infty$ or $-\infty$ instead of a number. The output of such a limit conveys the behavior of the graph at its far right or far left end.

\begin{definition}\label{def:limitAtInfty}\index{limit!at infinity}
If, as we plug in values of $x$ increasingly large in magnitude, the corresponding outputs get as close as desired to $L$, we write
\[
\lim_{x\to \infty} f(x) = L
\]
and we say, the \dfn{limit at infinity} of $f(x)$ is $L$. Equivalently, as $x\rightarrow\infty$, $f(x)\rightarrow L$.

If, as we plug in negative values of $x$ increasingly large in magnitude, the corresponding outputs get as close as desired to $L$, we write
\[
\lim_{x\to -\infty} f(x) = L
\]
and we say, the \dfn{limit at negative infinity} of $f(x)$ is $L$. Equivalently, as $x\rightarrow -\infty$, $f(x)\rightarrow L$. 
\end{definition}

Since we already know a lot about the end behavior of rational functions, we begin there.

\begin{example}
	Consider $\displaystyle\lim_{x\rightarrow\infty}\dfrac{1}{x^2}$ and $\displaystyle\lim_{x\rightarrow -\infty}\dfrac{1}{x^2}$.
	\begin{explanation}
		From the Rational Functions section, we know that for large magnitude $x$, the outputs of a function like this will approach $0$. Thus we conclude that $\displaystyle\lim_{x\rightarrow\infty}\dfrac{1}{x^2}=0$ and $\displaystyle\lim_{x\rightarrow -\infty}\dfrac{1}{x^2}=0$.
	\end{explanation}
\end{example}

\begin{example}
	Consider $\displaystyle\lim_{x\rightarrow\infty}\dfrac{6x^2-12x}{x^2+1}$.
  \begin{explanation} Since $$\dfrac{6x^2-12x}{x^2+1}=\dfrac{x^2(6-\frac{12}{x})}{x^2(1+\frac{1}{x^2})},$$ we can see that $f(x)=\dfrac{6x^2-12x}{x^2+1}$ behaves like $y=\frac{6x^2}{x^2}=6$ for large magnitude $x$. Thus we conclude that $\displaystyle\lim_{x\rightarrow\infty}\dfrac{6x^2-12x}{x^2+1}=\answer[given]{6}$.
  \end{explanation}
\end{example}

\begin{example}
	Consider $\displaystyle\lim_{x\rightarrow\infty}\dfrac{x^5+7x^2}{4x^2+x-5}$.
  \begin{explanation} Since the degree of the numerator is larger than that of the denominator, we know that this function does not have a horizontal asymptote. That is, the outputs of the function don't approach any particular value for large magnitude $x$. Thus, this limit does not exist.
  \end{explanation}
\end{example}

The connection between horizontal asymptotes and limits at infinity is strong. In fact, we can update our definition of a horizontal asymptote to include our new limit notation.

\begin{definition}\label{def:horiz asymptote}\index{asymptote!horizontal}\index{horizontal asymptote}
If  
\[
\lim_{x\to \infty} f(x) = L \qquad\text{or}\qquad \lim_{x\to -\infty} f(x) = L,
\]
then the line $y=L$ is a \dfn{horizontal asymptote} of $f(x)$.
\end{definition}

This means that our method of computing the horizontal asymptote of a rational function is now our method for computing their limits at positive and negative infinity! 

\begin{example} 
Give the horizontal asymptotes of
\[
f(x) = \frac{6x-9}{x-1}
\]
\begin{explanation}
We compute $\lim_{x\to \infty} f(x) = 6$, so we conclude that $y=6$ is a horizontal asymptote of the function. Since $\lim_{x\to -\infty} f(x) =6$ as well, this is the only horizontal asymptote of $f(x)$.
\end{explanation}
\end{example}

Recall, however, that rational functions aren't the only functions with horizontal asymptotes. We now turn our attention to limits at infinity of non-rational functions.
\\
\\First, we make an update to our rule that rationals of the form $\frac{1}{x^n}$ for $n>0$, where defined, shrink to $0$ for large magnitude $x$. In general, any expression with a constant numerator and a denominator getting larger without bound will shrink to $0$ for the same reasons that $\frac{1}{x^n}$ will. This means we can compute limits like the following without much work required.

\begin{example} $$\displaystyle\lim_{x\rightarrow\infty}\frac{2}{\sqrt{x}}=\answer[given]{0}$$
 $$\displaystyle\lim_{x\rightarrow\infty}\frac{5}{3x^{2/3}}=\answer[given]{0}$$
 $$\displaystyle\lim_{x\rightarrow -\infty}\frac{100}{\sqrt[3]{x}}=\answer[given]{0}$$
 $$\displaystyle\lim_{x\rightarrow -\infty}\frac{7}{xe^x}=\answer[given]{0}$$
  $$\mbox{and }\displaystyle\lim_{x\rightarrow \infty}\frac{6+\frac{1}{e^x}}{3-\frac{1}{x}}=\lim_{x\rightarrow \infty}\frac{6+\answer[given]{0}}{3-\answer[given]{0}}=\answer[given]{2}\mbox{  using limit laws.}$$
\end{example}

We also have a few reliable ``tricks" for specific situations. For example, in limits at infinity of some ratios, it is useful to divide the numerator and denominator by some $x^n$.

\begin{example}
$$\displaystyle\lim_{x\rightarrow \infty}\frac{\sqrt{5x^4-3x^2+1}}{3x^2+4x+5}$$

In this case we divide the numerator and denominator by $x^2$, which is the highest power of $x$ in the denominator.
\begin{align*}
\displaystyle\lim_{x\rightarrow \infty}\frac{\sqrt{5x^4-3x^2+1}}{3x^2+4x+5}&=\lim_{x\rightarrow \infty}\frac{\sqrt{5x^4-3x^2+1}}{3x^2+4x+5}\cdot\frac{1/x^2}{1/x^2}\\
&=\lim_{x\rightarrow \infty}\frac{\sqrt{1/x^4}\sqrt{5x^4-3x^2+1}}{(3x^2+4x+5)/x^2}\\
&=\lim_{x\rightarrow \infty}\frac{\sqrt{5-3/x^2+1/x^4}}{3+4/x+5/x^2}\\
&=\lim_{x\rightarrow \infty}\frac{\sqrt{5-\answer[given]{0}+\answer[given]{0}}}{3+\answer[given]{0}+\answer[given]{0}}\\
&=\frac{\sqrt{\answer[given]{5}}}{\answer[given]{3}}\end{align*}
\vspace{.2in}

\textbf{Alternative:} As an alternative to the method above, we can also compute this limit by factoring $\sqrt{x^4}=x^2$ (for positive $x$) out of the numerator and denominator instead of multiplying anything in. As you might imagine, we arrive at the same conclusion either way.
\begin{align*}
\displaystyle\lim_{x\rightarrow \infty}\frac{\sqrt{5x^4-3x^2+1}}{3x^2+4x+5}&=\lim_{x\rightarrow \infty}\frac{\sqrt{x^4(5-3/x^2+1/x^4)}}{x^2(3+4/x+5/x^2)}\\
&=\lim_{x\rightarrow \infty}\frac{\cancel{x^2}\sqrt{5-3/x^2+1/x^4}}{\cancel{x^2}(3+4/x+5/x^2)}\\
&=\lim_{x\rightarrow \infty}\frac{\sqrt{5-3/x^2+1/x^4}}{3+4/x+5/x^2}\\
&=\lim_{x\rightarrow \infty}\frac{\sqrt{5-\answer[given]{0}+\answer[given]{0}}}{3+\answer[given]{0}+\answer[given]{0}}\\
&=\frac{\sqrt{\answer[given]{5}}}{\answer[given]{3}}\end{align*}
\end{example}

We must be particularly careful in this process when $x\rightarrow -\infty$, as shown below.

\begin{example}

\[
\lim_{x\to -\infty} \frac{x^3+1}{\sqrt{x^6+5}}
\]
\begin{explanation}
In this case we divide the numerator and denominator by $x^3$; this is the highest power of $x$ we can use that won't reduce the entire denominator to zeroes. However, since $x$ is negative here we have $1/x^3=-\sqrt{1/x^6}$.
\begin{align*}
\lim_{x\to -\infty} \frac{x^3+1}{\sqrt{x^6+5}} &= \lim_{x\to -\infty} \frac{x^3+1}{\sqrt{x^6+5}} \cdot \frac{1/x^3}{1/x^3}\\
&= \lim_{x\to -\infty} \frac{1+1/x^3}{-\sqrt{x^6/x^6+5/x^6}}\\
&= \lim_{x\to -\infty} \frac{1+1/x^3}{-\sqrt{1+5/x^6}}\\
&= \lim_{x\to -\infty}\frac{1+0}{-\sqrt{1+0}}\\
&= \frac{1+0}{-\sqrt{1+0}}\\
&= -1.
\end{align*}
\end{explanation}
\end{example}

Notice that if the limit in the above example had gone to positive infinity, we would have gotten a different result ($1$). This is our first example of a non-rational function that can have two different horizontal asymptotes. 
\\
\\We can adapt this method to other situations as needed.

\begin{example}
$$\displaystyle\lim_{x\rightarrow \infty}\frac{5e^{2x}+3e^x}{9+e^{2x}}$$

In this case we divide the numerator and denominator by $e^{2x}$, which is the dominant term in the denominator.
\begin{align*}
\displaystyle\lim_{x\rightarrow \infty}\frac{5e^{2x}+3e^x}{9+e^{2x}}&=\lim_{x\rightarrow \infty}\frac{5e^{2x}+3e^x}{9+e^{2x}}\cdot\frac{1/e^{2x}}{1/e^{2x}}\\
&=\lim_{x\rightarrow \infty}\frac{5+3/e^x}{9/e^{2x}+1}\\
&=\lim_{x\rightarrow \infty}\frac{5+\answer[given]{0}}{\answer[given]{0}+1}\\
&=\answer[given]{5}\end{align*}

\end{example}

We can also adapt the conjugate trick to limits at infinity that take the form $\infty-\infty$, as shown below.

\begin{example}
$$\displaystyle\lim_{x\rightarrow \infty}\left(x^2-\sqrt{x^4+3}\right)$$

In this case we multiply and divide by the conjugate.
\begin{align*}
\displaystyle\lim_{x\rightarrow \infty}\left(x^2-\sqrt{x^4+3}\right)&=\lim_{x\rightarrow \infty}(x^2-\sqrt{x^4+3})\cdot\dfrac{x^2+\sqrt{x^4+3}}{x^2+\sqrt{x^4+3}}\\
&=\lim_{x\rightarrow \infty}\frac{(x^2)^2-(\sqrt{x^4+3})^2}{x^2+\sqrt{x^4+3}}\\
&=\lim_{x\rightarrow \infty}\frac{x^4-(x^4+3)}{x^2+\sqrt{x^4+3}}\\
&=\lim_{x\rightarrow \infty}\frac{-3}{x^2+\sqrt{x^4+3}}\\
&=0\end{align*}

\end{example}


Also note that since
\(
\lim_{x\to \infty} f(x) = \lim_{x\to 0^+} f\left(\frac{1}{x}\right)
\)
and
\(
\lim_{x\to -\infty} f(x) = \lim_{x\to 0^-} f\left(\frac{1}{x}\right)
\)
we can apply the Squeeze Theorem when taking limits at infinity.
Here is an example of a limit at infinity that uses the Squeeze
Theorem, and shows that functions can, in fact, cross their horizontal
asymptotes.


\begin{example}
Compute $\lim_{x\to \infty} f(x)$ given that for all $x>0$, $$\frac{4x-1}{x}\leq f(x)\leq \frac{4x+1}{x}.$$

\begin{image}
\begin{tikzpicture}
	\begin{axis}[
            domain=2:20,
            ymax=5,
            ymin=3,
            samples=100,
            axis lines =middle, xlabel=$x$, ylabel=$y$,
            every axis y label/.style={at=(current axis.above origin),anchor=south},
            every axis x label/.style={at=(current axis.right of origin),anchor=west}
          ]
	  \addplot [dashed, penColor, smooth] {(1/x) * 1+4};
	  	  \addplot [dashed, penColor, smooth] {(1/x) * -1+4};
	  	  	  \addplot [very thick, penColor, smooth] {(1/x) * sin(deg(7*x))+4};
        \end{axis}
\end{tikzpicture}
\end{image}

In the image above, $f(x)$ is the oscillating function, with $g(x)=\frac{4x-1}{x}$ below and $h(x)=\frac{4x+1}{x}$ above it. In this case the exact formula for $f(x)$ involves trigonometry, which we will cover later in the course.

\begin{explanation}
We can bound our function
\[
g(x)\leq f(x)\leq h(x).
\]
Then
\begin{align*}
\lim_{x\to\infty}g(x) &= \lim_{x\to\infty}\frac{4-1/x}{1}\\
&=\answer[given]{4}.
\end{align*}
And we also have
\begin{align*}
\lim_{x\to\infty}\frac{4x+1}{x} \cdot \frac{1/x}{1/x} &= \lim_{x\to\infty}\frac{4+1/x}{1}\\
&=\answer[given]{4}.
\end{align*}
Since 
\[
\lim_{x\to \infty} \frac{4x-1}{x}  = \answer[given]{4} = \lim_{x\to \infty}\frac{4x+1}{x} 
\] 
we conclude by the Squeeze Theorem,
$\lim_{x\to\infty}f(x) = \answer[given]{4}$.
\end{explanation}
\end{example}











%It is a common misconception that a function cannot cross an
%asymptote. As the next example shows, a function can cross a horizontal
%asymptote, and in the example this occurs an infinite number of times!
%
%\begin{example}
%Give a horizontal asymptote of
%\[
%	f(x) = \frac{\sin(7x)+4x}{x}.
%\]
%\begin{explanation}
%Again from previous work, we see that $\lim_{x\to \infty} f(x) =
%\answer{4}$. Hence $y=\answer{4}$ is a horizontal asymptote of $f(x)$.
%\end{explanation}
%\end{example}

%CUT THIS BECAUSE LIM=INF HASNT BEEN COVERED YET

%We conclude with an infinite limit at infinity.
%
%\begin{example}
%Compute
%\[ \lim_{x\to\infty}\sqrt{x}. \]
%\begin{image}
%\begin{tikzpicture}
%	\begin{axis}[
%            domain=0:10,
%            ymax=4,
%            ymin=-1,
%            samples=100,
%            axis lines =middle, xlabel=$x$, ylabel=$y$,
%            every axis y label/.style={at=(current axis.above origin),anchor=south},
%            every axis x label/.style={at=(current axis.right of origin),anchor=west}
%          ]
%	  \addplot [very thick, penColor, smooth] {x^(1/2)};
%        \end{axis}
%\end{tikzpicture}
%\end{image}
%
%\begin{explanation}
%The function $\sqrt{x}$ grows slowly, and seems like it may have a
%horizontal asymptote, see the graph above. However, if we consider the
%square root as the inverse of the function $f(x) = x^2 , \quad x \geq 0$.
%
%\begin{quote}
%  $\sqrt{x} = y$ means that $y^2 = x$ and that $y \geq 0$.
%\end{quote}
%
%We see that we may square higher and higher values to obtain
%larger outputs.  This means that $\sqrt{x}$ is unbounded, and hence
%$\displaystyle \lim_{x\to\infty}\sqrt{x}=\infty$.
%\end{explanation}
%\end{example}

\subsection{Learning Objectives}
After completing this section, students should be able to:
\vspace{.05in}

\noindent$\bullet$ understand the limit definition of a horizontal asymptote
\\$\bullet$ compute limits at $\pm\infty$ of rational functions
\\$\bullet$ compute limits at $\pm\infty$ of non-rational functions
\\$\bullet$ divide the numerator and denominator of a ratio by a carefully chosen term to compute the limit
\\$\bullet$ adapt this strategy in similar scenarios
\\$\bullet$ use the conjugate trick to compute limits of the form $\infty-\infty$
\\$\bullet$ use the Squeeze Theorem in limits at $\pm\infty$

%END WITH LEARNING OBJECTIVES
%\outcome{Find horizontal asymptotes using limits.}
%\outcome{Recognize that a curve can cross a horizontal asymptote.}
%\outcome{Calculate the limit as $x$ approaches $\pm \infty$ of common functions algebraically.}
%\outcome{Find the limit as $x$ approaches $\pm \infty$ from a graph.}
%\outcome{Define a horizontal asymptote.}
%\outcome{Compute limits at infinity of famous functions.}
%\outcome{Identify horizontal asymptotes by looking at a graph.}

\end{document}