\documentclass{ximera}

%\usepackage{todonotes}

\newcommand{\todo}{}

\usepackage{esint} % for \oiint
\graphicspath{
  {./}
  {ximeraTutorial/}
  {basicPhilosophy/}
  {functionsOfSeveralVariables/}
  {normalVectors/}
  {lagrangeMultipliers/}
  {vectorFields/}
  {greensTheorem/}
  {shapeOfThingsToCome/}
}
\usepackage{comment} %% used in what is a limit
\usepackage[valunder]{signchart} %% used in graphing sign chart

\newcommand{\mooculus}{\textsf{\textbf{MOOC}\textnormal{\textsf{ULUS}}}}

\usepackage{tkz-euclide}
\tikzset{>=stealth} %% cool arrow head
\tikzset{shorten <>/.style={ shorten >=#1, shorten <=#1 } } %% allows shorter vectors


\pgfplotsset{soldot/.style={color=black,only marks,mark=*}} %% USED by piecewise functions
\pgfplotsset{holdot/.style={color=black,fill=white,only marks,mark=*}}
\usetikzlibrary{arrows.meta}


\usepackage{tkz-tab}  %% sign charts
\usepackage{polynom}

\usetikzlibrary{backgrounds} %% for boxes around graphs
\usetikzlibrary{shapes,positioning}  %% Clouds and stars
\usetikzlibrary{matrix} %% for matrix
\usepgfplotslibrary{polar} %% for polar plots
%\usetkzobj{all}
\usepackage[makeroom]{cancel} %% for strike outs
%\usepackage{mathtools} %% for pretty underbrace % Breaks Ximera
\usepackage{multicol}
\usepackage{pgffor} %% required for integral for loops


%% http://tex.stackexchange.com/questions/66490/drawing-a-tikz-arc-specifying-the-center
%% Draws beach ball
\tikzset{pics/carc/.style args={#1:#2:#3}{code={\draw[pic actions] (#1:#3) arc(#1:#2:#3);}}}



\usepackage{array}
\setlength{\extrarowheight}{+.1cm}   
\newdimen\digitwidth
\settowidth\digitwidth{9}
\def\divrule#1#2{
\noalign{\moveright#1\digitwidth
\vbox{\hrule width#2\digitwidth}}}




\newcommand{\RR}{\mathbb R}
\newcommand{\R}{\mathbb R}
\newcommand{\N}{\mathbb N}
\newcommand{\Z}{\mathbb Z}

\newcommand{\sagemath}{\textsf{SageMath}}


%\renewcommand{\d}{\,d\!}
\renewcommand{\d}{\mathop{}\!d}
\newcommand{\dd}[2][]{\frac{\d #1}{\d #2}}
\newcommand{\pp}[2][]{\frac{\partial #1}{\partial #2}}
\renewcommand{\l}{\ell}
\newcommand{\ddx}{\frac{d}{\d x}}
\newcommand{\ddt}{\frac{d}{\d t}}


\newcommand{\zeroOverZero}{\ensuremath{\boldsymbol{\tfrac{0}{0}}}}
\newcommand{\inftyOverInfty}{\ensuremath{\boldsymbol{\tfrac{\infty}{\infty}}}}
\newcommand{\zeroOverInfty}{\ensuremath{\boldsymbol{\tfrac{0}{\infty}}}}
\newcommand{\zeroTimesInfty}{\ensuremath{\small\boldsymbol{0\cdot \infty}}}
\newcommand{\inftyMinusInfty}{\ensuremath{\small\boldsymbol{\infty - \infty}}}
\newcommand{\oneToInfty}{\ensuremath{\boldsymbol{1^\infty}}}
\newcommand{\zeroToZero}{\ensuremath{\boldsymbol{0^0}}}
\newcommand{\inftyToZero}{\ensuremath{\boldsymbol{\infty^0}}}



\newcommand{\numOverZero}{\ensuremath{\boldsymbol{\tfrac{\#}{0}}}}
\newcommand{\dfn}{\textbf}
%\newcommand{\unit}{\,\mathrm}
\newcommand{\unit}{\mathop{}\!\mathrm}
\newcommand{\eval}[1]{\bigg[ #1 \bigg]}
\newcommand{\seq}[1]{\left( #1 \right)}
\renewcommand{\epsilon}{\varepsilon}
\renewcommand{\phi}{\varphi}


\renewcommand{\iff}{\Leftrightarrow}

\DeclareMathOperator{\arccot}{arccot}
\DeclareMathOperator{\arcsec}{arcsec}
\DeclareMathOperator{\arccsc}{arccsc}
\DeclareMathOperator{\si}{Si}
\DeclareMathOperator{\proj}{\vec{proj}}
\DeclareMathOperator{\scal}{scal}
\DeclareMathOperator{\sign}{sign}


%% \newcommand{\tightoverset}[2]{% for arrow vec
%%   \mathop{#2}\limits^{\vbox to -.5ex{\kern-0.75ex\hbox{$#1$}\vss}}}
\newcommand{\arrowvec}{\overrightarrow}
%\renewcommand{\vec}[1]{\arrowvec{\mathbf{#1}}}
\renewcommand{\vec}{\mathbf}
\newcommand{\veci}{{\boldsymbol{\hat{\imath}}}}
\newcommand{\vecj}{{\boldsymbol{\hat{\jmath}}}}
\newcommand{\veck}{{\boldsymbol{\hat{k}}}}
\newcommand{\vecl}{\boldsymbol{\l}}
\newcommand{\uvec}[1]{\mathbf{\hat{#1}}}
\newcommand{\utan}{\mathbf{\hat{t}}}
\newcommand{\unormal}{\mathbf{\hat{n}}}
\newcommand{\ubinormal}{\mathbf{\hat{b}}}

\newcommand{\dotp}{\bullet}
\newcommand{\cross}{\boldsymbol\times}
\newcommand{\grad}{\boldsymbol\nabla}
\newcommand{\divergence}{\grad\dotp}
\newcommand{\curl}{\grad\cross}
%\DeclareMathOperator{\divergence}{divergence}
%\DeclareMathOperator{\curl}[1]{\grad\cross #1}
\newcommand{\lto}{\mathop{\longrightarrow\,}\limits}

\renewcommand{\bar}{\overline}

\colorlet{textColor}{black} 
\colorlet{background}{white}
\colorlet{penColor}{blue!50!black} % Color of a curve in a plot
\colorlet{penColor2}{red!50!black}% Color of a curve in a plot
\colorlet{penColor3}{red!50!blue} % Color of a curve in a plot
\colorlet{penColor4}{green!50!black} % Color of a curve in a plot
\colorlet{penColor5}{orange!80!black} % Color of a curve in a plot
\colorlet{penColor6}{yellow!70!black} % Color of a curve in a plot
\colorlet{fill1}{penColor!20} % Color of fill in a plot
\colorlet{fill2}{penColor2!20} % Color of fill in a plot
\colorlet{fillp}{fill1} % Color of positive area
\colorlet{filln}{penColor2!20} % Color of negative area
\colorlet{fill3}{penColor3!20} % Fill
\colorlet{fill4}{penColor4!20} % Fill
\colorlet{fill5}{penColor5!20} % Fill
\colorlet{gridColor}{gray!50} % Color of grid in a plot

\newcommand{\surfaceColor}{violet}
\newcommand{\surfaceColorTwo}{redyellow}
\newcommand{\sliceColor}{greenyellow}




\pgfmathdeclarefunction{gauss}{2}{% gives gaussian
  \pgfmathparse{1/(#2*sqrt(2*pi))*exp(-((x-#1)^2)/(2*#2^2))}%
}


%%%%%%%%%%%%%
%% Vectors
%%%%%%%%%%%%%

%% Simple horiz vectors
\renewcommand{\vector}[1]{\left\langle #1\right\rangle}


%% %% Complex Horiz Vectors with angle brackets
%% \makeatletter
%% \renewcommand{\vector}[2][ , ]{\left\langle%
%%   \def\nextitem{\def\nextitem{#1}}%
%%   \@for \el:=#2\do{\nextitem\el}\right\rangle%
%% }
%% \makeatother

%% %% Vertical Vectors
%% \def\vector#1{\begin{bmatrix}\vecListA#1,,\end{bmatrix}}
%% \def\vecListA#1,{\if,#1,\else #1\cr \expandafter \vecListA \fi}

%%%%%%%%%%%%%
%% End of vectors
%%%%%%%%%%%%%

%\newcommand{\fullwidth}{}
%\newcommand{\normalwidth}{}



%% makes a snazzy t-chart for evaluating functions
%\newenvironment{tchart}{\rowcolors{2}{}{background!90!textColor}\array}{\endarray}

%%This is to help with formatting on future title pages.
\newenvironment{sectionOutcomes}{}{} 



%% Flowchart stuff
%\tikzstyle{startstop} = [rectangle, rounded corners, minimum width=3cm, minimum height=1cm,text centered, draw=black]
%\tikzstyle{question} = [rectangle, minimum width=3cm, minimum height=1cm, text centered, draw=black]
%\tikzstyle{decision} = [trapezium, trapezium left angle=70, trapezium right angle=110, minimum width=3cm, minimum height=1cm, text centered, draw=black]
%\tikzstyle{question} = [rectangle, rounded corners, minimum width=3cm, minimum height=1cm,text centered, draw=black]
%\tikzstyle{process} = [rectangle, minimum width=3cm, minimum height=1cm, text centered, draw=black]
%\tikzstyle{decision} = [trapezium, trapezium left angle=70, trapezium right angle=110, minimum width=3cm, minimum height=1cm, text centered, draw=black]


\outcome{To be able to use the method of substitution to solve some ``simple'' integrals, with an emphasis on being able to correctly identify what to substitute for.}
\outcome{Undo the Chain Rule.}
\outcome{Calculate indefinite integrals (antiderivatives) using basic substitution.}
\outcome{Calculate definite integrals using basic substitution.}
\title{8.10 - The Substitution Rule}
\begin{document}
\begin{abstract}
  We learn a new technique, called substitution, to help us solve
  problems involving integration.
\end{abstract}
\maketitle


Computing antiderivatives is not as easy as computing derivatives. For instance, we've seen that 
\[\int f(x)g(x)\d x \neq \int f(x)dx\int g(x)\d x.
\]
Additionally, the chain rule can be difficult to ``undo.''  We
have a general method called ``integration by substitution'' that will
somewhat help with this difficulty. 

If the functions are differentiable, we can apply the chain rule and obtain
\[
\ddx f(g(x)) = f'(g(x))g'(x)
\]
If the derivatives are continuous, we can use this equality to evaluate a definite integral. Namely,

 \[ \int f'(g(x))g'(x) \d x = {f(g(x))}+C \]

Let's start with a simple example that fits the above form.
\begin{example}
Compute:
\[
\int 2x e^{x^2}\d x
\]
\begin{explanation}
Notice that $\frac{d}{dx}\left[e^{x^2}\right]=e^{x^2}\frac{d}{dx}\left[x^2\right]=2xe^{x^2}$.

Thus, we know that 
\[
\int 2x e^{x^2}\d x=e^{x^2}+C.
\]
However, there are many integrals that we encounter that are not so obviously in that form. So how do we deal with those scenarios?

Notice this integral has the structure
\[
\int f'(g(x)) g'(x) \d x ,  
\]


where $g(x) =x^2$. Thus $g'(x)dx =2xdx$, and note that

\[
\int 2x e^{x^2}\d x
=\int e^{\underbrace{x^2}}\underbrace{2x\d x}
\]
So, let's introduce a new variable $u$. We will let $u=g(x)=x^2$. We now want to substitute anything that has an $x$ in it with it's corresponding expression in terms of $u$.
If 
\begin{align*}
    u&=x^2\\
    du&=2xdx
\end{align*}

Notice that the second line can be seen as using differentials.


\[
\int 2x e^{x^2}\d x=\int e^{g(x)}g'(x)\d x=\int e^{u}\d u
\]

We can easily evaluate the last integral
\[
\int e^{u}\d u= e^u+C
\]

We now need to substitute the $u$ in our answer with it's equivalent expression in terms of $x$. 
Thus,

\[
\int 2x e^{x^2}\d x=\int e^{u}\d u=e^u+C=e^{x^2}+C.
\]

Therefore, whenever we are faced with a problem of evaluating a difficult integral, we try to  replace it with  a simpler one.


\end{explanation}
\end{example}


\section{More examples}

With some experience, it is (usually) not too hard to see what to
substitute as $u$.  When determining what your $u$ should be, it is often helpful to look for an expression whose derivative (which can differ by a constant) is multiplied into the function. We will work through the following examples in the
same way that we did above.
\begin{example}
Compute:
\[
\int x^4(x^5+1)^{9} \d x
\]
\begin{explanation}
Here we set $u =  \answer[given]{x^5+1}$, so $\d u =  \answer[given]{5x^4} \d x$.  Then
we can solve for the $x^4$ that is present in the integrand. It follows that $\frac{1}{5}\d u=x^4\d x$. Therefore
\begin{align*}
  \int x^4(x^5+1)^{9} \d x &= \int \frac{1}{5} (u)^{9} \d u \\
  &= \frac{1}{5} \int u^{9} \d u\\
&=\frac{u^{10}}{ \answer[given]{50}}+C.
\end{align*}
We again need to back-substitute into our answer, so that our final
answer is a function of $x$.  Recalling that $u= x^5+1$, we have
our final answer
\[
\int x^4(x^5+1)^{9} \d x= \frac{(x^5+1)^{10}}{\answer[given]{50}}+C.
\]
Reminder: you can always verify your result by differentiating.

\end{explanation}
\end{example}


%If substitution works to solve an integral (and that is not always the
%case!), a common trick to find what to substitute for is to locate the
%``ugly'' part of the function being integrated.  We then substitute
%for the ``inside'' of this ugly part.  While this technique is
%certainly not rigorous, it can prove to be very helpful.  This is
%especially true for students new to the technique of substitution.
%The next two problems are really good examples of this philosophy.


\begin{example}
  Compute:
  \[
  \int \frac{\cos(\ln x)}{x} \d x
  \]
\begin{explanation}
Here the ``ugly'' part here is $\cos(\ln x)$.  So we substitute for
the inside:
\[
u=g(x)=\answer[given]{\ln x}.
\]
Then
\[
\d u =  \answer[given]{\frac{1}{x}} \d x 	
\]

Then we substitute into the original integral and solve:
\begin{align*}
\int\frac{\cos(\ln x)}{x} \d x &= \int\frac{\cos(u)}{x} x \d u  \\
&= \int \cos(u) \d u  \\
&= {\answer[given]{\sin(u)}}+C\\
&= \answer[given]{\sin}(\answer[given]{\ln(x)})+C
\end{align*}
\end{explanation}
\end{example}
Let's look at another example that will require a new technique.
\begin{example}
  Compute:
\[
\int x^2\sqrt{1-x}\d x
\]
\begin{explanation}

Here it is not apparent that the chain rule is involved. However, if
it was involved, perhaps a good guess for $g$ would be
\[
u = \answer[given]{1-x}
\]
and then
\begin{align*}
  \d u &= \answer[given]{-1} \d x, \\
  -\d u &= \d x.
\end{align*}
Now we consider the integral we are trying to compute
\[
\int x^2\sqrt{1-x}\d x
\]
and we substitute using our work above. Write with me
\begin{align*}
  \int x^2\sqrt{1-x}\d x &= \int x^2 \answer[given]{\sqrt{u}} ( \answer[given]{-1} ) \d u \\
  &= \int {-x^2 \answer[given]{\sqrt{u}}}\d u.
\end{align*}
At this point, it seems like this isn't going to work since we do not know how to substitute $x^2$ with an expression in terms of $u$.

Let's dig into this a little bit. We know $u=1-x$. With a little rearranging perhaps we can find something helpful!
\begin{align*}
u &= 1-x \\
 u -1 &= -x\\
 \answer[given]{1- u} &= x
\end{align*}
so now we can replace the $x^2$ as well. Notice 
\begin{align*}
    x^2&=(1-u)^2\\
    &=(1-2u+u^2)
\end{align*}
Picking up where we left off, 
\[ \int x^2\sqrt{1-x}\d x = \int -\answer[given]{(1-2u+u^2)} \sqrt{u}\d u.
\]
At this point, we are close to being done. Write
\begin{align*}
\int -{(1-2u+u^2)} \sqrt{u}\d u &= \int -u^{1/2}{(1-2u+u^2)} \d u\\
&=\int -u^{1/2}+2u^{3/2}-u^{5/2}\d u\\
&=-\answer[given]{\frac{2}{3}}u^{3/2}+\frac{4}{5}u^{\answer[given]{5/2}}-\frac{2}{7}u^{7/2}+C
\end{align*}
Now recall that $u = 1-x$. Hence our final answer is
\[
\int x^3\sqrt{1-x}\d x = -\answer[given]{\frac{2}{3}}(1-x)^{3/2}+\frac{4}{5}(1-x)^{\answer[given]{5/2}}-\frac{2}{7}(1-x)^{7/2}+C
\]
\end{explanation}
\end{example}



\begin{example}\label{key example}
Compute:
\[
\int \frac{u}{1-u^2} \d u
\]
\begin{explanation}
We want to substitute for $1-u^2$.  
But the variable ``$u$'' has already been used. We can substitute with whatever variable that we want.  
In particular, let's use ``$w$'' for this problem.  
So we let
\[
w = 1 - u^2
\]
and then
\begin{align*}
  \d w &= \answer[given]{-2u} \d u,\\
  \answer[given]{-\frac{1}{2}}\d w &=u  \d u .
\end{align*}
Thus
\begin{align*}
\int \frac{u}{1-u^2} \d u &= \int \left(-\frac{1}{2}\right)\left(\frac{1}{w}\right)\d w  \\
&= - \frac{1}{2} \int \frac{1}{w} \d w  \\
&= - \frac{1}{2} \ln|w| + C  \\
&= - \frac{1}{2} \ln|\answer[given]{1-u^2}| + C.
\end{align*}
\end{explanation}


\end{example}

There are many functions that may not look like prime candidates for integrating using substitution, but with a little rewriting it becomes clear that they are! One such example is $y=\tan (x)$

\begin{example}\label{example tan}
Compute:
\[
\int \tan(x) \d x
\]
\begin{explanation}
We begin by recalling that 

$\tan(x) = \frac{\sin(x)}{\cos(x)}  $
We then make the substitution
\[
u = \cos(x)
\]
and so
\begin{align*}
\d u &= \answer[given]{-\sin(x)} \d x,\\
-\d u &= {\answer[given]{\cos(x)}} \d x.
\end{align*}
Then
\begin{align*}
\int \tan(x) \d x &= \int \frac{\sin(x)}{\cos(x)} \d x  \\
&= \int -\frac{1}{u}  \d u  \\
&= \answer[given]{-\ln|u|}+C
\end{align*}
Substituting with $u=\cos(x)$, we get
\[\int \tan(x) \d x=-\ln|\answer[given]{\cos(x)}|+C=\ln|\sec(x)|+C\]
Notice that last equality was achieved using the log rule
\[n\log(x)=\log(x^n)\]
\end{explanation}
\end{example}

We have just proved

\begin{theorem}
\[
\int \tan(x) \d x = \ln|\sec(x)| + C.
\]
\end{theorem}


To summarize, if we suspect that a given function is the derivative of
another via the chain rule, we introduce a new variable $u=g(x)$, where $g$ is a likely candidate for
the inner function. We rewrite the integral
 entirely in terms of $u$, with no $x$ remaining in the
expression. If we can integrate this new function of $u$, then the
antiderivative of the original function is obtained by replacing $u$
by $g(x)$.
\end{document}