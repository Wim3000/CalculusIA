\documentclass{ximera}

%\usepackage{todonotes}

\newcommand{\todo}{}

\usepackage{esint} % for \oiint
\graphicspath{
  {./}
  {ximeraTutorial/}
  {basicPhilosophy/}
  {functionsOfSeveralVariables/}
  {normalVectors/}
  {lagrangeMultipliers/}
  {vectorFields/}
  {greensTheorem/}
  {shapeOfThingsToCome/}
}
\usepackage{comment} %% used in what is a limit
\usepackage[valunder]{signchart} %% used in graphing sign chart

\newcommand{\mooculus}{\textsf{\textbf{MOOC}\textnormal{\textsf{ULUS}}}}

\usepackage{tkz-euclide}
\tikzset{>=stealth} %% cool arrow head
\tikzset{shorten <>/.style={ shorten >=#1, shorten <=#1 } } %% allows shorter vectors


\pgfplotsset{soldot/.style={color=black,only marks,mark=*}} %% USED by piecewise functions
\pgfplotsset{holdot/.style={color=black,fill=white,only marks,mark=*}}
\usetikzlibrary{arrows.meta}


\usepackage{tkz-tab}  %% sign charts
\usepackage{polynom}

\usetikzlibrary{backgrounds} %% for boxes around graphs
\usetikzlibrary{shapes,positioning}  %% Clouds and stars
\usetikzlibrary{matrix} %% for matrix
\usepgfplotslibrary{polar} %% for polar plots
%\usetkzobj{all}
\usepackage[makeroom]{cancel} %% for strike outs
%\usepackage{mathtools} %% for pretty underbrace % Breaks Ximera
\usepackage{multicol}
\usepackage{pgffor} %% required for integral for loops


%% http://tex.stackexchange.com/questions/66490/drawing-a-tikz-arc-specifying-the-center
%% Draws beach ball
\tikzset{pics/carc/.style args={#1:#2:#3}{code={\draw[pic actions] (#1:#3) arc(#1:#2:#3);}}}



\usepackage{array}
\setlength{\extrarowheight}{+.1cm}   
\newdimen\digitwidth
\settowidth\digitwidth{9}
\def\divrule#1#2{
\noalign{\moveright#1\digitwidth
\vbox{\hrule width#2\digitwidth}}}




\newcommand{\RR}{\mathbb R}
\newcommand{\R}{\mathbb R}
\newcommand{\N}{\mathbb N}
\newcommand{\Z}{\mathbb Z}

\newcommand{\sagemath}{\textsf{SageMath}}


%\renewcommand{\d}{\,d\!}
\renewcommand{\d}{\mathop{}\!d}
\newcommand{\dd}[2][]{\frac{\d #1}{\d #2}}
\newcommand{\pp}[2][]{\frac{\partial #1}{\partial #2}}
\renewcommand{\l}{\ell}
\newcommand{\ddx}{\frac{d}{\d x}}
\newcommand{\ddt}{\frac{d}{\d t}}


\newcommand{\zeroOverZero}{\ensuremath{\boldsymbol{\tfrac{0}{0}}}}
\newcommand{\inftyOverInfty}{\ensuremath{\boldsymbol{\tfrac{\infty}{\infty}}}}
\newcommand{\zeroOverInfty}{\ensuremath{\boldsymbol{\tfrac{0}{\infty}}}}
\newcommand{\zeroTimesInfty}{\ensuremath{\small\boldsymbol{0\cdot \infty}}}
\newcommand{\inftyMinusInfty}{\ensuremath{\small\boldsymbol{\infty - \infty}}}
\newcommand{\oneToInfty}{\ensuremath{\boldsymbol{1^\infty}}}
\newcommand{\zeroToZero}{\ensuremath{\boldsymbol{0^0}}}
\newcommand{\inftyToZero}{\ensuremath{\boldsymbol{\infty^0}}}



\newcommand{\numOverZero}{\ensuremath{\boldsymbol{\tfrac{\#}{0}}}}
\newcommand{\dfn}{\textbf}
%\newcommand{\unit}{\,\mathrm}
\newcommand{\unit}{\mathop{}\!\mathrm}
\newcommand{\eval}[1]{\bigg[ #1 \bigg]}
\newcommand{\seq}[1]{\left( #1 \right)}
\renewcommand{\epsilon}{\varepsilon}
\renewcommand{\phi}{\varphi}


\renewcommand{\iff}{\Leftrightarrow}

\DeclareMathOperator{\arccot}{arccot}
\DeclareMathOperator{\arcsec}{arcsec}
\DeclareMathOperator{\arccsc}{arccsc}
\DeclareMathOperator{\si}{Si}
\DeclareMathOperator{\proj}{\vec{proj}}
\DeclareMathOperator{\scal}{scal}
\DeclareMathOperator{\sign}{sign}


%% \newcommand{\tightoverset}[2]{% for arrow vec
%%   \mathop{#2}\limits^{\vbox to -.5ex{\kern-0.75ex\hbox{$#1$}\vss}}}
\newcommand{\arrowvec}{\overrightarrow}
%\renewcommand{\vec}[1]{\arrowvec{\mathbf{#1}}}
\renewcommand{\vec}{\mathbf}
\newcommand{\veci}{{\boldsymbol{\hat{\imath}}}}
\newcommand{\vecj}{{\boldsymbol{\hat{\jmath}}}}
\newcommand{\veck}{{\boldsymbol{\hat{k}}}}
\newcommand{\vecl}{\boldsymbol{\l}}
\newcommand{\uvec}[1]{\mathbf{\hat{#1}}}
\newcommand{\utan}{\mathbf{\hat{t}}}
\newcommand{\unormal}{\mathbf{\hat{n}}}
\newcommand{\ubinormal}{\mathbf{\hat{b}}}

\newcommand{\dotp}{\bullet}
\newcommand{\cross}{\boldsymbol\times}
\newcommand{\grad}{\boldsymbol\nabla}
\newcommand{\divergence}{\grad\dotp}
\newcommand{\curl}{\grad\cross}
%\DeclareMathOperator{\divergence}{divergence}
%\DeclareMathOperator{\curl}[1]{\grad\cross #1}
\newcommand{\lto}{\mathop{\longrightarrow\,}\limits}

\renewcommand{\bar}{\overline}

\colorlet{textColor}{black} 
\colorlet{background}{white}
\colorlet{penColor}{blue!50!black} % Color of a curve in a plot
\colorlet{penColor2}{red!50!black}% Color of a curve in a plot
\colorlet{penColor3}{red!50!blue} % Color of a curve in a plot
\colorlet{penColor4}{green!50!black} % Color of a curve in a plot
\colorlet{penColor5}{orange!80!black} % Color of a curve in a plot
\colorlet{penColor6}{yellow!70!black} % Color of a curve in a plot
\colorlet{fill1}{penColor!20} % Color of fill in a plot
\colorlet{fill2}{penColor2!20} % Color of fill in a plot
\colorlet{fillp}{fill1} % Color of positive area
\colorlet{filln}{penColor2!20} % Color of negative area
\colorlet{fill3}{penColor3!20} % Fill
\colorlet{fill4}{penColor4!20} % Fill
\colorlet{fill5}{penColor5!20} % Fill
\colorlet{gridColor}{gray!50} % Color of grid in a plot

\newcommand{\surfaceColor}{violet}
\newcommand{\surfaceColorTwo}{redyellow}
\newcommand{\sliceColor}{greenyellow}




\pgfmathdeclarefunction{gauss}{2}{% gives gaussian
  \pgfmathparse{1/(#2*sqrt(2*pi))*exp(-((x-#1)^2)/(2*#2^2))}%
}


%%%%%%%%%%%%%
%% Vectors
%%%%%%%%%%%%%

%% Simple horiz vectors
\renewcommand{\vector}[1]{\left\langle #1\right\rangle}


%% %% Complex Horiz Vectors with angle brackets
%% \makeatletter
%% \renewcommand{\vector}[2][ , ]{\left\langle%
%%   \def\nextitem{\def\nextitem{#1}}%
%%   \@for \el:=#2\do{\nextitem\el}\right\rangle%
%% }
%% \makeatother

%% %% Vertical Vectors
%% \def\vector#1{\begin{bmatrix}\vecListA#1,,\end{bmatrix}}
%% \def\vecListA#1,{\if,#1,\else #1\cr \expandafter \vecListA \fi}

%%%%%%%%%%%%%
%% End of vectors
%%%%%%%%%%%%%

%\newcommand{\fullwidth}{}
%\newcommand{\normalwidth}{}



%% makes a snazzy t-chart for evaluating functions
%\newenvironment{tchart}{\rowcolors{2}{}{background!90!textColor}\array}{\endarray}

%%This is to help with formatting on future title pages.
\newenvironment{sectionOutcomes}{}{} 



%% Flowchart stuff
%\tikzstyle{startstop} = [rectangle, rounded corners, minimum width=3cm, minimum height=1cm,text centered, draw=black]
%\tikzstyle{question} = [rectangle, minimum width=3cm, minimum height=1cm, text centered, draw=black]
%\tikzstyle{decision} = [trapezium, trapezium left angle=70, trapezium right angle=110, minimum width=3cm, minimum height=1cm, text centered, draw=black]
%\tikzstyle{question} = [rectangle, rounded corners, minimum width=3cm, minimum height=1cm,text centered, draw=black]
%\tikzstyle{process} = [rectangle, minimum width=3cm, minimum height=1cm, text centered, draw=black]
%\tikzstyle{decision} = [trapezium, trapezium left angle=70, trapezium right angle=110, minimum width=3cm, minimum height=1cm, text centered, draw=black]


\title{Polynomials}
\begin{document}

\begin{abstract} \end{abstract}
\maketitle


The functions you are most familiar with are probably polynomial
functions.

\section{What are polynomial functions?}

\begin{definition}
  A \dfn{polynomial function} in the variable $x$ is a function
  which can be written in the form
  \[
  f(x) = a_nx^n + a_{n-1}x^{n-1} + \dots + a_1 x + a_0
  \]
  where the $a_i$'s are all constants (called the \dfn{coefficients})
  and $n$ is a nonnegative integer (called the \dfn{degree}). The notation for the degree of $f$ is written deg$(f)$. The domain of a polynomial function is $(-\infty,\infty)$.
\end{definition}

\begin{question}
	Which of the following are polynomial functions?
	\begin{selectAll}
		\choice[correct]{$f(x) = 0$}
		\choice[correct]{$f(x) = -9$}
		\choice[correct]{$f(x) = 3x+1$}
		\choice{$f(x) = x^{1/2}-x +8$}
		\choice{$f(x) = -4x^{-3}+5x^{-1}+7-18x^2$}
		\choice[correct]{$f(x) = (x+1)(x-1)+e^x - e^x $}
		\choice{$f(x) = \frac{x^2 - 3x + 2}{x-2}$}
		\choice[correct]{$f(x) = x^7-32x^6-\pi x^3+45/84$}
	\end{selectAll}
  \begin{feedback}\hfil
    \begin{itemize}
    \item $a_0$ in the definition is allowed to be any number, including $0$, so this is a polynomial. 
    \item $a_0$ in the definition is allowed to be any number, including $-9$, so this is a polynomial. 
    \item This is a degree one polynoimal.
    \item Because $1/2$ is not a whole number, this is not a polynomial.
    \item Because $-1$ is a whole number, this is a polynomial. (Notice it is still a polynomial even if the powers are not written in order.)
    \item This is a polynomial, but to know whether it is, you have to simplify first. $(x+1)(x-1)+e^x-e^x=(x+1)(x-1)=x^2-1$.
    \item This is not a polynomial. Even though you could simplify it as $\displaystyle \frac{(x-2)(x-1)}{x-2} = x-1,$ this simplification is not valid when $x=2$. Since the domain of this function excludes $2$, this cannot be a polynomial (in our definition, we said the domain of a polynomial function is $(-\infty,\infty)$.
\item This is a polynomial.
    \end{itemize}
  \end{feedback}
\end{question}


\section{Multiplying}

Multiplying polynomials is based on the familiar property of arithmetic, \dfn{distribution}: $\displaystyle a\cdot( b + c ) = ab + ac$.
\begin{example}
	Multiply $2x^3$ by $x^5 - 4x^4 + 7x - 2$.
	\begin{explanation}
		Use distribution. Recall that when you multiply the same base, you add the exponents, for example, $2^3\cdot2^2=(2\cdot2\cdot2)\cdot(2\cdot2) = 2\cdot2\cdot2\cdot2\cdot2 = 2^5$.
		\begin{align*}
			2x^3 \left( x^5 - 4x^4 + 7x - 2 \right) &= 2x^3 \cdot x^5 - 2x^3 \cdot 4x^4 + 2x^3 \cdot 7x \\
				& - 2x^3 \cdot 2\\
				&= 2 x^8 - 8x^7 + 14x^4 - 4x^3
		\end{align*}
	\end{explanation}
\end{example}


\begin{example}
	Multiply $3x^2 + 2x - 1$ by $2x^4 + 5x^3+ x + 1$.
	\begin{explanation}
		We'll start by distributing the first polynomial into the second.  Then we'll distribute back, and finish by combining like terms.
		\begin{align*}
			(3x^2+ 2x-1)(2x^4 + 5x^3 + x +1) &= 2x^4(3x^2+ 2x-1) \\
			 &+ 5x^3 (3x^2+ 2x-1) \\
			 &+ x (3x^2+ 2x-1) \\
			 &+ 1(3x^2+ 2x-1)\\
				&= (6x^6 + 4x^5 - 2x^4) \\
				&+ (15x^5 + 10x^4 - 5x^3) \\
				&+ (3x^3+2x^2-x) \\
				&+(3x^2+2x-1)
		\end{align*}
		After combining like-terms, we find \[6x^6 + 19x^5 +8x^4 - 2x^3 + 5x^2+ x - 1.\]
	\end{explanation}
\end{example}
The result is that we have multiplied every term of the first polynomial by every term of the second, then added the results together.  In the case of two
\dfn{binomials} (polynomials with only two terms), this is frequently referred to as \dfn{FOIL}. Example: $(a+b)^2 = (a+b)(a+b)=a(a+b)+b(a+b)=a^2+ab+ba+b^2=a^2+2ab+b^2$.

There are several product formulas that arise repeatedly when working with binomials.  You will likely have seen most of these before.
%\begin{xbox}{Multiplication Formulas}
		\begin{align*}
			(a + b)^2 &= a^2 + 2ab + b^2\\
			(a-b)^2 &= a^2 - 2ab + b^2\\
			(a+b)(a-b) &= a^2 - b^2\\
			(a+b)^3 &= a^3 + 3 a^2 b + 3 ab^2 + b^3\\
			(a-b)^3 &= a^3 - 3 a^2 b + 3 ab^2 - b^3\\
			(a-b)(a^2 + ab + b^2) &= a^3 - b^3\\
			(a+b)(a^2-ab+b^2) &= a^3 + b^3
		\end{align*}
%\end{xbox}


\begin{example}
	If $f$ is the polynomial function given by $f(x) = x^3-2x^2 +5$, find $f(x+1)$.  Find $f(2+h)$.
	\begin{explanation}
		Note that there are two ways to think of this,both leading to the same strategy.

		First, remember the notation $f(1)$ means that 1 is the input, and the notation has not changed here. $f(x+1)$ means $x+1$ is the input, so replace all instances of $x$ with $x+1$ instead.
		
		Another way to think of this is as a function composition. Consider $x^3-2x^2+5$ to be the outside function and $x+1$ as the inside function.
		\begin{align*}
			f(x+1) &= (x+1)^3 - 2(x+1)^2 + 5 \\
				&= (x^3 + 3x^2 + 3x + 1) - 2(x^2 + 2x + 1) + 5\\
				&= x^3 + x^2 -x + 4
		\end{align*}
		In the same way, $f(2+h)$ asks us to replace $x$ by $2+h$.
		\begin{align*}
			f(2+h) &= (2+h)^3 - 2(2+h)^2 + 5\\
				&= (8+12h + 6h^2 + h^3) - 2(4+4h+h^2) + 5\\
				&= h^3 + 4h^2+4h+5
		\end{align*}
	\end{explanation}
\end{example}

\begin{problem}
	If $f(x)=4x^2-3x+1$, find $f(x+h)-f(x)$.
	\begin{multipleChoice}
		\choice{$h$}
		\choice{$4h^2-3h+1$}
		\choice[correct]{$8xh + 4h^2 - 3h$}
		\choice{$4x^2+8xh+4h^2-3x-3h+1$}
	\end{multipleChoice}
\end{problem}

\section{Factoring}
Factoring is a bit like an inverse operation of multiplying polynomials.  We start with the multiplied out polynomial, and ask what the individual factors were.

The easiest factors to deal with are common factors.
\begin{example}
	Factor $8x^4 + 40x^3$.
	\begin{explanation}
		The coefficients of this polynomial have a greatest common divisor of $8$, and the least degree of the terms is $3$, so
		$8x^3$ is a common factor.  Taking this out leaves:
		$\displaystyle 8x^3( x+5)$.
	\end{explanation}
\end{example}


For \dfn{trinomials} (polynomials with three terms) of the form $ax^2 + bx + c$, we try to factor as $(px+r)(qx+s)$.  We'll start with an example with $a=1$.
\begin{example}
	Factor $x^2+2x-24$.
	\begin{explanation}
		Since the leading coefficient is $1$, we want to factor $x^2+2x-24$ as $(x+r)(x+s)$.  If we multiply out $(x+r)(x+s)$ we get
		$x^2 + (r+s)x + rs$.  We need to find two numbers that add to $2$,
		and multiply to $-24$.  Look at the different ways of multiplying two numbers to get $-24$:  $-1 \cdot 24$, $-2 \cdot 12$, $-3 \cdot 8$, $-4 \cdot 6$...
		That last one, $-4$ and $6$ are two numbers that add to $2$ and multiply to give $-24$.
		
		Our factorization is $x^2+2x-24 = (x-4)(x+6)$.
	\end{explanation}
\end{example}

%The process is slightly more complicated when $a \neq 1$.
%\begin{example}
%	Factor $6x^2 + 11 x + 3$.
%	\begin{explanation}
%		In this case, want to find $(px+r)(qx+s) = pq x^2 + (ps+qr) x+ rs$.  Ignore that middle term for now, and just focus on the leading term and the constant term.
%		We need to find numbers that multiply to $6$ for our coefficients, and that multiply to $3$ for the constants in each factor.
%		That means we're looking at things like $(6x+1)(x+3)$, $(6x+3)(x+1)$, $(2x+3)(3x+1)$, $(2x+1)(3x+3)$.   Actually, that's it, as $6$ can only
%		factor as $1 \cdot 6$ or $2 \cdot 3$, and $3$ only factors as $1 \cdot 3$.
%		
%		All of these possibilities will give the right $x^2$ term and the right constant term.  The only difference is the $x$-term they give.  Do any of them
%		give an $x$-coefficient of $11$?  Yes!
%		
%		$6x^2+11x+3 = (2x+3)(3x+1)$.
%	\end{explanation}
%\end{example}

Remember, a \dfn{root} of a polynomial function is an $x$-value where the polynomial is zero. 
There is a close relationship between roots of a polynomial and factors of that polynomial. 
\begin{theorem}
If $x=c$ is a root of a polynomial, then $x-c$ is a factor of that polynomial.
\end{theorem}
In our previous example, notice that plugging 4 into $x^2+2x-24$ gives 0. This theorem claims that $x-4$ must be a factor, and that was true.

%\begin{example}
%	Factor $3x^3-13x^2+2x+8$.
%	\begin{explanation}
%		Notice that if we substitute $x=1$ into the polynomial, we get zero out.  That means $x-1$ is a factor of $3x^3-12x^2+2x+8$.
%		Polynomial long division gives:
%
%		\begin{center}
%\newdimen\digitwidth
%	\settowidth\digitwidth{0}
%\def~{\hspace{\digitwidth}}
%\def\divrule#1#2{%
%	\noalign{\moveright#1\digitwidth%
%		\vbox{\hrule width#2\digitwidth}}}
%$x-1$\,\begin{tabular}[b]{@{}r@{}}
%$3x^2-10x-8$ \\ \hline
%\big)\begin{tabular}[t]{@{}l@{}}
%~$3x^3-13x^2+2x+8$ \\
%$-3x^3+3x^2$ \\ \divrule{0}{10}
%~~~~~$-10x^2+2x$ \\
%~~~~~~$10x^2-10x$ \\ \divrule{6}{10}
%~~~~~~~~~~~$-8x+8$ \\
%~~~~~~~~~~~~~$8x-8$ \\ \divrule{12}{8}
%~~~~~~~~~~~~~~~~~$0$
%\end{tabular}
%\end{tabular}
%		\end{center}
%		
%		The quotient was $3x^2-10x-8$, so $3x^3-13x^2+2x+8 = (x-1)(3x^2-10x-8)$.  It remains to factor the quotient.
%		
%		The leading coefficient $3$ factors as $1 \cdot 3$, and $8$ factors as $1 \cdot 8$ and $2 \cdot 4$.  Examining the possibilities, we find the factorization
%		of $3x^2-10x-8$ as $(3x+2)(x-4)$.  The entire polynomial factors as $(x-1)(3x+2)(x-4)$.
%	\end{explanation}
%\end{example}



\section{Equations}
A \emph{quadratic equation in $x$} is an equation which is equivalent to one with the form
	$ax^2 + bx + c = 0$, where $a$, $b$, and $c$ are constants, with $a \neq 0$.

There are three major techniques you are probably familiar with for solving quadratic equations:
\begin{itemize}
	\item Factoring.
	\item Completing the Square.
	\item Quadratic Formula.
\end{itemize}

Each of these methods are important.  Factoring is vital, because it is a valid approach to solve nearly any type of equation.  Completing
the Square is a technique that becomes useful when we need to rewrite certain types of expressions.  The Quadratic Formula will always
work, but has some limitations.

\begin{example}
	Solve the quadratic equation
	\[ 2x^2 +5x = 27 x - 60. \]
	\begin{explanation}
		We'll start by writing this in it's standard form: $2x^2 - 22x + 60 = 0$.  Notice how there is a common factor of 2? Let's divide by 2.
		
		$x^2 - 11x + 30 = 0$.  Any of the above methods will work here, so let's try factoring.  What numbers add to $-11$ and multiply to $30$?
		$-5$ and $-6$ do. That gives us:
		\begin{align*}
			x^2 - 11x + 30 &= 0\\
			(x-5)(x-6) &= 0
		\end{align*}
		Either $x-5 = 0$ (giving us $x=5$) or $x-6 = 0$ (giving us $x=6$).  The two solutions are $x = 5, 6$.
	\end{explanation}
\end{example}	

\begin{example}
	Solve the quadratic equation
	\[  x^2 = 6x -4.\]
	\begin{explanation}
		Again, we'll write it in standard form: $x^2 - 6x + 4 = 0$.  The quadratic $x^2 -6x+4$ does not factor, so we'll complete the square instead.
		
		Start by moving the constant term to the other side
		\[ x^2 - 6x = -4. \]
		This has $b = -6$.  That means $\displaystyle \dfrac{b}{2} = -3$ and $\displaystyle \left( \dfrac{b}{2}\right)^2 = 9$.  Add $9$ to both sides.
		\[ x^2 - 6x + 9 = -4 + 9 = 5. \]
		The left-hand side is now a perfect square, $\left(x-3\right)^2 = x^2 - 6x + 9$.  We can then solve using square roots.
		\begin{align*}
			x^2 - 6x + 9 &=  5\\
			\left( x-3 \right)^2 &= 5\\
			x-3 = \pm \sqrt{5} \\
			x = 3 \pm \sqrt{5}.
		\end{align*}
	\end{explanation}
\end{example}
If you had used the quadratic formula, $\displaystyle x = \dfrac{-b \pm \sqrt{b^2-4ac}}{2a}$, instead of factoring or completing the square above, you would have found the same solutions.

\begin{problem}
	Solve the quadratic equation $\displaystyle x^2 + 4 = 4\left(x+2\right)$.
	\begin{multipleChoice}
		\choice{$x = \pm 2$}
		\choice[correct]{$x=2\pm 2\sqrt{2}$}
		\choice{$x = -2, -4$}
		\choice{none of the above}
  \end{multipleChoice}
\end{problem}




%Building on these methods, we'll consider more general types of equations.
%A polynomial equation is an equation which is equivalent to one of the form $a_n x^n + a_{n-1} x^{n-1} + \dots + a_1 x + a_0 = 0$. 
%\begin{example}
%	Solve the equation $x^4 = 6x^2 - 4$.
%	\begin{explanation}
%		Start by rewriting as $x^4 - 6x^2 + 4=0$.  This is not a quadratic equation in $x$, but it IS a quadratic in $x^2$.
%		We continue by completing the square.
%		\begin{align*}
%			x^4 - 6x^2 &= 4 \\
%			x^4 - 6x^2 + 9 &= 4 + 9\\
%			(x^2 - 3)^2 &= 13\\
%			x^2 - 3 &= \pm \sqrt{13}\\
%			x^2 &= 3 \pm \sqrt{13}
%		\end{align*}
%		Either $x^2 = 3 + \sqrt{13}$, or $x^2 = 3 - \sqrt{13}$.  Since $3 - \sqrt{13}$ is negative, only $x^2 = 3 + \sqrt{13}$ is possible.
%		
%		The only real solutions are $x = \pm \sqrt{3 + \sqrt{13}}$.
%	\end{explanation}
%\end{example}
%
Factoring is a process that helps us solve more than just quadratic equations, as long as we first get one side of the equation equal to zero.
\begin{problem}
	Solve the equation
	\[ 2x^2 \left( x-4 \right) \left( 2x-5\right)^3 = 0. \] Enter your answers from smallest to largest value.
	\begin{prompt}
		$x=\answer{0}$, $x=\answer{\frac{5}{4}}$,$x=\answer{4}$.
	\end{prompt}

\end{problem}

%
%\begin{example}
%	Solve  $2x(2x^2 + 1) = x^2(x +13) - 8$.
%	\begin{explanation}
%		Start by rewriting the equation as $3x^3-13x^2+2x+8 = 0$.  This is not a quadratic-type equation, so we resort to factoring.
%		\begin{align*}
%			3x^3-13x^2+2x+8 &= 0\\
%			(x-1)(3x^2-10x-8) &= 0\\
%			(x-1)(3x+2)(x-4) &= 0
%		\end{align*}
%		Thus, either $x -1 = 0$ (so $x = 1$), or $3x+2=0$ (so $x = -2/3$), or $x-4 = 0$ (so $x=4$).
%		
%		The solutions are $x = 1, 4, -3/2$.
%	\end{explanation}
%\end{example}


The next theorem above is a deep fact of mathematics. The great mathematician Gauss proved the theorem in 1799. 
\begin{theorem}[The Fundamental Theorem of Algebra]\index{Fundamental Theorem of Algebra}
  Every polynomial of the form
  \[
  a_n x^n + a_{n-1} x^{n-1} + \dots + a_1 x + a_0
  \]
  where the $a_i$'s are real (or even complex!) numbers and $a_n \ne 0$ has exactly
  $n$ (possibly repeated) complex roots.
\end{theorem}

Recall that the \dfn{multiplicity} of a root indicates how many times that particular root is repeated.
The Fundamental Theorem of Algebra tells us that a polynomial equation of degree $n$, will have exactly $n$ complex solutions, once multiplicity
is taken into account.  As non-real solutions appear in complex-conjugate pairs  (if our equation has real coefficients), we will always have an even number of non-real
solutions.  That means, taking multiplicity into account, a quadratic equation will have either $2$ or $0$ real solutions.  A cubic equation will have either $3$ or $1$ real
solution.


\begin{problem}

	The equation $x^3 = 3x-2$ has $x=1$ as one solution.  Find another solution.
	
	\begin{prompt}
		\[ x= \answer{-2} \]
	\end{prompt}
	\begin{feedback}
  		Notice that there were only two real numbers that solve this equation.  How is that possible, if we said a cubic equation has to have either
		$3$ or $1$ real solutions?
  	\end{feedback}
\end{problem}

Learning Objectives:

The following are \textbf{facts} or \textbf{definitions} you must memorize from this section:
\begin{itemize}
\item A \dfn{polynomial function} in the variable $x$ is a function
  which can be written in the form
  \[
  f(x) = a_nx^n + a_{n-1}x^{n-1} + \dots + a_1 x + a_0
  \]
  where the $a_i$'s are all constants (called the \dfn{coefficients})
  and $n$ is a nonnegative integer (called the \dfn{degree}). The notation for the degree of $f$ is written deg$(f)$. The domain of a polynomial function is $(-\infty,\infty)$.
\item The distributive law: $\displaystyle a\cdot( b + c ) = ab + ac$
\item Multiplying exponents: $x^m\cdot x^n = x^{m+n}$
\item A \dfn{root} of a polynomial function is an $x$-value where the polynomial is zero. 
\item The Fundamental Theorem of Algebra: a degree $n$ polynomial has exactly $n$ (possibly complex) roots, up to multiplicity.
\end{itemize}

The following are \textbf{procedures} you must be able to perform from this section:
\begin{itemize}
\item FOIL: $(a+b)^2 = (a+b)(a+b)=a(a+b)+b(a+b)=a^2+ab+ba+b^2=a^2+2ab+b^2$
\item Finding a function value of an expression; such as $f(a+h)$ given a formula for $f(x)$.
\item Factoring common factors out of an expression
\item Factoring a trinomial
\item Factoring to solve an equation
\item Completing the square: first factor constants if necessary to get your polynomial in the form $x^2+bx+c=0$, move $c$ to the other side and add $\displaystyle \left(\frac{b}{2}\right)^2$ to both sides. The left hand side will factor as a perfect square.
\item The quadratic formula: the solutions to $ax^2+bx+c=0$ are 

$x=\displaystyle \frac{-b\pm \sqrt{b^2-4ac}}{2a}$.
\end{itemize}


\end{document}