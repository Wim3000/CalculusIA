

\documentclass{ximera}

%\usepackage{todonotes}

\newcommand{\todo}{}

\usepackage{esint} % for \oiint
\graphicspath{
  {./}
  {ximeraTutorial/}
  {basicPhilosophy/}
  {functionsOfSeveralVariables/}
  {normalVectors/}
  {lagrangeMultipliers/}
  {vectorFields/}
  {greensTheorem/}
  {shapeOfThingsToCome/}
}
\usepackage{comment} %% used in what is a limit
\usepackage[valunder]{signchart} %% used in graphing sign chart

\newcommand{\mooculus}{\textsf{\textbf{MOOC}\textnormal{\textsf{ULUS}}}}

\usepackage{tkz-euclide}
\tikzset{>=stealth} %% cool arrow head
\tikzset{shorten <>/.style={ shorten >=#1, shorten <=#1 } } %% allows shorter vectors


\pgfplotsset{soldot/.style={color=black,only marks,mark=*}} %% USED by piecewise functions
\pgfplotsset{holdot/.style={color=black,fill=white,only marks,mark=*}}
\usetikzlibrary{arrows.meta}


\usepackage{tkz-tab}  %% sign charts
\usepackage{polynom}

\usetikzlibrary{backgrounds} %% for boxes around graphs
\usetikzlibrary{shapes,positioning}  %% Clouds and stars
\usetikzlibrary{matrix} %% for matrix
\usepgfplotslibrary{polar} %% for polar plots
%\usetkzobj{all}
\usepackage[makeroom]{cancel} %% for strike outs
%\usepackage{mathtools} %% for pretty underbrace % Breaks Ximera
\usepackage{multicol}
\usepackage{pgffor} %% required for integral for loops


%% http://tex.stackexchange.com/questions/66490/drawing-a-tikz-arc-specifying-the-center
%% Draws beach ball
\tikzset{pics/carc/.style args={#1:#2:#3}{code={\draw[pic actions] (#1:#3) arc(#1:#2:#3);}}}



\usepackage{array}
\setlength{\extrarowheight}{+.1cm}   
\newdimen\digitwidth
\settowidth\digitwidth{9}
\def\divrule#1#2{
\noalign{\moveright#1\digitwidth
\vbox{\hrule width#2\digitwidth}}}




\newcommand{\RR}{\mathbb R}
\newcommand{\R}{\mathbb R}
\newcommand{\N}{\mathbb N}
\newcommand{\Z}{\mathbb Z}

\newcommand{\sagemath}{\textsf{SageMath}}


%\renewcommand{\d}{\,d\!}
\renewcommand{\d}{\mathop{}\!d}
\newcommand{\dd}[2][]{\frac{\d #1}{\d #2}}
\newcommand{\pp}[2][]{\frac{\partial #1}{\partial #2}}
\renewcommand{\l}{\ell}
\newcommand{\ddx}{\frac{d}{\d x}}
\newcommand{\ddt}{\frac{d}{\d t}}


\newcommand{\zeroOverZero}{\ensuremath{\boldsymbol{\tfrac{0}{0}}}}
\newcommand{\inftyOverInfty}{\ensuremath{\boldsymbol{\tfrac{\infty}{\infty}}}}
\newcommand{\zeroOverInfty}{\ensuremath{\boldsymbol{\tfrac{0}{\infty}}}}
\newcommand{\zeroTimesInfty}{\ensuremath{\small\boldsymbol{0\cdot \infty}}}
\newcommand{\inftyMinusInfty}{\ensuremath{\small\boldsymbol{\infty - \infty}}}
\newcommand{\oneToInfty}{\ensuremath{\boldsymbol{1^\infty}}}
\newcommand{\zeroToZero}{\ensuremath{\boldsymbol{0^0}}}
\newcommand{\inftyToZero}{\ensuremath{\boldsymbol{\infty^0}}}



\newcommand{\numOverZero}{\ensuremath{\boldsymbol{\tfrac{\#}{0}}}}
\newcommand{\dfn}{\textbf}
%\newcommand{\unit}{\,\mathrm}
\newcommand{\unit}{\mathop{}\!\mathrm}
\newcommand{\eval}[1]{\bigg[ #1 \bigg]}
\newcommand{\seq}[1]{\left( #1 \right)}
\renewcommand{\epsilon}{\varepsilon}
\renewcommand{\phi}{\varphi}


\renewcommand{\iff}{\Leftrightarrow}

\DeclareMathOperator{\arccot}{arccot}
\DeclareMathOperator{\arcsec}{arcsec}
\DeclareMathOperator{\arccsc}{arccsc}
\DeclareMathOperator{\si}{Si}
\DeclareMathOperator{\proj}{\vec{proj}}
\DeclareMathOperator{\scal}{scal}
\DeclareMathOperator{\sign}{sign}


%% \newcommand{\tightoverset}[2]{% for arrow vec
%%   \mathop{#2}\limits^{\vbox to -.5ex{\kern-0.75ex\hbox{$#1$}\vss}}}
\newcommand{\arrowvec}{\overrightarrow}
%\renewcommand{\vec}[1]{\arrowvec{\mathbf{#1}}}
\renewcommand{\vec}{\mathbf}
\newcommand{\veci}{{\boldsymbol{\hat{\imath}}}}
\newcommand{\vecj}{{\boldsymbol{\hat{\jmath}}}}
\newcommand{\veck}{{\boldsymbol{\hat{k}}}}
\newcommand{\vecl}{\boldsymbol{\l}}
\newcommand{\uvec}[1]{\mathbf{\hat{#1}}}
\newcommand{\utan}{\mathbf{\hat{t}}}
\newcommand{\unormal}{\mathbf{\hat{n}}}
\newcommand{\ubinormal}{\mathbf{\hat{b}}}

\newcommand{\dotp}{\bullet}
\newcommand{\cross}{\boldsymbol\times}
\newcommand{\grad}{\boldsymbol\nabla}
\newcommand{\divergence}{\grad\dotp}
\newcommand{\curl}{\grad\cross}
%\DeclareMathOperator{\divergence}{divergence}
%\DeclareMathOperator{\curl}[1]{\grad\cross #1}
\newcommand{\lto}{\mathop{\longrightarrow\,}\limits}

\renewcommand{\bar}{\overline}

\colorlet{textColor}{black} 
\colorlet{background}{white}
\colorlet{penColor}{blue!50!black} % Color of a curve in a plot
\colorlet{penColor2}{red!50!black}% Color of a curve in a plot
\colorlet{penColor3}{red!50!blue} % Color of a curve in a plot
\colorlet{penColor4}{green!50!black} % Color of a curve in a plot
\colorlet{penColor5}{orange!80!black} % Color of a curve in a plot
\colorlet{penColor6}{yellow!70!black} % Color of a curve in a plot
\colorlet{fill1}{penColor!20} % Color of fill in a plot
\colorlet{fill2}{penColor2!20} % Color of fill in a plot
\colorlet{fillp}{fill1} % Color of positive area
\colorlet{filln}{penColor2!20} % Color of negative area
\colorlet{fill3}{penColor3!20} % Fill
\colorlet{fill4}{penColor4!20} % Fill
\colorlet{fill5}{penColor5!20} % Fill
\colorlet{gridColor}{gray!50} % Color of grid in a plot

\newcommand{\surfaceColor}{violet}
\newcommand{\surfaceColorTwo}{redyellow}
\newcommand{\sliceColor}{greenyellow}




\pgfmathdeclarefunction{gauss}{2}{% gives gaussian
  \pgfmathparse{1/(#2*sqrt(2*pi))*exp(-((x-#1)^2)/(2*#2^2))}%
}


%%%%%%%%%%%%%
%% Vectors
%%%%%%%%%%%%%

%% Simple horiz vectors
\renewcommand{\vector}[1]{\left\langle #1\right\rangle}


%% %% Complex Horiz Vectors with angle brackets
%% \makeatletter
%% \renewcommand{\vector}[2][ , ]{\left\langle%
%%   \def\nextitem{\def\nextitem{#1}}%
%%   \@for \el:=#2\do{\nextitem\el}\right\rangle%
%% }
%% \makeatother

%% %% Vertical Vectors
%% \def\vector#1{\begin{bmatrix}\vecListA#1,,\end{bmatrix}}
%% \def\vecListA#1,{\if,#1,\else #1\cr \expandafter \vecListA \fi}

%%%%%%%%%%%%%
%% End of vectors
%%%%%%%%%%%%%

%\newcommand{\fullwidth}{}
%\newcommand{\normalwidth}{}



%% makes a snazzy t-chart for evaluating functions
%\newenvironment{tchart}{\rowcolors{2}{}{background!90!textColor}\array}{\endarray}

%%This is to help with formatting on future title pages.
\newenvironment{sectionOutcomes}{}{} 



%% Flowchart stuff
%\tikzstyle{startstop} = [rectangle, rounded corners, minimum width=3cm, minimum height=1cm,text centered, draw=black]
%\tikzstyle{question} = [rectangle, minimum width=3cm, minimum height=1cm, text centered, draw=black]
%\tikzstyle{decision} = [trapezium, trapezium left angle=70, trapezium right angle=110, minimum width=3cm, minimum height=1cm, text centered, draw=black]
%\tikzstyle{question} = [rectangle, rounded corners, minimum width=3cm, minimum height=1cm,text centered, draw=black]
%\tikzstyle{process} = [rectangle, minimum width=3cm, minimum height=1cm, text centered, draw=black]
%\tikzstyle{decision} = [trapezium, trapezium left angle=70, trapezium right angle=110, minimum width=3cm, minimum height=1cm, text centered, draw=black]



\title{2.12 - Increasing, Decreasing, and Concavity}

\begin{document}
\begin{abstract}
\end{abstract}
\maketitle

Derivatives can tell us a lot about the shape of a graph.

\section{Increasing and Decreasing Functions}

Let's compare the graphs of $f$ and $f'$ for some derivatives we have computed before:

\[f(x)=x^2, f'(x)=2x\]
\begin{image}
    \begin{tikzpicture}
      \begin{axis}[
          xmin=-2,xmax=2,ymin=-4,ymax=4,
          axis lines=center,
          ticks=none,
          width=6in,
          height=3in,
          every axis y label/.style={at=(current axis.above origin),anchor=south},
          every axis x label/.style={at=(current axis.right of origin),anchor=west},
        ]
        %\addplot [ultra thick,dashed, penColor,smooth, domain=(-2:2)] {x^3+.3*x^2-2*x)};
        \addplot [ultra thick,penColor,smooth, domain=(-2:2)] {x^2} node [pos=0.9, below right] {$f$};
        \addplot [ultra thick,penColor2,smooth, domain=(-2:2)] {2*x} node [pos=0.9, above left] {$f'$};
      \end{axis}
    \end{tikzpicture}
  \end{image}

\[f(x)=3x+2, f'(x)=3\]
\begin{image}
    \begin{tikzpicture}
      \begin{axis}[
          xmin=-2,xmax=2,ymin=-4,ymax=8,
          axis lines=center,
          ticks=none,
          width=6in,
          height=3in,
          every axis y label/.style={at=(current axis.above origin),anchor=south},
          every axis x label/.style={at=(current axis.right of origin),anchor=west},
        ]
        %\addplot [ultra thick,dashed, penColor,smooth, domain=(-2:2)] {x^3+.3*x^2-2*x)};
        \addplot [ultra thick,penColor,smooth, domain=(-2:2)] {3*x+2} node [pos=0.9, below right] {$f$};
        \addplot [ultra thick,penColor2,smooth, domain=(-2:2)] {3} node [pos=0.9, below right] {$f'$};

      \end{axis}
    \end{tikzpicture}
  \end{image}

  % (su18:TK) fixed an error below
  % $$f(x)=|x|, f'(x)=\begin{cases} 1 &  \text{for } x<0 \\ -1 &
  %   \text{for } x>0 \end{cases}$$ %fliiped signs
  \[f(x)=|x|, f'(x)=\begin{cases} 1 &  \text{for } x>0 \\ -1 & \text{for } x<0 \end{cases}\]

  \begin{image}
    \begin{tikzpicture}
      \begin{axis}[
          xmin=-2,xmax=2,ymin=-2,ymax=2,
          axis lines=center,
          ticks=none,
          width=6in,
          height=3in,
          every axis y label/.style={at=(current axis.above origin),anchor=south},
          every axis x label/.style={at=(current axis.right of origin),anchor=west},
        ]
        %\addplot [ultra thick,dashed, penColor,smooth, domain=(-2:2)] {x^3+.3*x^2-2*x)};
        \addplot [ultra thick,penColor,smooth, domain=(-2:2)] {abs(x)} node [pos=0.9, above left] {$f$};
        \addplot [ultra thick,penColor2,smooth, domain=(-2:0)] {-1};
        \addplot [ultra thick,penColor2,smooth, domain=(0:2)] {1} node [pos=0.8, below right] {$f'$};
        \addplot[color=penColor,fill=background,only marks,mark=*] coordinates{(0,-1)};
        \addplot[color=penColor,fill=background,only marks,mark=*] coordinates{(0,1)};
      \end{axis}
    \end{tikzpicture}
  \end{image}
  \begin{question}
 For each of the three pairs of functions, describe $y=f(x)$ when $f'$
  is positive, and when $f'$  is negative.\\

NOTE: When we say a graph is ``increasing", we mean loosely that the $y$ values are going up as we read the graph from left to right.

Similarly, we say a graph is ``decreasing" when the $y$ values are going down as we read the graph from left to right.

Later, we will have a more mathematical definition for increasing and decreasing. For now, the concepts above will suffice.

  \begin{prompt}
    When $f'$ is positive, $y=f(x)$ is \wordChoice{\choice{positive}\choice[correct]{increasing}\choice{negative}\choice{decreasing}}.
    When $f'$ is negative, $y=f(x)$ is \wordChoice{\choice{positive}\choice{increasing}\choice{negative}\choice[correct]{decreasing}}
  \end{prompt}
      \end{question}

Think about the lines tangent to the graph of the function in the examples above.
In the first graph, on $(-\infty,0)$, the slopes of the tangent lines are all negative, producing negative $y$ values in the graph of the derivative (underneath the $x$ axis).

In the same graph, the slopes of the tangent lines are positive from $(0,\infty)$, producing positive $y$ values in the graph of the derivative (above the $x$-axis).

Since the derivative gives us a formula for the slope of a tangent
line to a curve, we can gain information about a function purely from
the sign of the derivative.  In particular, we have the following theorem

\begin{theorem}
A function $f$ is \textbf{increasing} on any interval $I$ where $f'(x)>0$, for all $x$ in $I$.\\
A function $f$  is \textbf{decreasing} on any interval $I$ where $f'(x)<0$, for all $x$ in $I$.\\
 \end{theorem}

\begin{question}
  Here we see the graph of $f'$, the derivative of some function $f$.
  \begin{image}
    \begin{tikzpicture}
      \begin{axis}[
          xmin=-2,xmax=2,ymin=-8,ymax=8,
          axis lines=center,
          ticks=none,
          width=6in,
          height=3in,
          every axis y label/.style={at=(current axis.above origin),anchor=south},
          every axis x label/.style={at=(current axis.right of origin),anchor=west},
        ]
        %\addplot [ultra thick,dashed, penColor,smooth, domain=(-2:2)] {x^3+.3*x^2-2*x)};
        \addplot [ultra thick,penColor,smooth, domain=(-2:2)] {3*x^2+2*.3*x-2)};
      \end{axis}
    \end{tikzpicture}
  \end{image}
\end{question}

    Which of the following graphs could be $y = f(x)$?
     \begin{multipleChoice}
       \choice{\begin{tikzpicture}[framed,scale=1,baseline=3ex]
           \begin{axis}[
               xmin=-2,xmax=2,ymin=-8,ymax=8,
               axis lines=center,
               ticks=none,
               width=2in,
               height=1in,
               every axis y label/.style={at=(current axis.above origin),anchor=south},
               every axis x label/.style={at=(current axis.right of origin),anchor=west},
             ]
             \addplot [ultra thick,penColor,smooth, domain=(-2:2)] {3*x^2+2*.3*x-2)};
           \end{axis}
       \end{tikzpicture}}
       \choice[correct]{\begin{tikzpicture}[framed,scale=1,baseline=3ex]
           \begin{axis}[
               xmin=-2,xmax=2,ymin=-8,ymax=8,
               axis lines=center,
               ticks=none,
               width=2in,
               height=1in,
               every axis y label/.style={at=(current axis.above origin),anchor=south},
               every axis x label/.style={at=(current axis.right of origin),anchor=west},
             ]
             \addplot [ultra thick,penColor,smooth, domain=(-2:2)] {x^3+.3*x^2-2*x)};
           \end{axis}
       \end{tikzpicture}}
       \choice{\begin{tikzpicture}[framed,scale=1,baseline=3ex]
           \begin{axis}[
               xmin=-2,xmax=2,ymin=-8,ymax=8,
               axis lines=center,
               ticks=none,
               width=2in,
               height=1in,
               every axis y label/.style={at=(current axis.above origin),anchor=south},
               every axis x label/.style={at=(current axis.right of origin),anchor=west},
             ]
             \addplot [ultra thick,penColor,smooth, domain=(-2:2)] {6*x+2*.3)};
           \end{axis}
       \end{tikzpicture}}
     \end{multipleChoice}

\section{Higher Order Derivatives}

Since the derivative is a function, we could take the derivative again. What does this mean graphically?

  \begin{image}
      \begin{tikzpicture}
        \begin{axis}[
            xmin=-6.75,xmax=6.75,ymin=-1.5,ymax=1.5,
            axis lines=center,
            ticks=none,
            width=6in,
            height=3in,
            every axis y label/.style={at=(current axis.above origin),anchor=south},
            every axis x label/.style={at=(current axis.right of origin),anchor=west},
          ]        
          \addplot [very thick, penColor, samples=100,smooth, domain=(-6.75:6.75)] {-sin(deg(x))};
          \addlegendentry{$A$};
          \addplot [very thick, dashed,penColor, samples=100,smooth, domain=(-6.75:6.75)] {cos(deg(x))};
          \addlegendentry{$B$};
          \addplot [very thick, dotted,penColor, samples=100,smooth, domain=(-6.75:6.75)] {sin(deg(x))};
          \addlegendentry{$C$};
        \end{axis}
  \end{tikzpicture}
  \end{image}


\begin{problem}
Which of the following is true?
\begin{multipleChoice}
  \choice{Curve $A$ is increasing when curve $B$ is positive.}
  \choice{Curve $A$ is increasing when curve $C$ is positive.}
  \choice[correct]{None of the above.}
\end{multipleChoice}
\end{problem}

\begin{problem}
Which of the following is true?
\begin{multipleChoice}
  \choice[correct]{Curve $B$ is increasing when curve $A$ is positive.}
  \choice{Curve $B$ is increasing when curve $C$ is positive.}
  \choice{None of the above.}
\end{multipleChoice}
\end{problem}

\begin{problem}
Which of the following is true?
\begin{multipleChoice}
  \choice{Curve $C$ is increasing when curve $A$ is positive.}
  \choice[correct]{Curve $C$ is increasing when curve $B$ is positive.}
  \choice{None of the above.}
\end{multipleChoice}
\end{problem}

\begin{question}
\author{Nela Lakos}
	Consider the graph of the function $f$ below:
	\begin{image}
          \begin{tikzpicture}
	    \begin{axis}[
	        ticks=none,
                domain=-2.5:2.5,
                width=6in,
                height=3in,
                xmin=-1.5, xmax=1.5,
                ymin=-1, ymax=1,
                axis lines =middle, xlabel=$x$, ylabel=$y$,
                every axis y label/.style={at=(current axis.above origin),anchor=south},
                every axis x label/.style={at=(current axis.right of origin),anchor=west},
              ]
              \addplot [very thick, penColor2, smooth, samples=100,domain=-2:2] {x^3-x-1/6};
              \node at (axis cs:1.3,1.5) [penColor2, anchor=west] {$f$};

              \addplot[color=penColor,fill=penColor,only marks,mark=*] coordinates{(-.58,0)};%%closed
          
              \addplot[color=penColor,fill=penColor,only marks,mark=*] coordinates{(.58,0)};%%closed
                   
              
	     
              \node at (axis cs:-0.67,-0.1) [penColor, anchor=west] {$a$};
      
              \node at (axis cs:0.47,-0.1) [penColor, anchor=west] {$b$};
         
              \node at (axis cs:1.15,0.6) [penColor2] {$f$};
            \end{axis}
          \end{tikzpicture}
        \end{image}
	On which of the following intervals is $f$ increasing?
	\begin{selectAll}
		\choice[correct]{$(-\infty,a)$}
		\choice{$(-\infty,b)$}
		\choice{$(a,b)$}
		\choice{$(a,+\infty)$}
		\choice[correct]{$(b,+\infty)$}
	\end{selectAll}
	\begin {explanation} 
		The function $f$ is not increasing on the interval $(-\infty,b)$, because if we pick a pair of numbers from $(-\infty,b)$, say, $x_{1}=a$, and $x_{2}=0$,
		then $x_{1}<x_{2}$, but  $f(x_{1})>f(x_{2})$.
	\end {explanation}
\end{question}
 
\begin{question}
  Below we have graphed $y=f(x)$:
  \begin{image}
  \begin{tikzpicture}
	\begin{axis}[
            xmin=-2,xmax=2,ymin=-8,ymax=8,
            axis lines=center,
            width=6in,
            height=3in,
            every axis y label/.style={at=(current axis.above origin),anchor=south},
            every axis x label/.style={at=(current axis.right of origin),anchor=west},
          ]        
          \addplot [very thick,penColor,smooth, domain=(-2:2)] {x^3+x^2-2*x)};
        \end{axis}
  \end{tikzpicture}
  \end{image}
  Is the first derivative positive or negative on the interval $-1<x<1/2$?
  \begin{prompt}
    \begin{multipleChoice}
      \choice{Positive}
      \choice[correct]{Negative}
    \end{multipleChoice}
  \end{prompt}
\end{question}

\begin{question}
  Below we have graphed $y=f'(x)$:
  \begin{image}
  \begin{tikzpicture}
	\begin{axis}[
            xmin=-2,xmax=2,ymin=-8,ymax=8,
            axis lines=center,
            width=6in,
            height=3in,
            every axis y label/.style={at=(current axis.above origin),anchor=south},
            every axis x label/.style={at=(current axis.right of origin),anchor=west},
          ]        
          \addplot [very thick,penColor,smooth, domain=(-2:2)] {x^3+x^2-2*x)};
        \end{axis}
  \end{tikzpicture}
  \end{image}
  Is the graph of $f(x)$ increasing or decreasing as $x$ increases on
  the interval $-1<x<0$?
  \begin{prompt}
    \begin{multipleChoice}
      \choice[correct]{Increasing}
      \choice{Decreasing}
    \end{multipleChoice}
  \end{prompt}
\end{question}

We call the derivative of the derivative the \dfn{second
  derivative}, the derivative of the derivative of the derivative the
\dfn{third derivative}, and so on. We have special notation for
higher derivatives, check it out:
\begin{description}
\item[First derivative:] $\frac{d}{dx} f(x) = f'(x) = f^{(1)}(x)$.
\item[Second derivative:] $\dfrac{d^2}{dx^2} f(x) = f''(x) = f^{(2)}(x)$.
\item[Third derivative:] $\frac{d^3}{dx^3} f(x) = f'''(x) = f^{(3)}(x)$.
\end{description}

We use the facts above in our next example.

\begin{example}
  Here we have unlabeled graphs of $f$, $f'$, and $f''$:
  \begin{image}
  \begin{tikzpicture}
	\begin{axis}[
            xmin=-2,xmax=2,ymin=-8,ymax=8,
            axis lines=center,
            ticks=none,
            width=6in,
            height=3in,
            every axis y label/.style={at=(current axis.above origin),anchor=south},
            every axis x label/.style={at=(current axis.right of origin),anchor=west},
          ]        
          \addplot [very thick,penColor,smooth, domain=(-2:2)] {x^3+.3*x^2-2*x)};
          \addlegendentry{$A$};
          \addplot [very thick, dashed,penColor,smooth, domain=(-2:2)] {3*x^2+2*.3*x-2)};
          \addlegendentry{$B$};
          \addplot [very thick, dotted,penColor,smooth, domain=(-2:2)] {6*x+2*.3)};
          \addlegendentry{$C$};
        \end{axis}
  \end{tikzpicture}
  \end{image}
  Identify each curve above as a graph of $f$, $f'$, or $f''$.
  \begin{explanation} 
    Here we see three curves, $A$, $B$, and $C$. Since $A$ is
    \wordChoice{\choice{positive} \choice{negative}
      \choice[correct]{increasing} \choice{decreasing}} when $B$ is
    positive and
    \wordChoice{\choice{positive}\choice{negative}\choice{increasing}\choice[correct]{decreasing}}
    when $B$ is negative, we see
    \[
    A'=B.
    \]
    Since $B$ is increasing when $C$ is
    \wordChoice{\choice[correct]{positive}\choice{negative}\choice{increasing}
      \choice{decreasing}} and decreasing when $C$ is
    \wordChoice{\choice{positive}\choice[correct]{negative}\choice{increasing}\choice{decreasing}}, we see
    \[
    B'=C.
    \]
    Hence $f=A$, $f'=B$, and $f''=C$.
  \end{explanation}
\end{example}




\begin{example}
    Here we have unlabeled graphs of $f$, $f'$, and $f''$:
    \begin{image}
      \begin{tikzpicture}
	\begin{axis}[
            domain=-4:4,
            ticks=none,
            ymax=2, ymin=-2,
            xmax=4, xmin=-4,
            axis lines =middle,
            every axis y label/.style={at=(current axis.above origin),anchor=south},
            every axis x label/.style={at=(current axis.right of origin),anchor=west},
            width=6in,
            height=3in,
          ]
          \addplot [very thick, penColor,smooth,samples=100] {2/(.75*sqrt(2*pi))*exp(-((x)^2)/(2*.75^2)) *(-x)/(.75^2)};
          \addlegendentry{$A$};
          \addplot [very thick, dashed,penColor,smooth,samples=100] {2*gauss(0,.75)};
          \addlegendentry{$B$};
          \addplot [very thick, dotted,penColor,smooth,samples=100] {2/(.75*sqrt(2*pi))*exp(-((x)^2)/(2*.75^2)) *(-1)/(.75^2) +
            2/(.75*sqrt(2*pi))*exp(-((x)^2)/(2*.75^2)) *(x^2)/(.75^4)};
          \addlegendentry{$C$};
        \end{axis}
          \end{tikzpicture}
\end{image}
    Identify each curve above as a graph of $f$, $f'$, or $f''$.
      \begin{explanation} 
        Here we see three curves, $A$, $B$, and $C$. Since $B$ is
        \wordChoice{\choice{positive}\choice{negative}\choice[correct]{increasing}\choice{decreasing}} when $A$
        is positive and
        \wordChoice{\choice{positive}\choice{negative}\choice{increasing}\choice[correct]{decreasing}} when $A$
        is negative, we see
        \[
        B'=A.
        \]
        Since $A$ is increasing when $C$ is
        \wordChoice{\choice[correct]{positive}\choice{negative}\choice{increasing}\choice{decreasing}}
          and decreasing when $C$ is
          \wordChoice{\choice{positive}\choice[correct]{negative}\choice{increasing}\choice{decreasing}}, we
          see
        \[
        A'=C.
        \]
        Hence $f=\answer[given]{B}$, $f'=\answer[given]{A}$, and
        $f''=\answer[given]{C}$.
      \end{explanation}
\end{example}

\begin{example}
  Here we have unlabeled graphs of $f$, $f'$, and $f''$:
  \begin{image}
  \begin{tikzpicture}
	\begin{axis}[
            xmin=-6.75,xmax=6.75,ymin=-1.5,ymax=1.5,
            axis lines=center,
            ticks=none,
            width=6in,
            height=3in,
            every axis y label/.style={at=(current axis.above origin),anchor=south},
            every axis x label/.style={at=(current axis.right of origin),anchor=west},
          ]        
          \addplot [very thick, penColor, samples=100,smooth, domain=(-6.75:6.75)] {-sin(deg(x))};
          \addlegendentry{$A$};
          \addplot [very thick, dashed,penColor, samples=100,smooth, domain=(-6.75:6.75)] {cos(deg(x))};
          \addlegendentry{$B$};
          \addplot [very thick, dotted,penColor, samples=100,smooth, domain=(-6.75:6.75)] {sin(deg(x))};
          \addlegendentry{$C$};
        \end{axis}
  \end{tikzpicture}
  \end{image}
  Identify each curve above as a graph of $f$, $f'$, or $f''$.
  %One is of $f$, another is of $f'$ and a third is of $f''$.  Explain
  %what strategies you could use to identify which graph corresponds
    \begin{explanation} %%BADBAD Need Dropdown
    Here we see three curves, $A$, $B$, and $C$. Since $C$ is
    \wordChoice{\choice{positive}\choice{negative}\choice[correct]{increasing}\choice{decreasing}} when $B$ is
    positive and \wordChoice{\choice{positive}\choice{negative}\choice{increasing}\choice[correct]{decreasing}}
    when $B$ is negative, we see
    \[
    C'=B.
    \]
    Since $B$ is increasing when $A$ is
    \wordChoice{\choice[correct]{positive}\choice{negative}\choice{increasing}\choice{decreasing}} and
    decreasing when $A$ is
    \wordChoice{\choice{positive}\choice[correct]{negative}\choice{increasing}\choice{decreasing}}, we see
    \[
    B'=A.
    \]
    Hence $f=\answer[given]{C}$, $f'=\answer[given]{B}$, and
    $f''=\answer[given]{A}$.
  \end{explanation}
\end{example}

So far, we know how to determine if a function is increasing or decreasing by looking at the sign of its
derivative.  If all we know about $f$ is the increasing/decreasing information, we no not have enough
information to say how $f$ is behaving accurately enough.  Consider the following four possibilities:

\begin{image}
  \begin{tikzpicture}
    \draw (0,0) -- (0,12);
    \draw (0,0) -- (12,0);
    \draw (6,0) -- (6,12);
    \draw (0,6) -- (12,6);
    \draw (12,0) -- (12,12);
    \draw (0,12) -- (12,12);
    
    \node at (3,12.4) {\Large $f$ decreasing};
    \node at (9,12.4) {\Large$f$ increasing};
    
    \draw [penColor,ultra thick,domain=180:270] plot ({2*cos(\x)+4}, {2*sin(\x)+11});
    \draw [penColor,ultra thick,domain=270:360] plot ({2*cos(\x)+8}, {2*sin(\x)+11});
    \draw [penColor,ultra thick,domain=0:90] plot ({2*cos(\x)+2}, {2*sin(\x)+3});
    \draw [penColor,ultra thick,domain=180:90] plot ({2*cos(\x)+10}, {2*sin(\x)+3});

  \end{tikzpicture}
\end{image}

In both graphs in the left-hand column, $f$ is decreasing.  In both graphs in the right-hand column,
$f$ is increasing.  What is the difference between the two rows?  Their concavity.
\begin{definition}\index{concave up}\index{concave down}
	Let $f$ be a  function differentiable on an open interval $I$.\\
		We say that the graph of  $f$ is \textbf{concave up} on $I$ if  $f'$, the derivative of $f$, is \textbf{increasing} on $I$.\\
		We say that the graph of  $f$ is \textbf{concave down} on $I$ if $f'$, the derivative of $f$, is \textbf{decreasing} on $I$.
\end{definition}
That is, the graph of $f$ is concave up if the graph locally lies above its tangent lines, and is concave down if the graph locally lies below its tangent lines.

The graphs of two functions, $f$ and $g$, both increasing on the given interval, are given below.
\begin{image}
  \begin{tikzpicture}
    \draw [penColor,  ultra thick,domain=180:270] plot ({2*cos(\x)+4},{9-2*sin(\x)});
    \draw [penColor, ultra thick,domain=270:360] plot ({2*cos(\x)+8}, {2*sin(\x)+11});
    \draw [color=red,thick, domain=1.92:2.2] plot({\x},{cot(193)*(\x-(2*cos(193)+4))+9-2*sin(193)});
    \draw [color=red, thick, domain=2.3:3.1] plot({\x},{cot(225)*(\x-(2*cos(225)+4))+9-2*sin(225)});    
    \draw [color=red, thick, domain=3.19:4.2] plot({\x},{cot(260)*(\x-(2*cos(260)+4))+9-2*sin(260)});
    \draw [color=red, thick, domain=7.8:9.01] plot({\x},{-cot(280)*(\x-(2*cos(280)+8))+11+2*sin(280)});
    \draw [color=red, thick, domain=9.05:9.79] plot({\x},{-cot(315)*(\x-(2*cos(315)+8))+11+2*sin(315)});
    \draw [color=red, thick, domain=9.82:10.08] plot({\x},{-cot(347)*(\x-(2*cos(347)+8))+11+2*sin(347)});
    \node at (3,7.5) [text width=5cm] {
      The function $f$ is increasing, while the rate itself is decreasing.
      In this case the curve  $y=f(x)$is \dfn{concave down}.};
    \node at (9,7.5) [text width=5cm] {
      The function $g$ is increasing, while the rate itself is increasing.
      In this case the curve  $y=g(x)$is \dfn{concave up}.};
  \end{tikzpicture}
\end{image}

\begin{example}
A graph of $y=f(x)$ is given below, with domain $\left(-\frac{3}{2} , \frac{3}{2} \right)$.
        \begin{image}
    \begin{tikzpicture}
    \begin{axis}[
        xmin=-1.57,xmax=1.57,ymin=-1.5,ymax=1.5,
        axis lines=center,
        width=6in,
        height=3in,
        every axis y label/.style={at=(current axis.above origin),anchor=south},
        every axis x label/.style={at=(current axis.right of origin),anchor=west},
      ]
      \addplot [penColor,ultra thick,domain=-1.5:1.5,smooth] {sin(2*deg(x))};

    \end{axis}
  \end{tikzpicture}
  \end{image}
On what intervals is $f$ concave up?  On what intervals is $f$ concave down?
\begin{explanation}
	Let's draw some tangent lines on the graph.
	        \begin{image}
    \begin{tikzpicture}
    \begin{axis}[
        xmin=-1.57,xmax=1.57,ymin=-1.5,ymax=1.5,
        axis lines=center,
        width=6in,
        height=3in,
        every axis y label/.style={at=(current axis.above origin),anchor=south},
        every axis x label/.style={at=(current axis.right of origin),anchor=west},
      ]
      \addplot [penColor,ultra thick,domain=-1.5:1.5,smooth] {sin(2*deg(x))};
      \addplot [penColor2,ultra thick,domain=-1:-0.57,smooth] {-1};
      \addplot [penColor2,ultra thick,domain=-0.7:-0.3,smooth] {2*cos(2*deg(-0.5))*(x+0.5)+sin(2*deg(-0.5))};
      \addplot [penColor2,ultra thick,domain=-1.3:-0.9,smooth] {2*cos(2*deg(-1.1))*(x+1.1)+sin(2*deg(-1.1))};
      \addplot [penColor3,ultra thick,domain=0.9:1.3,smooth] {2*cos(2*deg(1.1))*(x-1.1)+sin(2*deg(1.1))};
      \addplot [penColor3,ultra thick,domain=0.3:0.7,smooth] {2*cos(2*deg(0.5))*(x-0.5)+sin(2*deg(0.5))};
      \addplot [penColor3,ultra thick,domain=0.57:1,smooth] {1};
    \end{axis}
  \end{tikzpicture}
  \end{image}
  
  The graph is above the tangent lines for $x$ in $\left(-\frac{3}{2}, 0\right)$, and below the tangent lines we sketched on $\left(0, \frac{3}{2}\right)$.
  
  Therefore, $f$ is concave up on $\left(\answer{-\frac{3}{2}}, \answer{0}\right)$, and concave down on $\left(\answer{0}, \answer{\frac{3}{2}}\right)$.
\end{explanation}
\end{example}
In that example, examine the intervals where $f$ was concave up.  At the beginning of that interval, $f$ was decreasing, and at the end, $f$ was increasing.
The slopes in that interval increased from a negative value to a positive value.  In the interval where $f$ was concave down, the slopes started positive
and ended negative.  The slopes in that interval were decreasing.

We know that the sign of the derivative tells us whether a function is increasing or decreasing at some point. Likewise, the sign of the
second derivative $f''(x)$ tells us whether $f'(x)$ is increasing or decreasing at $x$.  Let's use this to add more details into the chart from above.

\begin{image}
  \begin{tikzpicture}
    \draw (0,0) -- (0,12);
    \draw (0,0) -- (12,0);
    \draw (6,0) -- (6,12);
    \draw (0,6) -- (12,6);
    \draw (12,0) -- (12,12);
    \draw (0,12) -- (12,12);
    
    \node at (-1.3,9) {\Large$0<f''(x)$};
    \node at (-1.3,3) {\Large$f''(x)<0$};
    \node at (3,12.4) {\Large$f'(x)<0$};
    \node at (9,12.4) {\Large$0<f'(x)$};
    
    \draw [penColor,ultra thick,domain=180:270] plot ({2*cos(\x)+4}, {2*sin(\x)+11});
    \draw [penColor,ultra thick,domain=270:360] plot ({2*cos(\x)+8}, {2*sin(\x)+11});
    \draw [penColor,ultra thick,domain=0:90] plot ({2*cos(\x)+2}, {2*sin(\x)+3});
    \draw [penColor,ultra thick,domain=180:90] plot ({2*cos(\x)+10}, {2*sin(\x)+3});

    \node at (3,7.5) [text width=5cm] {\large
      Here $y=f(x)$ is decreasing, while the rate itself is increasing.
      In this case the curve is \dfn{concave up}.};

    \node at (9,7.5) [text width=5cm] {\large
      Here $y=f(x)$ is increasing, while the rate itself is increasing.
      In this case the curve is \dfn{concave up}.};

    \node at (3,1.5) [text width=5cm] {\large
      Here $y=f(x)$ is decreasing, while the rate itself is decreasing.
      In this case the curve is \dfn{concave down}.};

    \node at (9,1.5) [text width=5cm] {\large
      Here $y=f(x)$ is increasing, while the rate itself is decreasing.
      In this case the curve is \dfn{concave down}.};
  \end{tikzpicture}
\end{image}



\end{document}
