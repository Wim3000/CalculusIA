\documentclass{ximera}

%\usepackage{todonotes}

\newcommand{\todo}{}

\usepackage{esint} % for \oiint
\graphicspath{
  {./}
  {ximeraTutorial/}
  {basicPhilosophy/}
  {functionsOfSeveralVariables/}
  {normalVectors/}
  {lagrangeMultipliers/}
  {vectorFields/}
  {greensTheorem/}
  {shapeOfThingsToCome/}
}
\usepackage{comment} %% used in what is a limit
\usepackage[valunder]{signchart} %% used in graphing sign chart

\newcommand{\mooculus}{\textsf{\textbf{MOOC}\textnormal{\textsf{ULUS}}}}

\usepackage{tkz-euclide}
\tikzset{>=stealth} %% cool arrow head
\tikzset{shorten <>/.style={ shorten >=#1, shorten <=#1 } } %% allows shorter vectors


\pgfplotsset{soldot/.style={color=black,only marks,mark=*}} %% USED by piecewise functions
\pgfplotsset{holdot/.style={color=black,fill=white,only marks,mark=*}}
\usetikzlibrary{arrows.meta}


\usepackage{tkz-tab}  %% sign charts
\usepackage{polynom}

\usetikzlibrary{backgrounds} %% for boxes around graphs
\usetikzlibrary{shapes,positioning}  %% Clouds and stars
\usetikzlibrary{matrix} %% for matrix
\usepgfplotslibrary{polar} %% for polar plots
%\usetkzobj{all}
\usepackage[makeroom]{cancel} %% for strike outs
%\usepackage{mathtools} %% for pretty underbrace % Breaks Ximera
\usepackage{multicol}
\usepackage{pgffor} %% required for integral for loops


%% http://tex.stackexchange.com/questions/66490/drawing-a-tikz-arc-specifying-the-center
%% Draws beach ball
\tikzset{pics/carc/.style args={#1:#2:#3}{code={\draw[pic actions] (#1:#3) arc(#1:#2:#3);}}}



\usepackage{array}
\setlength{\extrarowheight}{+.1cm}   
\newdimen\digitwidth
\settowidth\digitwidth{9}
\def\divrule#1#2{
\noalign{\moveright#1\digitwidth
\vbox{\hrule width#2\digitwidth}}}




\newcommand{\RR}{\mathbb R}
\newcommand{\R}{\mathbb R}
\newcommand{\N}{\mathbb N}
\newcommand{\Z}{\mathbb Z}

\newcommand{\sagemath}{\textsf{SageMath}}


%\renewcommand{\d}{\,d\!}
\renewcommand{\d}{\mathop{}\!d}
\newcommand{\dd}[2][]{\frac{\d #1}{\d #2}}
\newcommand{\pp}[2][]{\frac{\partial #1}{\partial #2}}
\renewcommand{\l}{\ell}
\newcommand{\ddx}{\frac{d}{\d x}}
\newcommand{\ddt}{\frac{d}{\d t}}


\newcommand{\zeroOverZero}{\ensuremath{\boldsymbol{\tfrac{0}{0}}}}
\newcommand{\inftyOverInfty}{\ensuremath{\boldsymbol{\tfrac{\infty}{\infty}}}}
\newcommand{\zeroOverInfty}{\ensuremath{\boldsymbol{\tfrac{0}{\infty}}}}
\newcommand{\zeroTimesInfty}{\ensuremath{\small\boldsymbol{0\cdot \infty}}}
\newcommand{\inftyMinusInfty}{\ensuremath{\small\boldsymbol{\infty - \infty}}}
\newcommand{\oneToInfty}{\ensuremath{\boldsymbol{1^\infty}}}
\newcommand{\zeroToZero}{\ensuremath{\boldsymbol{0^0}}}
\newcommand{\inftyToZero}{\ensuremath{\boldsymbol{\infty^0}}}



\newcommand{\numOverZero}{\ensuremath{\boldsymbol{\tfrac{\#}{0}}}}
\newcommand{\dfn}{\textbf}
%\newcommand{\unit}{\,\mathrm}
\newcommand{\unit}{\mathop{}\!\mathrm}
\newcommand{\eval}[1]{\bigg[ #1 \bigg]}
\newcommand{\seq}[1]{\left( #1 \right)}
\renewcommand{\epsilon}{\varepsilon}
\renewcommand{\phi}{\varphi}


\renewcommand{\iff}{\Leftrightarrow}

\DeclareMathOperator{\arccot}{arccot}
\DeclareMathOperator{\arcsec}{arcsec}
\DeclareMathOperator{\arccsc}{arccsc}
\DeclareMathOperator{\si}{Si}
\DeclareMathOperator{\proj}{\vec{proj}}
\DeclareMathOperator{\scal}{scal}
\DeclareMathOperator{\sign}{sign}


%% \newcommand{\tightoverset}[2]{% for arrow vec
%%   \mathop{#2}\limits^{\vbox to -.5ex{\kern-0.75ex\hbox{$#1$}\vss}}}
\newcommand{\arrowvec}{\overrightarrow}
%\renewcommand{\vec}[1]{\arrowvec{\mathbf{#1}}}
\renewcommand{\vec}{\mathbf}
\newcommand{\veci}{{\boldsymbol{\hat{\imath}}}}
\newcommand{\vecj}{{\boldsymbol{\hat{\jmath}}}}
\newcommand{\veck}{{\boldsymbol{\hat{k}}}}
\newcommand{\vecl}{\boldsymbol{\l}}
\newcommand{\uvec}[1]{\mathbf{\hat{#1}}}
\newcommand{\utan}{\mathbf{\hat{t}}}
\newcommand{\unormal}{\mathbf{\hat{n}}}
\newcommand{\ubinormal}{\mathbf{\hat{b}}}

\newcommand{\dotp}{\bullet}
\newcommand{\cross}{\boldsymbol\times}
\newcommand{\grad}{\boldsymbol\nabla}
\newcommand{\divergence}{\grad\dotp}
\newcommand{\curl}{\grad\cross}
%\DeclareMathOperator{\divergence}{divergence}
%\DeclareMathOperator{\curl}[1]{\grad\cross #1}
\newcommand{\lto}{\mathop{\longrightarrow\,}\limits}

\renewcommand{\bar}{\overline}

\colorlet{textColor}{black} 
\colorlet{background}{white}
\colorlet{penColor}{blue!50!black} % Color of a curve in a plot
\colorlet{penColor2}{red!50!black}% Color of a curve in a plot
\colorlet{penColor3}{red!50!blue} % Color of a curve in a plot
\colorlet{penColor4}{green!50!black} % Color of a curve in a plot
\colorlet{penColor5}{orange!80!black} % Color of a curve in a plot
\colorlet{penColor6}{yellow!70!black} % Color of a curve in a plot
\colorlet{fill1}{penColor!20} % Color of fill in a plot
\colorlet{fill2}{penColor2!20} % Color of fill in a plot
\colorlet{fillp}{fill1} % Color of positive area
\colorlet{filln}{penColor2!20} % Color of negative area
\colorlet{fill3}{penColor3!20} % Fill
\colorlet{fill4}{penColor4!20} % Fill
\colorlet{fill5}{penColor5!20} % Fill
\colorlet{gridColor}{gray!50} % Color of grid in a plot

\newcommand{\surfaceColor}{violet}
\newcommand{\surfaceColorTwo}{redyellow}
\newcommand{\sliceColor}{greenyellow}




\pgfmathdeclarefunction{gauss}{2}{% gives gaussian
  \pgfmathparse{1/(#2*sqrt(2*pi))*exp(-((x-#1)^2)/(2*#2^2))}%
}


%%%%%%%%%%%%%
%% Vectors
%%%%%%%%%%%%%

%% Simple horiz vectors
\renewcommand{\vector}[1]{\left\langle #1\right\rangle}


%% %% Complex Horiz Vectors with angle brackets
%% \makeatletter
%% \renewcommand{\vector}[2][ , ]{\left\langle%
%%   \def\nextitem{\def\nextitem{#1}}%
%%   \@for \el:=#2\do{\nextitem\el}\right\rangle%
%% }
%% \makeatother

%% %% Vertical Vectors
%% \def\vector#1{\begin{bmatrix}\vecListA#1,,\end{bmatrix}}
%% \def\vecListA#1,{\if,#1,\else #1\cr \expandafter \vecListA \fi}

%%%%%%%%%%%%%
%% End of vectors
%%%%%%%%%%%%%

%\newcommand{\fullwidth}{}
%\newcommand{\normalwidth}{}



%% makes a snazzy t-chart for evaluating functions
%\newenvironment{tchart}{\rowcolors{2}{}{background!90!textColor}\array}{\endarray}

%%This is to help with formatting on future title pages.
\newenvironment{sectionOutcomes}{}{} 



%% Flowchart stuff
%\tikzstyle{startstop} = [rectangle, rounded corners, minimum width=3cm, minimum height=1cm,text centered, draw=black]
%\tikzstyle{question} = [rectangle, minimum width=3cm, minimum height=1cm, text centered, draw=black]
%\tikzstyle{decision} = [trapezium, trapezium left angle=70, trapezium right angle=110, minimum width=3cm, minimum height=1cm, text centered, draw=black]
%\tikzstyle{question} = [rectangle, rounded corners, minimum width=3cm, minimum height=1cm,text centered, draw=black]
%\tikzstyle{process} = [rectangle, minimum width=3cm, minimum height=1cm, text centered, draw=black]
%\tikzstyle{decision} = [trapezium, trapezium left angle=70, trapezium right angle=110, minimum width=3cm, minimum height=1cm, text centered, draw=black]


\outcome{To be able to use the method of substitution to solve some ``simple'' integrals, with an emphasis on being able to correctly identify what to substitute for.}
\outcome{Undo the Chain Rule.}
\outcome{Calculate indefinite integrals (antiderivatives) using basic substitution.}
\outcome{Calculate definite integrals using basic substitution.}
\title{8.11 - The Substitution Rule with Definite Integrals}
\begin{document}

\maketitle


In the previous section we discussed solving indefinite integrals using the method of substitution. In this section, we will continue using substitution, but will now apply the method to definite integrals. 

Once again, consider chain rule. 

We know that if the functions $f,g$ are differentiable, we can apply the chain rule and obtain
\[
\ddx f(g(x)) = f'(g(x))g'(x)
\]
If the derivatives are continuous, we can use this equality to evaluate a definite integral. Namely,
\begin{align*}
  \int_a^b f'(g(x))g'(x) \d x &= \eval{f(g(x))}_a^b \\
  &= f(g(b)) - f(g(a)). \\
 \end{align*}
On the other hand, it is also true that
\begin{align*}
  \int_{g(a)}^{g(b)} f'(u)\d u &= \eval{f(u)}_{g(a)}^{g(b)}\\
  &= f(g(b)) - f(g(a)). \\
 \end{align*}
 Since the right hand sides of these equalities are equal,  the left hand sides must be equal, too. So, it follows that
 
\[
  \int_a^b f'(g(x))g'(x) \d x =  \int_{g(a)}^{g(b)} f'(u)\d u .
\]

 
This simple observation  leads to the following theorem. 


\begin{theorem}[Integral Substitution Formula]\index{integral substitution formula} 
Let $g'$ be the derivative of a  function $g$ and let $f'$ be the derivative of a function $f$. If $g'$ is continuous on the interval $[a,b]$ and  $f'$ is
continuous on the interval $[g(a),g(b)]$, then
\[
\int_a^b f'(g(x)) g'(x) \d x =\int_{g(a)}^{g(b)} f'(u) \d u.
\]
\end{theorem}
Just as we saw with indefinite integrals, sometimes we can find a substitution that will make a more complicated integral into a much easier integral. Notice a key difference between substitution for a definite integral versus an indefinite integral is that we will be changing the limits of integration. 

Let's jump in with an example.

\begin{example}
Compute:
\[
\int_0^2 x e^{x^2}\d x
\]
\begin{explanation}
We, as usual, start by selecting our $u$.

Notice,
\begin{align*}
u&=x^2\\
du&=2xdx\\
\frac{1}{2}du&=xdx.
\end{align*}
Further, we need to substitute the limits of integration since at the moment our integral is going from $x=0$ to $x=2$ and we need it to go between $u=?$ and $u=?$.

We get,
\begin{align*}
u(0) &= 0^2 = 0  \\
u(2) &=2^2 = 4.
\end{align*}
Finally, we can substitute our integral
\[
\int_0^2 x e^{x^2}\d x
=\frac{1}{2}\int_0^{4} e^{u} du=\frac{1}{2}\eval {e^u}_0^4=\frac{1}{2}[e^4-e^0]=\frac{1}{2}[e^4-1]\]
Here you will notice the second key difference between evaluating a definite versus indefinite integral using the method of substitution: we do not need to back-substitute! Since we changed the limits of integration to be $u$-values, once we complete all of our substitution, we are able to evaluate the integral as normal. 


Notice, that we have just shown that

\[
\int_0^2 x e^{x^2}\d x=\frac{1}{2}\int_{0}^{4} e^{u}\d u.
\]
  The figure below illustrates both integrals.
\begin{image}
  \begin{tikzpicture}
    \begin{axis}[
        xmin=-0.3,xmax=4.2,ymin=-0.3,ymax=3.5,
        clip=true,
        unit vector ratio*=2 2 2,
        axis lines=center,
        grid = major,
         width=3in,
        height=3in,
        ytick={0,3.125},
        xtick={0,2,4},
         yticklabels={0,100},
        xlabel=$x$, ylabel=$y$,
        every axis y label/.style={at=(current axis.above origin),anchor=south},
        every axis x label/.style={at=(current axis.right of origin),anchor=west},
      ]
     \addplot[fill=penColor,fill opacity=0.3,draw=none,domain=0:{2},samples=50]  {((1/32)*x*e^(x^2)} \closedcycle;;   
      \addplot[thick,penColor,domain=0:{2},samples=150] {((1/32)*x*e^(x^2)};   
  % \addplot [draw=none,fill=fillp,domain=8:9, smooth] {f(x)} \closedcycle;
	  \node at (axis cs:1.75,0.55) { $A$};  
	   \node at (axis cs:3,2.55) { $A=\int_0^2 xe^{x^2}\d x$};       
      \node at (axis cs:1.2,1.75) {$y=xe^{x^2}$};
      \end{axis}`
  \end{tikzpicture}
  \begin{tikzpicture}
	  \begin{axis}[
        xmin=-0.3,xmax=4.2,ymin=-0.3,ymax=3.5,
        clip=true,
        unit vector ratio*=2 2 2,
        axis lines=center,
        grid = major,
         width=3in,
        height=3in,
        ytick={0,3.125},
        xtick={0,2,4},
         yticklabels={0,100},
        xlabel=$u$, ylabel=$y$,
        every axis y label/.style={at=(current axis.above origin),anchor=south},
        every axis x label/.style={at=(current axis.right of origin),anchor=west},
      ]
      \addplot[fill=penColor,fill opacity=0.3,draw=none,domain=0:{4},samples=50] {(1/64)*e^x} \closedcycle;
      \addplot[thick,draw=penColor,domain=0:{4},samples=50] {(1/64)*e^x};

	  \node at (axis cs:3.6,0.3) { $B$}; 
	   \node at (axis cs:1,2.55) {$B=\int_0^4 \frac{1}{2}e^{u}\d u$};           
      \node at (axis cs:2.8,0.7) {$y=\frac{1}{2}e^u$};
      \end{axis}`
  \end{tikzpicture}
\end{image}
The two shaded areas in the figure, $A= \int_0^2 x e^{x^2}\d x$ and $B=\frac{1}{2}\int_{0}^{4} e^{u}\d u$, appear to be equal,
and this confirms what was established by computation before.
  
  
Therefore, whenever we are faced with a problem of evaluating a difficult integral, we try to  replace it with  a simpler one, as long as  both  integrals represent the same net area.


\end{explanation}
\end{example}

\section{More examples}

Let's continue sharpening our substitution skills with additional practice.


\begin{example}
Compute:
\[
\int_{-1}^0 12x^3 \cos({x^4}) \d x
\]
\begin{explanation}
Let's start by selecting our $u$:
\[
u = x^4.
\]
Then
\[
\d u = \answer[given]{4x^3} \d x 	\qquad	\Rightarrow	\qquad	\frac{1}{4}\d u = \answer[given]{x^3} \d x.
\]
Substituting our endpoints, we get
\[u(-1)=(-1)^4=\answer[given]{1}\]
\[u(0)=(0)^4=\answer[given]{0}\]
\begin{align*}
\int_{-1}^0 12x^3 \cos({x^4}) \d x &= \int_{\answer[given]{1}}^{\answer[given]{0}} \frac{1}{4}(12\cos({u})) \d u  \\
&= \int_{\answer[given]{1}}^{\answer[given]{0}} 3 \cos({u}) \d u  \\
&= \eval{3\sin({u})}_{\answer[given]{1}}^{\answer[given]{0}}  \\
&= \answer[given]{3(0-\sin({1}))}.
\end{align*}
\end{explanation}
\end{example}




\begin{example}
Compute:
\[
\int_{2}^{3} \frac{1}{x\ln(x)} \d x
\]
\begin{explanation}
  Let
  \[
  u =\answer[given]{\ln(x)},
  \]
  computing $\d u$, we find
  \[
  \d u =\answer[given]{\frac{1}{x}}\d x.
  \]
Notice that we do not need to adjust any further since $\frac{1}{x}$ is present in our integrand. Meaning that we can do directly substitute $\d u$ for $\frac{1}{x} \d x$.

Now for the limits of integration,
\[u(2)=\answer[given]{\ln(2)}\]
\[u(3)=\answer[given]{\ln(3)}\]
  Now
\begin{align*}
\int_{2}^{3} \frac{1}{x\ln(x)} \d x &=  \int_{\answer[given]{\ln(2)}}^{\answer[given]{\ln(3)}} \answer[given]{\frac{1}{u}} \d u\\
&= \eval{\answer[given]{\ln(u)}}_{\answer[given]{\ln(2)}}^{\answer[given]{\ln(3)}}\\
& = \ln(\ln(3)) - \ln(\ln(2)).
\end{align*}
\end{explanation}
\end{example}

 Sometimes in the course of evaluating an integral with the method of substitution, you will find that completing one substitution leads to an integral that requires a second one. In those cases, be sure to use different variable names for each substitution.
\begin{example}
Compute:
\[
\int_0^{16} \sqrt{4 - \sqrt{x}} \d x
\]
\begin{explanation}
While it is not obvious at all, let us try the substitution
\[
u = \sqrt{x}.
\]
Then
\begin{align*}
\d u &= \answer[given]{\frac{1}{2 \sqrt{x}}} \d x=\frac{1}{2u}dx,\\
2u \d u &= \d x.
\end{align*}
Then, the limits of integration will be adjusted:
\[u(0)=\sqrt{0}=\answer[given]{0}\]
\[u(16)=\sqrt{16}=\answer[given]{4}.\]
Substituting in, we get,

\[\int_0^{16} \sqrt{4 - \sqrt{x}} \d x =\int_{\answer[given]{0}}^{\answer[given]{4}} \answer[given]{2u \sqrt{4-u}} \d u.\]

From here we now make the second (and more obvious) substitution
\[
w = 4-u.
\]
Then $u = 4-w$, and
\begin{align*}
\d w &= - \d u,\\
-\d w &=  \d u.
\end{align*}
Again, changing the limits of integration again, we get
\[w(0)=\answer[given]{4}\]
\[w(4)=\answer[given]{0}\]
So
\begin{align*}
\int_0^{16} \sqrt{4 - \sqrt{x}} \d x &= \int_0^4 2u \sqrt{4-u} \d u  \\
&= \int_{\answer[given]{4}}^{\answer[given]{0}} 2 (4-w) \sqrt{w} (-1) \d w  \\
&= - \int_{\answer[given]{4}}^{\answer[given]{0}} ( 8w^{\frac{1}{2}} - 2w^{\frac{3}{2}} ) \d w  \\
&= \int_{\answer[given]{0}}^{\answer[given]{4}} ( 8w^{\frac{1}{2}} - 2w^{\frac{3}{2}} ) \d w  \\
&= \eval{ \answer[given]{8 \left( \frac{2}{3} \right) w^{\frac{3}{2}} - 2 \left( \frac{2}{5} \right) w^{\frac{5}{2}}} }_{0}^{4}  \\
&= \left( \frac{16}{3} (4)^{\frac{3}{2}} - \frac{4}{5} (4)^{\frac{5}{2}} \right) - \left( 0 - 0 \right)  \\
&= \frac{128}{3} - \frac{128}{5}   \\
&= \frac{256}{15}.
\end{align*}
\end{explanation}
\end{example}



\end{document}